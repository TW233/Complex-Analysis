\ifx\allfiles\undefined
\documentclass[12pt, a4paper,oneside, UTF8]{ctexbook}
\usepackage[dvipsnames]{xcolor}
\usepackage{amsmath}   % 数学公式
\usepackage{amsthm}    % 定理环境
\usepackage{amssymb}   % 更多公式符号
\usepackage{graphicx}  % 插图
%\usepackage{mathrsfs}  % 数学字体
%\usepackage{newtxtext,newtxmath}
%\usepackage{arev}
\usepackage{kmath,kerkis}
\usepackage{newtxtext}
\usepackage{bbm}
\usepackage{enumitem}  % 列表
\usepackage{geometry}  % 页面调整
%\usepackage{unicode-math}
\usepackage[colorlinks,linkcolor=black]{hyperref}


\usepackage{ulem}	   % 用于更多的下划线格式,
					   % \uline{}下划线,\uuline{}双下划线,\uwave{}下划波浪线,\sout{}中间删除线,\xout{}斜删除线
					   % \dashuline{}下划虚线,\dotuline{}文字底部加点


\graphicspath{ {flg/},{../flg/}, {config/}, {../config/} }  % 配置图形文件检索目录
\linespread{1.5} % 行高

% 页码设置
\geometry{top=25.4mm,bottom=25.4mm,left=20mm,right=20mm,headheight=2.17cm,headsep=4mm,footskip=12mm}

% 设置列表环境的上下间距
\setenumerate[1]{itemsep=5pt,partopsep=0pt,parsep=\parskip,topsep=5pt}
\setitemize[1]{itemsep=5pt,partopsep=0pt,parsep=\parskip,topsep=5pt}
\setdescription{itemsep=5pt,partopsep=0pt,parsep=\parskip,topsep=5pt}

% 定理环境
% ########## 定理环境 start ####################################
\theoremstyle{definition}
\newtheorem{defn}{\indent 定义}[section]

\newtheorem{lemma}{\indent 引理}[section]    % 引理 定理 推论 准则 共用一个编号计数
\newtheorem{thm}[lemma]{\indent 定理}
\newtheorem{corollary}[lemma]{\indent 推论}
\newtheorem{criterion}[lemma]{\indent 准则}

\newtheorem{proposition}{\indent 命题}[section]
\newtheorem{example}{\indent \color{SeaGreen}{例}}[section] % 绿色文字的 例 ,不需要就去除\color{SeaGreen}{}
\newtheorem*{rmk}{\indent \color{red}{注}}

% 两种方式定义中文的 证明 和 解 的环境:
% 缺点:\qedhere 命令将会失效【技术有限,暂时无法解决】
\renewenvironment{proof}{\par\textbf{证明.}\;}{\qed\par}
\newenvironment{solution}{\par{\textbf{解.}}\;}{\qed\par}

% 缺点:\bf 是过时命令,可以用 textb f等替代,但编译会有关于字体的警告,不过不影响使用【技术有限,暂时无法解决】
%\renewcommand{\proofname}{\indent\bf 证明}
%\newenvironment{solution}{\begin{proof}[\indent\bf 解]}{\end{proof}}
% ######### 定理环境 end  #####################################

% ↓↓↓↓↓↓↓↓↓↓↓↓↓↓↓↓↓ 以下是自定义的命令  ↓↓↓↓↓↓↓↓↓↓↓↓↓↓↓↓

% 用于调整表格的高度  使用 \hline\xrowht{25pt}
\newcommand{\xrowht}[2][0]{\addstackgap[.5\dimexpr#2\relax]{\vphantom{#1}}}

% 表格环境内长内容换行
\newcommand{\tabincell}[2]{\begin{tabular}{@{}#1@{}}#2\end{tabular}}

% 使用\linespread{1.5} 之后 cases 环境的行高也会改变,重新定义一个 ca 环境可以自动控制 cases 环境行高
\newenvironment{ca}[1][1]{\linespread{#1} \selectfont \begin{cases}}{\end{cases}}
% 和上面一样
\newenvironment{vx}[1][1]{\linespread{#1} \selectfont \begin{vmatrix}}{\end{vmatrix}}

\def\d{\textup{d}} % 直立体 d 用于微分符号 dx
\def\R{\mathbb{R}} % 实数域
\def\N{\mathbb{N}} % 自然数域
\def\C{\mathbb{C}} % 复数域
\def\Z{\mathbb{Z}} % 整数环
\def\Q{\mathbb{Q}} % 有理数域
\newcommand{\bs}[1]{\boldsymbol{#1}}    % 加粗,常用于向量
\newcommand{\ora}[1]{\overrightarrow{#1}} % 向量

% 数学 平行 符号
\newcommand{\pll}{\kern 0.56em/\kern -0.8em /\kern 0.56em}

% 用于空行\myspace{1} 表示空一行 填 2 表示空两行  
\newcommand{\myspace}[1]{\par\vspace{#1\baselineskip}}

%s.t. 用\st就能打出s.t.
\DeclareMathOperator{\st}{s.t.}

%罗马数字 \rmnum{}是小写罗马数字, \Rmnum{}是大写罗马数字
\makeatletter
\newcommand{\rmnum}[1]{\romannumeral #1}
\newcommand{\Rmnum}[1]{\expandafter@slowromancap\romannumeral #1@}
\makeatother
\begin{document}
	% \title{{\Huge{\textbf{$Complex \,\, Analysis$\footnote{课堂教材:\textbf{《$Complex \,\, Analysis$》---  $Elias \,\, M. \,\, Stein$}}}}}}
\author{$-TW-$}
\date{\today}
\maketitle                   % 在单独的标题页上生成一个标题

\thispagestyle{empty}        % 前言页面不使用页码
\begin{center}
	\Huge\textbf{序}
\end{center}


\vspace*{3em}
\begin{center}
	\large{\textbf{天道几何,万品流形先自守;}}\\
	
	\large{\textbf{变分无限,孤心测度有同伦。}}
\end{center}

\vspace*{3em}
\begin{flushright}
	\begin{tabular}{c}
		\today \\ \small{\textbf{长夜伴浪破晓梦,梦晓破浪伴夜长}}
	\end{tabular}
\end{flushright}


\newpage                      % 新的一页
\pagestyle{plain}             % 设置页眉和页脚的排版方式(plain:页眉是空的,页脚只包含一个居中的页码)
\setcounter{page}{1}          % 重新定义页码从第一页开始
\pagenumbering{Roman}         % 使用大写的罗马数字作为页码
\tableofcontents              % 生成目录

\newpage                      % 以下是正文
\pagestyle{plain}
\setcounter{page}{1}          % 使用阿拉伯数字作为页码
\pagenumbering{arabic}
\setcounter{chapter}{-1}    % 设置 -1 可作为第零章绪论从第零章开始 
	\else
	\fi
	%  ############################ 正文部分

\chapter{$Week \,\, 2 \,\, -- \,\, Functions \,\, on \,\, \C$}
\section{连续函数和极值}
\begin{defn}\label{def 2.1.1}
	$Let \,\, \Omega \subseteq \C \,\, be \,\, open. \,\, We \,\, say \,\, f : \Omega \longrightarrow \C \,\, is \,\, continuous \,\, at \,\, z_0 \in \Omega \,\, $\\
	$if \,\, \forall \epsilon > 0 , \exists \delta > 0 , \st$
	\begin{align}
		whenever \,\, \left| z - z_0 \right| < \delta , \,\, z \in \Omega , \,\, then \,\, \left| f(z) - f(z_0) \right| < \epsilon
	\end{align}
	$To \,\, say \,\, it \,\, another \,\, way , \,\, \forall \epsilon > 0 , \exists \delta > 0 , \st \,\, f(D_{\delta}(z_0) \cap \Omega) \subseteq D_{\epsilon}(f(z_0))$
	
	\begin{rmk}
		$We \,\, say \,\, f \,\, is \,\, continuous \,\, on \,\, \Omega \,\, if \,\, f \,\, is \,\, continuous \,\, at \,\, every \,\, point \,\, of \,\, \Omega.$
	\end{rmk}
\end{defn}

\vspace*{2em}
$Here \,\, are \,\, some \,\, facts.$
\begin{enumerate}
	\item[$Fact \,\, 1.$] $If \,\, f \,\, is \,\, continuous \,\, on \,\, \Omega , \,\, then \,\, so \,\, are \,\, \overline{f} , \,\, \left| f \right| , \,\, \frac{1}{f} \,\, (if \,\, f(z) \neq 0 \,\, for \,\, all \,\, z \in \Omega).$
	\begin{proof}
		$For \,\, \left| f \right| , \,\, use \,\, \left| \left| f(z) \right| - \left| f(z_0) \right| \right| \leq \left| f(z) - f(z_0) \right|$
	\end{proof}
	
	\item[$Fact \,\, 2.$]$f \,\, is \,\, continuous \,\, iff \,\, Ref \,\, and \,\, Imf \,\, are \,\, continuous.$
\end{enumerate}

\vspace*{2em}
\begin{proposition}\label{prop 2.1.1}
	$Let \,\, \Omega \subseteq \C \,\, and \,\, let \,\, f \,\, be \,\, continuous \,\, on \,\, \Omega . \,\, Then$
	\begin{enumerate}
		\item[(1)]$For \,\, every \,\, open \,\, set \,\, S \subseteq \C , \,\, f^{-1}(S) = \{ z \in \Omega \mid f(z) \in S \} \,\, is \,\, open.$
		
		\item[(2)]$For \,\, every \,\, compact \,\, set \,\, K \subseteq \C , \,\, f(K) \,\, is \,\, compact.$
	\end{enumerate}
	
	\vspace*{2em}
	\begin{proof}
		\begin{enumerate}
			\item[(1)]$If \,\, f^{-1}(S) = \varnothing , \,\, true.$\\
			$Assume \,\, f^{-1}(S) \neq \varnothing \,\, and \,\, let \,\, z_0 \in f^{-1}(S) . \,\, Write \,\, w_0 = f(z_0) \in S.$\\
			$Since \,\, S \,\, is \,\, open , \,\, \exists \epsilon > 0 , \st \,\, D_{\epsilon}(w_0) \subseteq S$\\
			$Since \,\, f \,\, is \,\, continuous , \,\, taking \,\, \epsilon \,\, in \,\, the \,\, definition , \,\, we \,\, get \,\, a \,\, \delta > 0 , \st$
			\begin{align}
				D_{\delta}(z_0) \subseteq \Omega \,\, and \,\, f(D_{\delta}(z_0)) \subseteq D_{\epsilon}(f(z_0)) = D_{\epsilon}(w_0) \subseteq S
			\end{align}
			$Thus \,\, D_{\delta}(z_0) \subseteq f^{-1}(S) , \,\, and \,\, so \,\, f^{-1}(S) \,\, is \,\, open.$
			
			\vspace*{2em}
			
			\item[(2)]$Let \,\, \{ \Omega_j \}_{j \in J} \,\, be \,\, an \,\, open \,\, cover \,\, of \,\, f(K) , \,\, i.e.$
			\begin{align}
				f(K) \subseteq \bigcup_{j \in J}{\Omega_j}
			\end{align}
			$Then$
			\begin{align}
				K \subseteq f^{-1}(\bigcup_{j \in J}{\Omega_j}) = \bigcup_{j \in J}{f^{-1}(\Omega_j)}
			\end{align}
			$By \,\, (1) , \,\, f^{-1}(\Omega_j) \,\, is \,\, open \,\, for \,\, all \,\, j \in J. \,\, Thus \,\, \{ f^{-1}(\Omega_j) \}_{j \in J} \,\, is \,\, an \,\, open \,\, cover \,\, of \,\, K.$\\
			$Since \,\, K \,\, is \,\, compact , \,\, \exists j_1 , \cdots , j_n \in J , \st $
			\begin{align}
				&k \subseteq \bigcup_{k = 1}^{n}{f^{-1}(\Omega_{j_k})} = f^{-1}(\bigcup_{k = 1}^{n}{\Omega_{j_k}}) \\
				&\Rightarrow f(K) \subseteq \bigcup_{k = 1}^{n}{\Omega_{j_k}}
			\end{align}
		\end{enumerate}
	\end{proof}
\end{proposition}

\vspace*{2em}
$We \,\, say \,\, that \,\, f \,\, contains \,\, a \,\, maximum \,\, at \,\, z_0 \in \Omega \,\, if $
\begin{align}
	\left| f(z) \right| \leq \left| f(z_0) \right| , \,\, \forall z \in \Omega
\end{align}

\begin{proposition}\label{prop 2.1.2}
	$A \,\, continuous \,\, function \,\, on \,\, a \,\, compact \,\, set \,\, \Omega \,\, is \,\, bounded \,\, and \,\, attains \,\, a \,\, maximum $\\ 
	$and \,\, a \,\, minimum \,\, on \,\, \Omega.$
	
	\vspace*{2em}
	\begin{proof}
		$use \,\, \left| f \right|^2 = (Ref)^2 + (Imf)^2$.
	\end{proof}
\end{proposition}

\newpage
\section{复变函数的极限,全纯函数}
\begin{defn}\label{def 2.2.1}
	$Assume \,\, \Omega \subseteq \C , \,\, \Omega \neq \varnothing \,\, and \,\, \alpha \in Acc(\Omega) , \,\, f : \Omega \longrightarrow \C , \,\, \underset{z \to \alpha , \, z \in \Omega}{\lim}{f(z)} = w \,\, means$
	\begin{align}
		\forall \epsilon > 0 , \exists \delta > 0 , \st \,\, 0 < \left| z - z_0 \right| < \delta \,\, \Rightarrow \,\, \left| f(z) - w \right| < \epsilon
	\end{align}
	
	\begin{rmk}
		容易证明若极限存在,则极限唯一.
	\end{rmk}
\end{defn}

\vspace*{2em}
\begin{defn}\label{def 2.2.2}
	$Let \,\, \Omega \subseteq \C \,\, be \,\, open , \,\, f : \Omega \longrightarrow \C . \,\, We \,\, say \,\, f(z) \,\, is \,\, \underline{\textbf{$Complex \,\, differentiable \,\, at \,\, z_0 \in \Omega$}}$\\
	$if \,\, \underset{h \to 0}{\lim}{\frac{f(z_0 + h) - f(z_0)}{h}} \,\, exists. \,\, If \,\, f \,\, is \,\, complex \,\, differentiable at \,\, z_0 , \,\, we \,\, denote \,\, the \,\, limit \,\, of \,\, the \,\, quotient $\\ 
	$by \,\, f^{'}(z_0) . \,\, i.e.$
	\begin{align}
		f^{'}(z_0) = \lim_{h \to 0}{\frac{f(z_0 + h) - f(z_0)}{h}}
	\end{align}
	$f^{'}(z_0) \,\, is \,\, called \,\, \underline{\textbf{$the \,\, derivative \,\, of \,\, f \,\, at \,\, z_0$}}$.
	
	\begin{rmk}
		$If \,\, f \,\, is \,\, complex \,\, differentiable \,\, at \,\, every \,\, point \,\, of \,\, \Omega , \,\, then \,\, we \,\, say \,\, f \,\, is \,\, \underline{\textbf{$holomorphic$}} \,\, on \,\,  \Omega.$
	\end{rmk}
\end{defn}

\vspace*{2em}
\begin{example}\label{ex 2.2.1}
	\begin{itemize}
		\item $f(z) = \frac{1}{z} \,\, is \,\, holomorphic \,\, on \,\, \C \backslash \{ 0 \}.$
		
		\item $f(z) = \bar{z} \,\, is \,\, not \,\, complex \,\, differentiable \,\, at \,\, any \,\, point \,\, of \,\, \C.$
		
		\item $f(z) = \left| z \right|^2 \,\, is \,\, only \,\, complex \,\, differentiable \,\, at \,\, z = 0.$
	\end{itemize}
\end{example}

\vspace*{2em}
\begin{align}
	\lim_{h \to 0}{\frac{f(z_0 + h) - f(z_0)}{h}} = f^{'}(z_0) \,\, \Leftrightarrow \,\, \lim_{h \to 0}{\frac{f(z_0 + h) - f(z_0) - hf^{'}(z_0)}{h}} = 0
\end{align}
$Let \,\, \underline{\textbf{$\circ(h)$}} \,\, denote \,\, \underline{\textbf{$any \,\, complexed \,\, valued \,\, function$}} \,\, with \,\, the \,\, property \,\, \frac{\circ(h)}{h} \to 0 , \,\, as \,\, h \to 0$\\
$Then \,\, f \,\, is \,\, complex \,\, differentiable \,\, at \,\, z_0 \,\, iff \,\, \exists a \in \C , \st$
\begin{align}
	f(z_0 + h) - f(z_0) - ha = \circ(h) , \,\, where \,\, a = f^{'}(z_0)
\end{align}
\begin{rmk}
	$According \,\, to \,\, equation(2.11) , \,\, holomorphic \,\, \Rightarrow \,\, continuity.$
\end{rmk}

\newpage
\begin{proposition}\label{prop 2.2.1}
	$If \,\, f , \,\, g \,\, are \,\, holomorphic \,\, on \,\, an \,\, open \,\, set \,\, \Omega \subseteq \C , \,\, then$
	\begin{align}
		(f + g)^{'} = f^{'} + g^{'} , \,\, (fg)^{'} = f^{'}g + fg^{'}
	\end{align}
	$If \,\, g(z_0) \neq 0 , \,\, then \,\, \frac{f}{g} \,\, is \,\, complex \,\, differentiable \,\, at \,\, z_0 \,\, and$
	\begin{align}
		\left( \frac{f}{g} \right)^{'}_{z = z_0} = \frac{f^{'}g - fg^{'}}{g^2}\Big|_{z = z_0}
	\end{align}
	$If \,\, f : \Omega \longrightarrow U \,\, and \,\, g : U \longrightarrow \C \,\, are \,\, holomorphic , \,\, then \,\, \underline{\textbf{$the \,\, chain \,\, rule$}} \,\, holds$
	\begin{align}
		(g \circ f)^{'}(z) = g^{'}(f(z)) \, f^{'}(z) , \,\, \forall z \in \Omega
	\end{align}
\end{proposition}

\newpage
\section{$Cauchy - Riemann \,\, Equations$}
\begin{align}
	f(z) = f(x + iy) = u(x , y) + iv(x , y)
\end{align}
$Assume \,\, \underset{h \to 0}{\lim}{\frac{f(z_0 + h) - f(z_0)}{h}} \,\, exists , \,\, we \,\, may \,\, let \,\, h \to 0 \,\, in \,\, whichever \,\, manner \,\, we \,\, please. $\\
$(let \,\, z_0 = x_0 + iy_0)$
\begin{itemize}
	\item $Let \,\, h = t \in \R , $
	\begin{align}
		f^{'}(z_0) = \lim_{t \to 0 , \,\, t \in \R}{\frac{f(z_0 + h) - f(z_0)}{h}} &= u_x (x_0 , y_0) + i v_x (x_0 , y_0) \\
		&= \frac{\partial u}{\partial x}(x_0 , y_0) + i\frac{\partial v}{\partial x}(x_0 , y_0)
	\end{align}
	
	\item $Let \,\, h = it , t \in \R ,$
	\begin{align}
		f^{'}(z_0) = \lim_{t \to 0 , \,\, t \in \R}{\frac{f(z_0 + h) - f(z_0)}{it}} &= v_y(x_0 , y_0) - iu_y(x_0 , y_0) \\
		&= \frac{\partial v}{\partial y}(x_0 , y_0) - i\frac{\partial u}{\partial y}(x_0 , y_0)
	\end{align}
\end{itemize}
$Thus , \,\, we \,\, conclude \,\, f = u + iv \,\, is \,\, holomorphic \,\, \Rightarrow \,\, u , \,\, v \,\, satisfy$
\begin{align}
	\begin{cases}
		u_x = v_x\\
		u_y = - v_y
	\end{cases}
\end{align}
$The \,\, equations(2.20) \,\, is \,\, called \,\, \underline{\textbf{$Cauthy - Riemann \,\, Equations$}}$.

\vspace*{2em}
\begin{example}\label{ex 2.3.1}
	$f(x + iy) = x^2 - y^2 - 2xyi , \,\, x , y \in \R \,\, is \,\, not \,\, holomorphic \,\, on \,\, \C \backslash \{ 0 \}.$
\end{example}

\newpage
\section{全纯条件}
Let $f = u + i v : \Omega \longrightarrow \C$ be holomorphic. Then
\begin{align}
	\begin{cases}
		u_x = v_y \\
		u_y = - v_x
	\end{cases} \,\, on \,\, \Omega
\end{align}

\vspace{2em}
下面给出函数$holomorphic$ 的充分条件.
\begin{thm}\label{thm 2.4.1}
	Let $\Omega \subset \C$ be open, $f = u + iv : \Omega \longrightarrow \C$. If $u , v$ are differentiable on $\Omega$ and satisfy the $Cauchy-Riemann \,\, equations$, then $f$ is holomorphic on $\Omega$.
	
	\vspace{2em}
	\begin{proof}
		(Goal : $\forall z_0 = x_0 + i y_0 \in \Omega , \,\, h = h_1 + i h_2 \in \C , \,\, z_0 + h \in \Omega , \,\, \left| h \right| \,\, small \,\, enough , $\\
		$f(z_0+  h) - f(z_0) = ah + \circ(h)$)\\
		
		\vspace{1em}
		Since $u(x , y)$ is differentiable on $\Omega$, 
		\begin{align}
			u(x_0 + h_1 , y_0 + h_2) - u(x_0 , y_0) = h_1 u_x(x_0 , y_0) + h_2 u_y(x_0 , y_0) + \circ(h_1 , h_2)
		\end{align}
		Here $\circ(h_1 , h_2)$ is any expression with the property that $\frac{\circ(h_1 , h_2)}{\sqrt{h_{1}^2 + h_{2}^2}} \to 0$ , as $(h_1 , h_2) \to 0$.\\
		Similarly, 
		\begin{align}
			v(x_0 + h_1 , y_0 + h_2) - v(x_0 , y_0) = h_1 v_x(x_0 , y_0) + h_2 v_y(x_0 , y_0) + \circ(h_1 , h_2)
		\end{align}
		Then
		\begin{align}
			f(z_0 + h) - f(z_0) 
			&= h_1 u_x + h_2 u_y + i (h_1 v_x + h_2 v_y) + \circ(h_1 , h_2) \\
			&= h_1 u_x - h_2 v_x + i (h_1 v_x + h_2 u_x) + \circ(h_1 , h_2) \\
			&= (u_x + i v_x)(h_1 + i h_2) + \circ(h_1 , h_2)
		\end{align}
		Note that we may write $\circ(h)$ instead of $\circ(h_1 , h_2)$, since
		\begin{align}
			(h_1 , h_2) \to 0 \Leftrightarrow h \to 0 \Leftrightarrow \left| h \right| \to 0
		\end{align}
		Then the previous expression is equal to $f^{'}(z_0)h + \circ(h)$.\\
		Since $z_0$ is arbitrary, $f$ is holomorphic on $\Omega$.
	\end{proof}
\end{thm}

\newpage
$f = u + iv$ can be seen as a mapping
\begin{align}
	F : \R^2 &\longrightarrow \R^2 \\
	(x , y) &\longmapsto (u(x , y) , v(x , y))
\end{align}
F is said to be differentiable at a point $P_0 = (x_0 , y_0)$, if $\exists$ a linear transformation $J : \R^2 \longrightarrow \R^2 , \,\, \st$
\begin{align}
	F(P_0 + H) - F(P_0) = J(H) + \left| H \right| \psi(H) , \,\, with \,\, \left| \psi(H) \right| \to 0 \,\, as \,\, \left| H \right| \to 0
\end{align}

\vspace{2em}
\begin{proposition}\label{prop 2.4.1}
	If $f$ is complex differentiable at $z_0 = x_0 + i y_0$, then $F$ is differentiable at $(x_0 , y_0)$.
	
	\vspace{2em}
	\begin{proof}
		Since $f$ is complex differentiable at $z_0 = x_0 + i y_0$, we have
		\begin{align}
			f(z_0 + h) - f(z_0) 
			&= f^{'}(z_0)h + \circ(h) \\
			&= (u_x + i v_x)(h_1 + i h_2) + \circ(h) \\
			&= u_x h_1 - v_x h_2 + i (v_x h_1 + u_x h_2) + \circ(h) \\
			&= u_x h_1 + u_y h_2 + i (v_x h_1 + v_y h_2) + \circ(h)
		\end{align}
		Thus, $F(P_0 + H) - F(P_0) = J(H) + \circ(H)$, where $J = 
		\begin{pmatrix}
			u_x \,\, &u_y \\
			v_x &v_y
		\end{pmatrix}$ and $H = (h_1 , h_2)$.
	\end{proof}
\end{proposition}

\newpage
\section{复变函数微分}
\begin{center}
	$z = x + i y , \,\, \overline{z} = x - i y \,\, \Leftrightarrow \,\, x = \frac{z + \overline{z}}{2} , \,\, y = \frac{z - \overline{z}}{2i}$
\end{center}
A given function $f : \Omega \longrightarrow \C$ can be expressed either in variables $x , y$ or $z , \overline{z}$. That is, for the given $f$, we may write $f(x , y)$ or $f(z , \overline{z})$.
\begin{rmk}
	可视作复平面上可建立两个坐标系$xOy$ 和$zO\overline{z}$,即$\C$ 中存在两组基.由于将复数$z$ 转化为$x + i y$ 后再进行计算常常会产生不便,因此下面通过这两组基之间的转化,探讨不同形式下函数微分的表达方式.
\end{rmk}

\vspace{2em}
Suppose the relevant derivatives exist.
\begin{align}
	\frac{\partial f}{\partial z} &= \frac{\partial f}{\partial x} \cdot \frac{\partial x}{\partial z} + \frac{\partial f}{\partial y} \cdot \frac{\partial y}{\partial z} = \frac{1}{2} \left( \frac{\partial}{\partial x} - i \frac{\partial}{\partial y} \right) f \\
	\frac{\partial f}{\partial \overline{z}} &= \frac{\partial f}{\partial x} \cdot \frac{\partial x}{\partial \overline{z}} + \frac{\partial f}{\partial y} \cdot \frac{\partial y}{\partial \overline{z}} = \frac{1}{2} \left( \frac{\partial}{\partial x} + i \frac{\partial}{\partial y} \right) f
\end{align}
Define two operations.$\left( Wirtinger \,\, operations , \,\, 1927 \right)$
\begin{align}
	\frac{\partial}{\partial z} \coloneqq \frac{1}{2} \left( \frac{\partial}{\partial x} - i \frac{\partial}{\partial y} \right) \\
	\frac{\partial}{\partial \overline{z}} \coloneqq \frac{1}{2} \left( \frac{\partial}{\partial x} + i \frac{\partial}{\partial y} \right)
\end{align}

\vspace{2em}
\begin{proposition}\label{prop 2.5.1}
	$Cauchy - Riemann \,\, equations$ are equivalent to
	\begin{align}
		\frac{\partial f}{\partial \overline{z}} = 0
	\end{align}
	
	\vspace{2em}
	\begin{proof}
		Let $f = u + i v$. Then 
		\begin{align}
			\frac{\partial f}{\partial \overline{z}} = \frac{1}{2} \left( \frac{\partial f}{\partial x} + i \frac{\partial f}{\partial y} \right) = \frac{1}{2} \left( u_x + v_x + i (u_y + v_y) \right) = \frac{1}{2} \left( u_x - v_y + i (u_y + v_x) \right)
		\end{align}
		\begin{align}
			\frac{\partial f}{\partial \overline{z}} = 0 \Leftrightarrow 
			\begin{cases}
				u_x = v_y \\
				u_y = - v_x
			\end{cases}
		\end{align}
	\end{proof}
\end{proposition}

\begin{rmk}
	We note that $f^{'}(z) = u_x + i v_x = u_x - i u_y = 2 \frac{\partial u}{\partial z}$.
\end{rmk}

\newpage
\paragraph{调和算子 / 拉普拉斯算子}
Define the \underline{$Laplacian$}(or the \underline{$Laplace \,\, operator$}).
\begin{align}
	\Delta = \frac{\partial^2}{\partial x^2} + \frac{\partial^2}{\partial y^2}
\end{align}

\begin{rmk}
	$C^{k}(\Omega)$ denotes the set of all k times continuously differentiable functions on $\Omega$.
\end{rmk}

\vspace{2em}
下面给出调和函数的定义.
\begin{defn}\label{def 2.5.1}
	Let $\Omega \subset \C$ be an open set. $g : \Omega \longrightarrow \C$ is called \underline{$harmonic$} if $g \in C^{2}(\Omega)$ and $\Delta g = 0$.
\end{defn}

\vspace{2em}
下面的命题说明了全纯函数的实部和虚部均调和.(全纯的必要条件)
\begin{proposition}\label{prop 2.5.2}
	Let $f = u + i v : \Omega \longrightarrow \C$ be holomorphic. Assume $u , v \in C^{2}(\Omega)$. Then $u , v$ are harmonic.
	
	\begin{rmk}
		事实上后面会证明此处无需$u , v \in C^{2}(\Omega)$.
	\end{rmk}
	
	\vspace{2em}
	\begin{proof}
		The $Cauchy - Riemann \,\, equations$ tell $
		\begin{cases}
			u_x = v_y \\
			u_y = - v_x
		\end{cases}$
		\begin{align}
			\frac{\partial^2 u}{\partial x^2} &= \frac{\partial^2 v}{\partial x \partial y} \\
			\frac{\partial^2 u}{\partial y^2} &= -\frac{\partial^2 v}{\partial y \partial x}
		\end{align}
		Since $v \in C^{2}(\Omega)$, 
		\begin{align}
			\frac{\partial^2 v}{\partial x \partial y} = \frac{\partial^2 v}{\partial y \partial x}
		\end{align}
		Therefore, $u_{xx} + u_{yy} = 0$. Similarly we can proof that $v_{xx} + v_{yy} = 0$.
	\end{proof}
\end{proposition}

\vspace{2em}
A holomorphic function is necessarily harmonic, so is $\overline{f}$.
\begin{proposition}\label{prop 2.5.3}
	Let $\Omega \subset \C$ be a region, $f : \Omega \longrightarrow \C$. Then
	\begin{center}
		$f$ is consistant iff $f^{'}(z) = 0 , \forall z \in \Omega$.
	\end{center}
	
	\vspace{2em}
	\begin{proof}
		\begin{enumerate}
			\item[$\Rightarrow:$]clear
			
			\item[$\Leftarrow:$]Let $f = u + i v$, then
			\begin{align}
				f^{'}(z) = 0 \Rightarrow u_x + i v_x = 0 &\Rightarrow u_x = 0 , v_x = 0 \\
				&\overset{C-R}{\Rightarrow} v_y = 0 , u_y = 0 \\
				&\Rightarrow u = c_1 , v = c_2 \,\, (by \,\, mean \,\, value \,\, theorem) \\
				&(\text{区域连通,利用中值定理})
			\end{align}
		\end{enumerate}
	\end{proof}
\end{proposition}

\newpage
\section{课堂例题$2024-03-08$}
\begin{enumerate}
	\item $f(x + i y) = x^2 - y^2 + 2xy i$ is holomorphic.
	
	\vspace{2em}
	
	\item Is $f(z) = z^2 \overline{z} + \frac{1}{z} + \frac{1}{z^2}$ holomorphic on $\C \backslash \{ 0 \}$?
	
	\vspace{2em}
	
	\item Let $f = u + i v$ be holomorphic on a region $\Omega$. Assume $au + bv + c = 0$ for some $a , b , c \in \R$ and $a , b$ are not all zero. Show $f$ is consistant.
	
	\vspace{2em}
	
	\item Find a holomorphic function $f$ on $\C$ $\st$
	\begin{align}
		Re f = x^2 - y^2 + xy , \,\, f(0) = 0
	\end{align}
	
	\vspace{2em}
	
	\item Let $\Omega = \C \backslash \{ 0 \}$ and $u : \Omega \longrightarrow \R$ be given by $u(x , y) = \frac{1}{2} \ln(x^2 + y^2)$.\\
	Is there a holomorphic function $f : \Omega \longrightarrow \C$, $\st$ $Ref = u$ ?
	
	\vspace{2em}
	\begin{solution}
		Suppose $f = u + i v$ is holomorphic on $\Omega$. Then
		\begin{align}
			\begin{cases}
				v_x = -u_y = -\frac{y}{x^2 + y^2} \\
				v_y = u_x = \frac{x}{x^2 + y^2}
			\end{cases}
		\end{align}
		By $v_y = \frac{x}{x^2 + y^2}$,
		\begin{align}
			v = \arctan{\frac{y}{x}} + c(x)
		\end{align}
		Then by $v_x = -\frac{y}{x^2 + y^2}$, $c(x) = c$ is constant. $\Rightarrow$ $v = \arctan{\frac{y}{x}} + c$.\\
		However, $\arctan{\frac{y}{x}} : \R^2 \longrightarrow (-\pi , \pi]$ is not continuous on $\R_{\leq 0} = \{ x \leq 0 \mid x \in \R \}$.
		\begin{center}
			(Let $z = x + iy$, then $\arctan{\frac{y}{x}}$ is an argument of z.)
		\end{center}
		Therefore, there is no function satisfying the conditions.
		\begin{rmk}
			If the region $\Omega = \C \backslash \{ 0 \}$ is replaced by $\Omega = \C \backslash \R_{\leq 0}$, then the answer is yes.
		\end{rmk}
	\end{solution}
	
	\vspace{2em}
	
	\item 课本第一章练习$T8 , T9 , T10 , T13$.
\end{enumerate}






	%  ############################
	\ifx\allfiles\undefined
\end{document}
\fi