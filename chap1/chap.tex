\ifx\allfiles\undefined
\input{../config/config}
\begin{document}
	% \title{{\Huge{\textbf{$Complex \,\, Analysis$\footnote{课堂教材:\textbf{《$Complex \,\, Analysis$》---  $Elias \,\, M. \,\, Stein$}}}}}}
\author{$-TW-$}
\date{\today}
\maketitle                   % 在单独的标题页上生成一个标题

\thispagestyle{empty}        % 前言页面不使用页码
\begin{center}
	\Huge\textbf{序}
\end{center}


\vspace*{3em}
\begin{center}
	\large{\textbf{天道几何,万品流形先自守;}}\\
	
	\large{\textbf{变分无限,孤心测度有同伦。}}
\end{center}

\vspace*{3em}
\begin{flushright}
	\begin{tabular}{c}
		\today \\ \small{\textbf{长夜伴浪破晓梦,梦晓破浪伴夜长}}
	\end{tabular}
\end{flushright}


\newpage                      % 新的一页
\pagestyle{plain}             % 设置页眉和页脚的排版方式(plain:页眉是空的,页脚只包含一个居中的页码)
\setcounter{page}{1}          % 重新定义页码从第一页开始
\pagenumbering{Roman}         % 使用大写的罗马数字作为页码
\tableofcontents              % 生成目录

\newpage                      % 以下是正文
\pagestyle{plain}
\setcounter{page}{1}          % 使用阿拉伯数字作为页码
\pagenumbering{arabic}
\setcounter{chapter}{-1}    % 设置 -1 可作为第零章绪论从第零章开始 
	\else
	\fi
	%  ############################ 正文部分

\chapter{课程要求}
	\begin{itemize}
		\item \textbf{任课教师}:林明华
		
		\item \textbf{辅导时间}:周一$9a.m. - 11a.m.$
		
		\item \textbf{办公室}:数学楼$210$
		
		\item \textbf{$Email$}:$mh.lin@xjtu.edu.cn$
		
		\item \textbf{总评成绩组成}:阅读报告及汇报$20\%$ + 期末考试$80\%$
	\end{itemize}

\chapter{$Week \,\, 1$}
\section{复数的引入}
\paragraph{引入}
	\begin{center}
		下面从代数结构($Group , \,\, Ring , \,\, Field$)的角度引入复数的概念.
	\end{center}
	$Consider \,\, the \,\, set \,\, \R^2 . \,\, Define \,\, two \,\, operations . \,\, \forall (a , b) , (c , d) \in \R^2 , $
	\begin{align}
		(a , b) + (c , d) &\coloneqq (a + c , b + d)\\
		(a , b) \cdot (c , d) &\coloneqq (ac - bd , bc + ad)
	\end{align}
	$"\cdot" \,\, is \,\, commutative.$\\
	$"+" , \,\, "\cdot" \,\, satisfy \,\, associative \,\, and \,\, distributive \,\, laws.$
	\begin{align}
		(0 , 0) &: The \,\, additive \,\, identity \\
		(1 , 0) &: The \,\, multiplicative \,\, identity
	\end{align}
	$\Rightarrow (\R^2 , \,\, + , \,\, \cdot) \,\, is \,\, a \,\, communicative \,\, ring.$
	
	\vspace*{2em}
	$\forall (a , b) \in \R^2 , \,\, (a , b) \neq (0 , 0) , \,\, if$
	\begin{align}
		&(a , b) \cdot (x , y) = (1 , 0) \\
		&\Rightarrow x = \frac{a}{a^2 + b^2} , \,\, y = \frac{-b}{a^2 + b^2}
	\end{align}
	$Therefore , \,\, (\R^2 , \,\, + , \,\, \cdot) \,\, is \,\, a \,\, field , \,\, renoted \,\, as \,\, \C.$
	
\vspace*{2em}
\paragraph{复数的乘法}
	在上述对$\C$ 的定义中,唯一非平凡的点便是乘法运算$"\cdot"$ 的定义.
\begin{center}
	下面我们从代数的方法,从另一个角度理解复数的乘法.
\end{center}
	$We \,\, may \,\, ask \,\, a \,\, question \,\, : \,\, \uwave{Can \,\, we \,\, define \,\, a \,\, "\cdot" \,\, and  \,\, let \,\, (\R^3 , \,\, + , \,\, \cdot) \,\, be \,\, a \,\, field?}$\\
	$However , \,\, the \,\, answer \,\, is \,\, certainly \,\, not !$
	
	\vspace*{2em}
		$Consider \,\, M_2 = 
		\left\{ 
		\begin{pmatrix}
			a \,\, &-b\\
			b \,\, &a
		\end{pmatrix} \,\, \Bigg| \,\, a , b \in \R
		\right\}
		\,\, equipped \,\, with \,\, the \,\, usual \,\, matrix \,\, addition \,\, and \,\, multiplication.$\\
		
	$Define \,\, a \,\, map \,\, \sigma.$
	\begin{align}
		\sigma : \R^2 &\longrightarrow M_2 \\
		(a , b) &\longmapsto 
		\begin{pmatrix}
			a \,\, &-b\\
			b \,\, &a
		\end{pmatrix}
	\end{align}
	$Then , \,\, \sigma \,\, is \,\, bijective.$
	\begin{align}
		\sigma(a , b) \cdot \sigma(c , d) = 
		\begin{pmatrix}
			a \,\, &-b\\
			b \,\, &a
		\end{pmatrix} 
		\begin{pmatrix}
			c \,\, &-d\\
			d \,\, &c
		\end{pmatrix}
		= 
		\begin{pmatrix}
			ac - bd \,\, &-(bc + ad)\\
			bc + ad \,\, &ac - bd
		\end{pmatrix}
		= 
		\sigma((a , b) \cdot (c , d))
	\end{align}
	$\Rightarrow \sigma \,\, is \,\, an \,\, isomorphism(\text{同构映射}).$ \\
	于是复数乘法可视作复平面上带伸缩的旋转.	
	
\newpage
\section{复数的基本性质}
\paragraph{$Some \,\, Facts$}
	\begin{align}
		\left| Rez \right| \leq \left| z \right| &, \,\, \left| Imz \right| \leq \left| z \right| \\
		Rez = \frac{z + \bar{z}}{2} &, \,\, Imz = \frac{z - \bar{z}}{2i}
	\end{align}

\vspace*{2em}
\paragraph{性质}
	下面给出一些命题.
	\begin{enumerate}
		\item 三角不等式.
		\begin{proposition}[$Triangle \,\, Inequality$]
			$Let \,\, z , w \in \C . \,\, Then$
			\begin{align}
				\left| z + w \right| \leq \left| z \right| + \left| w \right|
			\end{align}
			
			\begin{proof}
				$Let \,\, z = a + bi , \,\, w = c + di . \,\, Then$
				\begin{align}
					&\Leftrightarrow \sqrt{(a + c)^2 + (b + d)^2} \leq \sqrt{a^2 + b^2} + \sqrt{c^2 + d^2} \\
					&\Leftrightarrow ac + bd \leq \sqrt{(a^2 + b^2)(c^2 + d^2)} = \sqrt{(ac)^2 + (bd)^2 + a^2 d^2 + b^2 c^2}
				\end{align}
			\end{proof}
		\end{proposition}
	
		\begin{corollary}
			$If \,\, z , w \in \C , \,\, then$
			\begin{align}
				\left| \left| z \right| - \left| w \right| \right| \leq \left| z - w \right|
			\end{align}
		
			\begin{proof}
				\begin{align}
					\left| z \right| &= \left| z - w + w \right| \leq \left| z - w \right| + \left| w \right| \\
					\left| w \right| &= \left| z - w - z \right| \leq \left| z - w \right| + \left| z \right| \\
					\Rightarrow \left| z - w \right| 
					&\geq max\left\{ \left| z \right| - \left| w \right| , \,\, \left| w \right| - \left| z \right| \right\} 
					= \left| \left| z \right| - \left| w \right| \right|
				\end{align}
			\end{proof}
		\end{corollary}
	
		\item $Cauchy - Schwarz$ 不等式.
		\begin{proposition}[$Cauchy - Schwarz$]
			$Let \,\, z_1 , \cdots , z_n , w_1 , \cdots , w_n \in\C . \,\, Then$
			\begin{align}
				\left| \sum_{k = 1}^{n}{z_k w_k} \right|^2 \leq 
				\left( \sum_{k = 1}^{n}{\left| z_k \right|^2} \right)
				\left( \sum_{k = 1}^{n}{\left| w_k \right|^2} \right)
			\end{align}
		
			\begin{proof}
				$\forall \lambda \in \R , \,\, \vartheta \in \R ,$
				\begin{align}
					0 \leq 
					\sum_{k = 1}^{n}{\left| z_k - \lambda e^{i\vartheta} \overline{w_k} \right|^2} 
					&= \sum_{k = 1}^{n}
					{(z_k - \lambda e^{i\vartheta} \overline{w_k}) (\overline{z_k} - \lambda e^{-i\vartheta} w_k)}\\
					&= \sum_{k = 1}^{n}{\left| z_k \right|^2} - 2\left( Re \,\, e^{-i\vartheta} 
					\sum_{k = 1}^{n}{z_k w_k} \right) \lambda + \lambda^2 \sum_{k = 1}^{n}{\left| w_k \right|^2}\\
					&= a \lambda^2 - 2b \lambda + c \\
					&\Rightarrow b^2 \leq ac
				\end{align}
				$Then$
				\begin{align}
					\left( Re \,\, e^{-i\vartheta} \sum_{k = 1}^{n}{z_k w_k} \right)^2 \leq 
					\left( \sum_{k = 1}^{n}{\left| z_k \right|^2} \right)
					\left( \sum_{k = 1}^{n}{\left| w_k \right|^2} \right) 
				\end{align}
				$Suppose \,\, z = \sum_{k = 1}^{n}{z_k w_k} = \left| z \right| e^{i \varphi} \in \C , \,\, let \,\, \vartheta = \varphi . \,\, Then$
				\begin{align}
					Re \,\, e^{-i\vartheta} \sum_{k = 1}^{n}{z_k w_k} &= \left| \sum_{k = 1}^{n}{z_k w_k} \right| \\
					\left| \sum_{k = 1}^{n}{z_k w_k} \right|^2 &\leq 
					\left( \sum_{k = 1}^{n}{\left| z_k \right|^2} \right)
					\left( \sum_{k = 1}^{n}{\left| w_k \right|^2} \right)
				\end{align}
			\end{proof}
		\end{proposition}
	\end{enumerate}

\newpage
\section{课堂例题$2024-02-26$}
	\begin{enumerate}
		\item $Let \,\, z_1 , z_2 \in \C , \,\, \left| z_1 \right| \leq 1 , \,\, \left| z_2 \right| \leq 1 . \,\, If \,\, \left| z_1 - z_2 \right| \geq 1 , \,\, show \,\, that$
		\begin{align}
			\left| z_1 + z_2 \right| \leq \sqrt{3}
		\end{align}
	
	\vspace*{2em}
	\begin{proof}
		(平行四边形对角线的平方和等于四边的平方和.)
		\begin{align}
			\left| z_1 - z_2 \right|^2 
			&= (z_1 - z_2)(\overline{z_1} - \overline{z_2})
			= \left| z_1 \right|^2 + \left| z_2 \right|^2 - z_1 \overline{z_2} - \overline{z_1} z_2\\
			\left| z_1 + z_2 \right|^2 
			&= (z_1 + z_2)(\overline{z_1} + \overline{z_2})
			= \left| z_1 \right|^2 + \left| z_2 \right|^2 + z_1 \overline{z_2} + \overline{z_1} z_2
		\end{align}
		$\Rightarrow$
		\begin{align}
			\left| z_1 - z_2 \right|^2 + \left| z_1 + z_2 \right|^2 
			&= 2(\left| z_1 \right|^2 + \left| z_2 \right|^2) \\
			\left| z_1 + z_2 \right|^2 
			&= 2(\left| z_1 \right|^2 + \left| z_2 \right|^2) - \left| z_1 - z_2 \right|^2 
			\leq 3
		\end{align}
	\end{proof}

	\vspace*{3em}

	\item $Let \,\, z_1 , \cdots , z_n \in \C , \,\, and \,\, let \,\, e_0 , e_1 , \cdots , e_{n + 1} \in \C \,\, be \,\, the \,\, coefficients \,\, of \,\, (z + 1) \overset{n}{\underset{k = 1}{\prod}}{(z + z_k)} , $\\
	$i.e.$
	\begin{align}
		(z + 1) \prod_{k = 1}^{n}{(z + z_k)} = \sum_{k = 0}^{n + 1}{e_k z^{n + 1 - k}}
	\end{align}
	$Show \,\, that \,\, \overset{n + 1}{\underset{k = 0}{\sum}}{(k + 1) e_k z^{n + 1 - k}} = 0 \,\, has \,\, a \,\, root \,\, of \,\, modulus \,\, \geq 1 .$
	
	\vspace*{2em}
	$Specifically , \,\, try \,\, to \,\, show \,\, n = 1 \,\, case . $\\
	$\Leftrightarrow (Let \,\, c \in \C , \,\, show \,\, z^2 + 2(1 + c)z + 3c = 0 \,\, has \,\, a \,\, root \,\, of \,\, modulus \,\, \geq 1 .)$
	
	\vspace*{2em}
	\begin{proof}
		下面对方程$z^2 + 2(1 + c)z + 3c = 0$ 的根的情况进行分类(事实上同时对$c \in \C$ 的取值进行了分类).
		\begin{enumerate}
			\item[(1)]若方程存在实根$z_0 \in \R$,下面可以证明,事实上$(1) \Leftrightarrow c \in \R$.
			\begin{align}
				&z_{0}^2 + 2(1 + c)z_{0} + 3c = 0 \\
				&\Rightarrow (2z_{0} + 3)c = -z_{0}^2 - 2z_0 \\
				&\Rightarrow c = \frac{-z_{0}^2 - 2z_0}{2z_{0} + 3} \in \R \,\, 
				\text{或} \,\, z_0 = \frac{3}{2}(\textbf{此时$-z_{0}^2 - 2z_0 \neq 0$ 矛盾})
			\end{align}
			于是$c \in \R , \,\, z^2 + 2(1 + c)z + 3c = 0$ 为实系数一元二次方程.
			\begin{align}
				\Delta &= 4(1 + c)^2 - 12c = 4(c^2 - c + 1) > 0 , \,\, \forall c \in \R \\
				z &= - 1 - c \pm \sqrt{c^2 - c + 1} \in \R
			\end{align}
			下面再对实数$c \in \R$ 的范围分类讨论.
			\begin{enumerate}
				\item[\rmnum{1}).]$c \geq 0$,则其中一根$z = - 1 - c - \sqrt{c^2 - c + 1} < - 1 , \,\, \left| z \right| > 1$.
				
				\item[\rmnum{2}).]$c < 0$,考虑其中一根
				\begin{align}
					z &= - 1 - c - \sqrt{c^2 - c + 1} \\
					&= - 1 - (\sqrt{c^2 - c + 1} + c)
				\end{align}
				由于$c < 0$,因此$1 - c > 0$.
				\begin{align}
					\sqrt{c^2 - c + 1} = \sqrt{c^2 + (1 - c)} &> \sqrt{c^2} = \left| c \right| \\
					\sqrt{c^2 - c + 1} + c &> 0 \\
					z = - 1 - (\sqrt{c^2 - c + 1} + c) &< - 1 \\
					\left| z \right| &> 1
				\end{align}
			\end{enumerate}
			于是对于$\forall c \in \R$,都有$\left| z \right| > 1$.从而得证.\\
			事实上,根据上述证明过程可知,若$c \in \R$,则原方程必有实根,且两根均为实根,从而
			\begin{align}
				(1) : \text{方程存在实根} \Leftrightarrow c \in \R \Leftrightarrow \text{两根均为实根}
			\end{align}
		
			\vspace*{2em}
			\item[(2)]若方程无实根,即$c \in \C$
		\end{enumerate}
	\end{proof}
	\end{enumerate}

\newpage
\section{复数域$\C$ 上的拓扑概念$\&$ 性质}
	$Let \,\, \alpha \in \C , \,\, open \,\, disc \,\, of \,\, radius \,\, r \,\, centered \,\, at \,\, \alpha$
	\begin{align}
		D_r (\alpha) &\coloneqq \{ z \in \C \mid \left| z - \alpha \right| < r \} \\
		D_{r}^{*} (\alpha) &\coloneqq \{ z \in \C \mid 0 < \left| z - \alpha \right| < r \}
	\end{align}
	$closed \,\, disc \,\, of \,\, radius \,\, r \,\, centered \,\, at \,\, \alpha$
	\begin{align}
		\overline{D}_r (\alpha) &\coloneqq \{ z \in \C \mid \left| z - \alpha \right| \leq r \}
	\end{align}
	$unit \,\, disc:$
	\begin{align}
		\mathbb{D} \coloneqq D_1 (0)
	\end{align}

\vspace*{2em}
	$Let \,\, \Omega \subseteq \C$
	\begin{defn}
		$\alpha \in \Omega \,\, is \,\, an \,\, \underline{interior \,\, point \,\, of \,\, \Omega} \,\, if \,\, \exists r > 0 , \,\, \st D_{r}(\alpha) \subseteq \Omega.$
		
		\begin{rmk}
			$The \,\, set \,\, of \,\, all \,\, interior \,\, points \,\, of \,\, \Omega \,\, is \,\, called \,\, \underline{the \,\, interior \,\, of \,\, \Omega} , \,\, denoted \,\, by \,\, Int(\Omega).$
		\end{rmk}
	\end{defn}

\vspace*{2em}
	
	\begin{defn}
		$\Omega \,\, is \,\, open \,\, if \,\, \Omega = Int(\Omega).$
		\begin{rmk}
			$\C \,\, is \,\, open . \,\, \varnothing \,\, is \,\, open . \,\,(by \,\, convention)$
		\end{rmk}
	\end{defn}

\vspace*{2em}

	\begin{defn}
		$\Omega \,\, is \,\, closed \,\, if \,\, \Omega^c \coloneqq \C \backslash \Omega \,\, is \,\, open.$
	\end{defn}

\vspace*{2em}

	\begin{thm}
		$Every \,\, Cauchy \,\, sequence \,\, in \,\, \C \,\, has \,\, a \,\, limit \,\, in \,\, \C . \,\, That \,\, is , \,\, \C \,\, is \,\, Complete.$
	\end{thm}




\newpage
\section{课堂例题$2024-03-01$}
	\begin{enumerate}
		\item 
		\begin{align}
			\lim_{n \to +\infty}{z_n} = w \Leftrightarrow \lim_{n \to +\infty}{Re z_n} = Re w , \,\, \lim_{n \to +\infty}{Im z_n} = Im w
		\end{align}
		
		\vspace*{2em}
		\begin{proof}
			\begin{enumerate}
				\item[$\Rightarrow$]:$\left| Re z_n - Re w \right| = \left| Re (z_n - w) \right| \leq \left| z_n - w \right|$
				
				\item[$\Leftarrow$]:$\left| z_n - w \right| 
				\leq \left| Re (z_n - w) \right| + \left| Im (z_n - w) \right|
				= \left| Re z_n - Re w \right| + \left| Im z_n - Im w \right|$
			\end{enumerate}
		\end{proof}
	
		\vspace*{2em}
		\item $z \,\, is \,\, a \,\, limit \,\, point \,\, of \,\, \Omega \,\,\Leftrightarrow \,\, z \,\, is \,\, an \,\, accumulation \,\, point \,\, of \,\, \Omega$
		
		\vspace*{2em}
		\begin{proof}
			\begin{enumerate}
				\item[$\Rightarrow$]:$\forall r > 0 , \,\, \exists N_r , \,\, \st n > N , \,\, where \,\, z_n \in \Omega , \,\, z_n \neq z$.\\
				$z_n \in D_{r}^{*}(z) , \,\, z_n \in \Omega , \,\, \forall n > N_r$.\\
				$Hence \,\, z_n \in D_{r}^{*}(z) \cap \Omega \neq \varnothing , \,\, \forall r > 0 , \,\, n > N_r , \,\, i.e. \,\,$\\
				$z \,\, is \,\, an \,\, accumulation \,\, point \,\, of \,\, \Omega$
				
				\item[$\Leftarrow$]:$Take \,\, a \,\, point \,\, z_n \,\, from \,\, D_{\frac{1}{n}}^{*}(z) \cap \Omega \,\, which \,\, is \,\, not \,\, empty.$\\
				$Then \,\, \{ z_n \} \,\, is \,\, a \,\, Cauchy \,\, sequence \,\, which \,\, converges \,\, to \,\, z.$\\
				$Hence \,\, z \,\, is \,\, a \,\, limit \,\, point \,\, of \,\, \Omega.$
			\end{enumerate}
		\end{proof}
		\begin{rmk}
			$A \,\, limit \,\, point \,\, of \,\, \Omega \,\, may \,\, not \,\, belong \,\, to \,\, \Omega.$
		\end{rmk}
	
		\vspace{2em}
		
		\item 课本第一章练习$T3 , T5 , T7$.
	\end{enumerate}

	%  ############################
	\ifx\allfiles\undefined
\end{document}
\fi