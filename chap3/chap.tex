\ifx\allfiles\undefined
\documentclass[12pt, a4paper,oneside, UTF8]{ctexbook}
\usepackage[dvipsnames]{xcolor}
\usepackage{amsmath}   % 数学公式
\usepackage{amsthm}    % 定理环境
\usepackage{amssymb}   % 更多公式符号
\usepackage{graphicx}  % 插图
%\usepackage{mathrsfs}  % 数学字体
%\usepackage{newtxtext,newtxmath}
%\usepackage{arev}
\usepackage{kmath,kerkis}
\usepackage{newtxtext}
\usepackage{bbm}
\usepackage{enumitem}  % 列表
\usepackage{geometry}  % 页面调整
%\usepackage{unicode-math}
\usepackage[colorlinks,linkcolor=black]{hyperref}


\usepackage{ulem}	   % 用于更多的下划线格式,
					   % \uline{}下划线,\uuline{}双下划线,\uwave{}下划波浪线,\sout{}中间删除线,\xout{}斜删除线
					   % \dashuline{}下划虚线,\dotuline{}文字底部加点


\graphicspath{ {flg/},{../flg/}, {config/}, {../config/} }  % 配置图形文件检索目录
\linespread{1.5} % 行高

% 页码设置
\geometry{top=25.4mm,bottom=25.4mm,left=20mm,right=20mm,headheight=2.17cm,headsep=4mm,footskip=12mm}

% 设置列表环境的上下间距
\setenumerate[1]{itemsep=5pt,partopsep=0pt,parsep=\parskip,topsep=5pt}
\setitemize[1]{itemsep=5pt,partopsep=0pt,parsep=\parskip,topsep=5pt}
\setdescription{itemsep=5pt,partopsep=0pt,parsep=\parskip,topsep=5pt}

% 定理环境
% ########## 定理环境 start ####################################
\theoremstyle{definition}
\newtheorem{defn}{\indent 定义}[section]

\newtheorem{lemma}{\indent 引理}[section]    % 引理 定理 推论 准则 共用一个编号计数
\newtheorem{thm}[lemma]{\indent 定理}
\newtheorem{corollary}[lemma]{\indent 推论}
\newtheorem{criterion}[lemma]{\indent 准则}

\newtheorem{proposition}{\indent 命题}[section]
\newtheorem{example}{\indent \color{SeaGreen}{例}}[section] % 绿色文字的 例 ,不需要就去除\color{SeaGreen}{}
\newtheorem*{rmk}{\indent \color{red}{注}}

% 两种方式定义中文的 证明 和 解 的环境:
% 缺点:\qedhere 命令将会失效【技术有限,暂时无法解决】
\renewenvironment{proof}{\par\textbf{证明.}\;}{\qed\par}
\newenvironment{solution}{\par{\textbf{解.}}\;}{\qed\par}

% 缺点:\bf 是过时命令,可以用 textb f等替代,但编译会有关于字体的警告,不过不影响使用【技术有限,暂时无法解决】
%\renewcommand{\proofname}{\indent\bf 证明}
%\newenvironment{solution}{\begin{proof}[\indent\bf 解]}{\end{proof}}
% ######### 定理环境 end  #####################################

% ↓↓↓↓↓↓↓↓↓↓↓↓↓↓↓↓↓ 以下是自定义的命令  ↓↓↓↓↓↓↓↓↓↓↓↓↓↓↓↓

% 用于调整表格的高度  使用 \hline\xrowht{25pt}
\newcommand{\xrowht}[2][0]{\addstackgap[.5\dimexpr#2\relax]{\vphantom{#1}}}

% 表格环境内长内容换行
\newcommand{\tabincell}[2]{\begin{tabular}{@{}#1@{}}#2\end{tabular}}

% 使用\linespread{1.5} 之后 cases 环境的行高也会改变,重新定义一个 ca 环境可以自动控制 cases 环境行高
\newenvironment{ca}[1][1]{\linespread{#1} \selectfont \begin{cases}}{\end{cases}}
% 和上面一样
\newenvironment{vx}[1][1]{\linespread{#1} \selectfont \begin{vmatrix}}{\end{vmatrix}}

\def\d{\textup{d}} % 直立体 d 用于微分符号 dx
\def\R{\mathbb{R}} % 实数域
\def\N{\mathbb{N}} % 自然数域
\def\C{\mathbb{C}} % 复数域
\def\Z{\mathbb{Z}} % 整数环
\def\Q{\mathbb{Q}} % 有理数域
\newcommand{\bs}[1]{\boldsymbol{#1}}    % 加粗,常用于向量
\newcommand{\ora}[1]{\overrightarrow{#1}} % 向量

% 数学 平行 符号
\newcommand{\pll}{\kern 0.56em/\kern -0.8em /\kern 0.56em}

% 用于空行\myspace{1} 表示空一行 填 2 表示空两行  
\newcommand{\myspace}[1]{\par\vspace{#1\baselineskip}}

%s.t. 用\st就能打出s.t.
\DeclareMathOperator{\st}{s.t.}

%罗马数字 \rmnum{}是小写罗马数字, \Rmnum{}是大写罗马数字
\makeatletter
\newcommand{\rmnum}[1]{\romannumeral #1}
\newcommand{\Rmnum}[1]{\expandafter@slowromancap\romannumeral #1@}
\makeatother
\begin{document}
	% \title{{\Huge{\textbf{$Complex \,\, Analysis$\footnote{课堂教材:\textbf{《$Complex \,\, Analysis$》---  $Elias \,\, M. \,\, Stein$}}}}}}
\author{$-TW-$}
\date{\today}
\maketitle                   % 在单独的标题页上生成一个标题

\thispagestyle{empty}        % 前言页面不使用页码
\begin{center}
	\Huge\textbf{序}
\end{center}


\vspace*{3em}
\begin{center}
	\large{\textbf{天道几何,万品流形先自守;}}\\
	
	\large{\textbf{变分无限,孤心测度有同伦。}}
\end{center}

\vspace*{3em}
\begin{flushright}
	\begin{tabular}{c}
		\today \\ \small{\textbf{长夜伴浪破晓梦,梦晓破浪伴夜长}}
	\end{tabular}
\end{flushright}


\newpage                      % 新的一页
\pagestyle{plain}             % 设置页眉和页脚的排版方式(plain:页眉是空的,页脚只包含一个居中的页码)
\setcounter{page}{1}          % 重新定义页码从第一页开始
\pagenumbering{Roman}         % 使用大写的罗马数字作为页码
\tableofcontents              % 生成目录

\newpage                      % 以下是正文
\pagestyle{plain}
\setcounter{page}{1}          % 使用阿拉伯数字作为页码
\pagenumbering{arabic}
\setcounter{chapter}{-1}    % 设置 -1 可作为第零章绪论从第零章开始 
	\else
	\fi
	%  ############################ 正文部分

\chapter{$Week \,\, 3$}
\section{幂级数,解析函数,复对数}
与数学分析中的概念一致,下面相当于来复习一下有关\textbf{幂级数}的概念.
\begin{itemize}
	\item 幂级数$\overset{\infty}{\underset{n = 0}{\sum}}{z_n}$ converges $\Leftrightarrow$ 部分和$\{ S_N = \overset{N}{\underset{n = 0}{\sum}}{z_n} \}$ converges.
	
	\item $\overset{\infty}{\underset{n = 0}{\sum}}{\left| z_n \right|}$ converges $\Rightarrow$ The series converges absolutely(绝对收敛).
	
	\item \textbf{Absolutely convergent} $\Rightarrow$ \textbf{convergent}
	
	\item If $\overset{\infty}{\underset{n = 0}{\sum}}{z_n}$ converges, then $\underset{n \to \infty}{\lim}{z_n} = 0$.
\end{itemize}

\vspace{2em}
A power series (with center 0) is an expansion of the form $\overset{\infty}{\underset{n = 0}{\sum}}{a_n z^n}$, where $a_n \in \C$ are fixed and $z$ varies in $\C$.(下面通常讨论形式为$\overset{\infty}{\underset{n = 0}{\sum}}{a_n z^n}$ 的幂级数)

\vspace{2em}
下面给出复幂级数的\textbf{收敛半径}的定义及\textbf{收敛圆盘}.
\begin{thm}\label{thm 3.1.1}
	Given a power series $\overset{\infty}{\underset{n = 0}{\sum}}{a_n z^n}$, define
	\begin{align}
		R = \varliminf_{n \to \infty}{\left| a_n \right|^{-\frac{1}{n}}} = \frac{1}{\underset{n \to \infty}{\varlimsup}{\left| a_n \right|^{\frac{1}{n}}}} 
		\,\,\,\, (\textbf{Hardamard's Formula})
	\end{align}
	(Here we use the convertion $\frac{1}{\infty} = 0$, $\frac{1}{0} = \infty$.) Then
	\begin{enumerate}
		\item[(1)]If $\left| z \right| < R$, the series converges absolutely.
		
		\item[(2)]If $\left| z \right| > R$, the series diverges.
	\end{enumerate}
	
	\begin{rmk}
		The number $R$ is called the \underline{\textcolor{blue}{\textbf{radius of convergence}}} of the power series, \\
		and the region $\left| z \right| < R$ is called the \underline{\textcolor{blue}{\textbf{disc of convergence}}}.
	\end{rmk}
\end{thm}

\newpage
\begin{example}\label{ex 3.1.1}
	下面给出一些用\uwave{幂级数定义}的常见函数的例子.
	\begin{itemize}
		\item Exponential function
		\begin{align}
			e^z \coloneqq \sum_{n = 0}^{\infty}{\frac{z^n}{n!}} , \,\, z \in \C
		\end{align}
		
		\item Trigonometric function
		\begin{align}
			\cos{z} \coloneqq \sum_{n = 0}^{\infty}{(-1)^{n} \frac{z^{2n}}{(2n)!}} \,\, , \,\, \sin{z} \coloneqq \sum_{n = 0}^{\infty}{(-1)^{n} \frac{z^{2n + 1}}{(2n + 1)!}}
		\end{align}
		
		\item 双曲余弦、正弦
		\begin{align}
			\cosh{z} \coloneqq \sum_{n = 0}^{\infty}{\frac{z^{2n}}{(2n)!}} \,\, , \,\, \sinh{z} \coloneqq \sum_{n = 0}^{\infty}{\frac{z^{2n + 1}}{(2n + 1)!}}
		\end{align}
	\end{itemize}
	
	\begin{rmk}
		由定义容易得到,$e^{iz} = \cos{z} + i\sin{z}$ $\Rightarrow$ 将$z$ 限制到$\R$ 上则有:$e^{i\vartheta} = \cos{\vartheta} + i\sin{\vartheta}$.
		\begin{align}
			\cos{z} = \frac{e^{iz} + e^{-iz}}{2} , \,\, \sin{z} = \frac{e^{iz} - e^{-iz}}{2}
		\end{align}
	\end{rmk}
\end{example}

\vspace{2em}
下面这个定理说明了幂级数在收敛圆盘内解析. 并给出了幂级数的导数.
\begin{thm}\label{thm 3.1.2}
	The power series $f(z) = \overset{\infty}{\underset{n = 0}{\sum}}{a_n z^n}$ defines a holomorphic function in its disc of convergence. Moreover, $f^{'}(z) = \overset{\infty}{\underset{n = 0}{\sum}}{n a_n z^{n - 1}}$, which has the same radius of convergence.
	
	\vspace{2em}
	\begin{proof}
		\textbf{Hadamard's formula} tells $\overset{\infty}{\underset{n = 0}{\sum}}{a_n z^n}$ and $\overset{\infty}{\underset{n = 0}{\sum}}{n a_n z^{n - 1}}$ have the same $R$.\\
		Let $g(z) = \overset{\infty}{\underset{n = 0}{\sum}}{n a_n z^{n - 1}}$, $\forall z$ with $\left| z \right| < R$, we can find $r$, $\st \left| z \right| < r < R$.\\
		For $\forall h \in \C$ $\st \left| h \right| < r - \left| z \right|$, we estimate
		\begin{align}
			\left| f(z + h) - f(z) - hg(z) \right| 
			&= \left| \sum_{n = 0}^{\infty}{a_n \left( (z + h)^n - z^n - nhz^{n - 1} \right)} \right| \\
			&= \left| \sum_{n = 2}^{\infty}{\left( a_n \sum_{k = 2}^{n}{\tbinom{n}{k} h^k z^{n - k}} \right)} \right| \\
			&\leq \left| h \right|^2 \sum_{n = 2}^{\infty}{\left| a_n \right|} \sum_{k = 0}^{n - 2}{\tbinom{n}{k + 2} \left| h^k z^{n - 2 - k} \right|}
		\end{align}
		\newpage
		Since $\tbinom{n}{k + 2} \leq n(n - 1)\tbinom{n - 2}{k}$, then
		\begin{align}
			\left| f(z + h) - f(z) - hg(z) \right| 
			&\leq \left| h \right|^2 \sum_{n = 2}^{\infty}{\left| a_n \right| n(n - 1)} \sum_{k = 0}^{n - 2}{\tbinom{n - 2}{k} \left| h \right|^k \left| z \right|^{n - 2 - k}} \\
			&= \left| h \right|^2 \sum_{n = 2}^{\infty}{\left| a_n \right| n(n - 1) \left( \left| z \right| + \left| h \right| \right)^{n - 2}} \\
			&< \left| h \right|^2 \sum_{n = 2}^{\infty}{\left| a_n \right| n(n - 1) r^{n - 2}} = \left| h \right|^2 \cdot c 
		\end{align}
		Thus
		\begin{align}
			\left| \frac{f(z + h) - f(z)}{h} - g(z) \right| < \left| h \right| \cdot c
		\end{align}
		Therefore, the result follows.
	\end{proof}
\end{thm}

\begin{corollary}\label{cor 3.1.3}
	A power series is infinitely differentiable in its disc of convergence.
	
	\begin{rmk}
		Thm \ref{thm 3.1.2} 即说明了幂级数在收敛圆盘内解析.
	\end{rmk}
\end{corollary}

\vspace{2em}
下面给出推广到更一般的幂级数的导数,即中心不一定在原点的情形.\\
A power series centered at $z_0 \in \C$ is an expression of the form
\begin{align}
	f(z) = \sum_{n = 0}^{\infty}{(z - z_0)^n}
\end{align}
Let $g(z) = \overset{\infty}{\underset{n = 0}{\sum}}{a_n z^n}$, then $f(z) = g(w)$, where $w = z - z_0$.\\
According to the \textbf{Chain Rule}(链式法则), $f^{'}(z) = \overset{\infty}{\underset{n = 0}{\sum}}{n a_n (z - z_0)^{n - 1}}$

\vspace{2em}
下面严格地给出\textbf{解析}的定义.
\begin{defn}\label{def 3.1.1}
	A function $f$ defined on an open set $\Omega$ is said to be \underline{\textcolor{blue}{\textbf{analytic}}} at $z_0 \in \C$ if there is a power series $\overset{\infty}{\underset{n = 0}{\sum}}{a_n (z - z_0)^n}$ with positive radius of convergence, $\st$
	\begin{align}
		f(z) = \sum_{n = 0}^{\infty}{a_n (z - z_0)^n} \,\, &for \,\, all \,\, z \,\, in \,\, a \,\, neighbourhood \,\, of \,\, z_0 \\
		&(i.e. \,\, \forall z \in D_{r}(z_0) , \,\, for \,\, some \,\, r > 0)
	\end{align}
	If $f$ is analytic at every point of $\Omega$, then we say f is \underline{\textcolor{blue}{\textbf{analytic on $\Omega$}}}.
\end{defn}

\newpage
下面给出有关指数函数$e^z$ 的一些等式(命题).\\
在此之前,先给出\textbf{Cauchy Multiplication Theorem}.
\begin{lemma}\label{lemma 3.1.4}
	If $\sum{a_n}$, $\sum{b_n}$ are absolutely convergent, then
	\begin{align}
		\sum_{n = 0}^{\infty}{\left( \sum_{k = 0}^{n}{a_k b_{n - k}} \right)} = \left( \sum{a_n} \right) \left( \sum{b_n} \right)
	\end{align}
\end{lemma}

\vspace{2em}
\begin{proposition}\label{prop 3.1.1}
	For $z_1 , z_2 \in \C$, $e^{(z_1 + z_2)} = e^{z_1} \cdot e^{z_2}$.
\end{proposition}

\vspace{2em}
\begin{corollary}\label{cor 3.1.5}
	If $z = x + iy$, $x , y \in \R$, then
	\begin{align}
		e^z = e^x (\cos{y} + i\sin{y})
	\end{align}
\end{corollary}

\vspace{2em}
\begin{corollary}\label{cor 3.1.6}
	\textbf{De Moire's Formula}.\\
	For $\vartheta \in \R$,
	\begin{align}
		(\cos{\vartheta} + i\sin{\vartheta})^n = \cos{n\vartheta} + i\sin{n\vartheta}
	\end{align}
\end{corollary}

\vspace{2em}
下面来引入复数域上的\textbf{对数函数(Complex Logarithm)}.\\
$\forall z \in \C \backslash \{ 0 \}$, write $z = r e^{i\vartheta}$. Then $e^w = z$ can be solved.\\
If $w = u + iv$, $u , v \in \R$, then
\begin{align}
	e^u \cdot e^{iv} = r e^{i\vartheta} \,\, \Rightarrow \,\, u = \log{r} , \,\, v = \vartheta + 2k\pi , k \in \Z
\end{align}

Let $Log(z)$ be the set of above, then we get Complex Logarithm.

\begin{defn}\label{def 3.1.2}
	$\forall z \in \C \backslash \{ 0 \}$. Define
	\begin{align}
		Log(z) \coloneqq \log{\left| z \right|} + i(\arg{z} + 2k\pi) , k \in \Z
	\end{align}
	Here $\arg{z}$ is an argument of $z$ satisfying $-\pi < \arg{z} \leq \pi$.
	\begin{center}
		(We call $\arg{z}$ the \textcolor{blue}{\textbf{principal argument}} of $z$.) 
	\end{center}
\end{defn}

\newpage
下面介绍复对数的\textbf{主值支}的概念.
\begin{defn}\label{def 3.1.3}
	Define the \underline{\textcolor{blue}{\textbf{principal branch}}} of the logarithm on a "cut plane"
	\begin{align}
		\log : \C \backslash \R_{\leq 0} &\longrightarrow \C \\
		z &\longmapsto \log{\left| z \right|} + i\arg{z} , \,\, -\pi < \arg{z} < \pi
	\end{align}
\end{defn}

\vspace{2em}
\begin{example}\label{ex 3.1.2}
	\begin{align}
		Log(-1) &= (2k + 1)\pi i \\
		Log(i) &= (2k + \frac{1}{2})\pi i \\
		\log{i} &= \frac{\pi}{2}i \\
		\log{(1 + i)} &= \frac{1}{2}\log{2} + \frac{\pi}{4}i
	\end{align}
\end{example}

\vspace{2em}
\begin{proposition}
	\begin{align}
		e^{Log(z)} &= z , \,\, z \neq 0 \\
		Log(z_1 z_2) &= Log(z_1) + Log(z_2) \\
		\log{z_1 z_2} &\neq \log{z_1} + \log{z_2} \,\, in \,\, general
	\end{align}
\end{proposition}

\newpage
\section{课堂例题$2024-03-11$}
\begin{enumerate}
	\item Let $z \neq 0$. Then $\exists n$ different $z_0 , \cdots , z_{n - 1}$, $\st$
	\begin{align}
		z_{k}^{n} = z , \,\, k = 0 , \cdots , n - 1
	\end{align}
	
	\vspace{2em}
	\begin{solution}
		Let $z = \left| z \right| e^{i\vartheta}$, $w = r e^{it}$, $r > 0$, $t \in \R$. Then
		\begin{align}
			w^n = z \,\, \Rightarrow \,\, r^n e^{int} = \left| z \right| e^{i\vartheta} \,\, \Rightarrow \,\, 
			\begin{cases}
				r = \left| z \right|^{\frac{1}{n}} \\
				nt = \vartheta + 2k\pi , k \in \Z
			\end{cases}
		\end{align}
	\end{solution}
	
	\vspace{2em}
	
	\item Proof
	\begin{align}
		\left| \sum_{k = 0}^{n}{e^{ikx}}  \right| \leq \left| \frac{1}{\sin{\frac{x}{2}}} \right| , \,\, \forall x \in \R \backslash \{ 2k\pi \mid k \in \Z \}
	\end{align}
	
	\vspace{2em}
	
	\item 课本第一章练习$T16 , T19$
\end{enumerate}

\newpage
\section{复对数的性质}
Let $\alpha \in \C$. We may define
\begin{align}
	z^\alpha = e^{\alpha \log{z}} , \,\, z \neq 0
\end{align}

\vspace{2em}
\begin{proposition}\label{prop 3.3.1}
	The function $f(z) = \log{z}$, $z \in \C \backslash \R_{\leq 0}$ is holomorphic.
	
	\vspace{2em}
	\begin{proof}
		$\forall z_0 \in \C \backslash \R_{\leq 0}$, let $w = \log{z}$, $w_0 = \log{z_0}$. Then
		\begin{align}
			\lim_{z \to z_0}{\frac{\log{z} - \log{z_0}}{z - z_0}} = \lim_{w \to w_0}{\frac{w - w_0}{e^w - e^{w_0}}} = \frac{1}{e^{w_0}} = \frac{1}{z_0}
		\end{align}
		Therefore, $(\log{z})^{'} = \frac{1}{z}$.
	\end{proof}
\end{proposition}

\vspace{2em}
\begin{proposition}\label{prop 3.3.2}
	Show
	\begin{align}
		\log{(1 + z)} = \sum_{n = 1}^{\infty}{(-1)^{n - 1}\frac{z^n}{n}} \,\, on \,\, \mathcal{D}
	\end{align}
	
	\vspace{2em}
	\begin{proof}
		Let $f(z) = \log{(1 + z)}$, $g(z) = \overset{\infty}{\underset{n = 1}{\sum}}{(-1)^n \frac{z^{n - 1}}{n}}$. Both are holomorphic on $\mathcal{D}$ and
		\begin{align}
			f^{'}(z) = \frac{1}{1 + z} \,\, , \,\, g^{'}(z) = \sum_{n = 1}^{\infty}{(-1)^n z^{n - 1}} = \frac{1}{1 + z}
		\end{align}
		And so $(f - g)^{'} = 0$ on $\mathcal{D}$. Therefore, $f - g = c$. Taking $z = 0$, $f(0) = g(0)$ $\Rightarrow$ $c = 0$.
	\end{proof}
\end{proposition}

\newpage
\section{道路}
先给出\textbf{道路(path)} 的定义.
\begin{defn}\label{def 3.4.1}
	A continuous function $z(t) = x(t) + i y(t)$ from $[a , b] \subset \R$ to $\C$ is called a \underline{\textcolor{blue}{\textbf{path}}} (or a parametric curve) connecting $z(a)$ and $z(b)$.($z(a)$ is called the starting point, $z(b)$ the end point)\\
	The path is \underline{\textcolor{blue}{\textbf{closed}}} if $z(a) = z(b)$.\\
	The path is \underline{\textcolor{blue}{\textbf{simple}}} if $z(t) \neq z(s)$ unless 
	$\begin{cases}
		(1) t = s\\
		(2) t = a , s = b
	\end{cases}$
\end{defn}

\vspace{2em}
下面给出道路\textbf{光滑性}的描述.
\begin{defn}\label{def 3.4.2}
	We say that a path $z(t) = x(t) + i y(t)$, $t \in [a , b]$ is \underline{\textcolor{blue}{\textbf{smooth}}} if $x(t) , y(t)$ are continuously differentiable and $z^{'}(t) = x^{'}(t) + i y^{'}(t) \neq 0$, $t \in [a , b]$. Here $z^{'}(a) , z^{'}(b)$ are understood as one-sided derivative.
\end{defn}

\vspace{2em}
下面给出两条道路\textbf{等价}的定义.
\begin{defn}
	Two paths $z : [a , b] \longrightarrow \C$, $\widetilde{z} : [c, d] \longrightarrow \C$ are \underline{\textcolor{blue}{\textbf{equivalent}}} if $\exists$ bijection and differential
	\begin{align}
		t : [c , d] &\longrightarrow [a , b] \\
		s &\longmapsto t(s)
	\end{align}
	$\st \widetilde{z}(s) = z(t(s))$ and $t^{'}(s) > 0$.
\end{defn}

\vspace{2em}
下面给出\textbf{道路反向}的定义.
\begin{defn}\label{def 3.4.4}
	Given a path $z$, we can define a path $\widetilde{z}$ obtained from $z$ by reversing the orietation
	\begin{align}
		z(t) &: [a , b] \longrightarrow \C \\
		\widetilde{z}(t) = z(a + b - t) &: [a , b] \longrightarrow \C
	\end{align}
\end{defn}

\newpage
这里我们规定一下道路的\textbf{正向 / 逆向}(逆时针为正向).
\begin{defn}\label{def 3.4.5}
	A path has \underline{\textcolor{blue}{\textbf{positive orientation}}} if it travels counterclockwisely.\\
	($\cdots$ \underline{\textcolor{blue}{\textbf{negative orientation}}} $\cdots$ clockwisely.)
\end{defn}

\vspace{2em}
下面我们给出\textbf{分段光滑}的定义.
\begin{defn}\label{def 3.4.6}
	A path $z(t) : [a , b] \longrightarrow \C$ is \underline{\textcolor{blue}{\textbf{piecewise smooth}}} if $\exists$ a partion $a = a_0 < a_1 < \cdots < a_n = b$, $\st$ $z(t)$ is smooth in each $[a_k , a_{k + 1}]$, $k = 0 , \cdots , n - 1$.
\end{defn}

\vspace{2em}
下面说明两条道路的\textbf{连接}.\\
Paths can be concatenated. If $z : [a , b] \longrightarrow \C$, $\widetilde{z} : [b , c] \longrightarrow \C$ and $z(b) = \widetilde{z}(b)$, we can define $w : [a , c] \longrightarrow \C$ as $w(t) = 
\begin{cases}
	z(t) , a \leq t \leq b\\
	\widetilde{z}(t) , b \leq t \leq c
\end{cases}$.
\underline{\textcolor{blue}{\textbf{Concatenation}}} of $z , \widetilde{z}$ is denoted as \textcolor{blue}{$z \circ \widetilde{z}$}.\\

\vspace{2em}
下面给出\textbf{zig-zag道路}的定义.
\begin{defn}\label{def 3.4.7}
	A path is \underline{\textcolor{blue}{\textbf{zig-zag}}} if it consists of finitely many horizontal or vertical line seqments.
\end{defn}

\vspace{2em}
下面的命题说明区域内的任两点可由一条zig-zag道路连接.
\begin{proposition}\label{prop 3.4.1}
	Let $\Omega \subset \C$ be a region. Then any two points in $\Omega$ can be joined by a zig-zag path.
	
	\vspace{2em}
	\begin{proof}
		\begin{itemize}
			\item Case when $\Omega = D_{R}(z_0)$, where $z_0 \in \C$, $R > 0$.\\
			$\forall \alpha , \beta \in \Omega$, we can join them to the horiziontal diameter via a vertical line seqment.
			
			\item Now let $\Omega$ be an arbitrary region. $\forall \alpha \in \Omega$. Let
			\begin{align}
				A \coloneqq \{ \beta \in \Omega \mid \exists \,\, zig-zag \,\, path \,\, \gamma \,\, connecting \,\, \beta \,\, and \,\, \alpha \}
			\end{align}
			Then 容易证$\alpha \in A \neq \varnothing$ 既开又闭,从而$A = \Omega$.
		\end{itemize}
	\end{proof}
\end{proposition}

\newpage
\section{课堂例题$2024-03-15$}
\begin{enumerate}
	\item Calculate $2^i$, $i^i$.
	
	\vspace{2em}
	
	\item Find all possible values of $(1 + \sqrt{3}i)^{\frac{1}{8}}$.
	
	\vspace{2em}
	
	\item Let $z_n \in \C$, $Rez_n \geq 0$, $n = 1 , 2 , \cdots$. If $\overset{\infty}{\underset{n = 1}{\sum}}{z_n}$ and $\overset{\infty}{\underset{n = 1}{\sum}}{z_{n}^2}$ both converge, show that $\overset{\infty}{\underset{n = 1}{\sum}}{\left| z_n \right|^2}$ converges.
	
	\vspace{2em}
	
	\item Let $f(z) = \overset{\infty}{\underset{n = 1}{\sum}}{a_n z^n}$ be holomorphic on $\mathcal{D}$. Assume $\left| f(z) \right| \leq 1$, $\forall z \in \mathcal{D}$. Show $\left| a_n \right| \leq 1$, $n = 1 , 2 , \cdots$.
	
	\vspace{2em}
\end{enumerate}






	%  ############################
	\ifx\allfiles\undefined
\end{document}
\fi