\ifx\allfiles\undefined
\documentclass[12pt, a4paper,oneside, UTF8]{ctexbook}
\usepackage[dvipsnames]{xcolor}
\usepackage{amsmath}   % 数学公式
\usepackage{amsthm}    % 定理环境
\usepackage{amssymb}   % 更多公式符号
\usepackage{graphicx}  % 插图
%\usepackage{mathrsfs}  % 数学字体
%\usepackage{newtxtext,newtxmath}
%\usepackage{arev}
\usepackage{kmath,kerkis}
\usepackage{newtxtext}
\usepackage{bbm}
\usepackage{enumitem}  % 列表
\usepackage{geometry}  % 页面调整
%\usepackage{unicode-math}
\usepackage[colorlinks,linkcolor=black]{hyperref}


\usepackage{ulem}	   % 用于更多的下划线格式,
					   % \uline{}下划线,\uuline{}双下划线,\uwave{}下划波浪线,\sout{}中间删除线,\xout{}斜删除线
					   % \dashuline{}下划虚线,\dotuline{}文字底部加点


\graphicspath{ {flg/},{../flg/}, {config/}, {../config/} }  % 配置图形文件检索目录
\linespread{1.5} % 行高

% 页码设置
\geometry{top=25.4mm,bottom=25.4mm,left=20mm,right=20mm,headheight=2.17cm,headsep=4mm,footskip=12mm}

% 设置列表环境的上下间距
\setenumerate[1]{itemsep=5pt,partopsep=0pt,parsep=\parskip,topsep=5pt}
\setitemize[1]{itemsep=5pt,partopsep=0pt,parsep=\parskip,topsep=5pt}
\setdescription{itemsep=5pt,partopsep=0pt,parsep=\parskip,topsep=5pt}

% 定理环境
% ########## 定理环境 start ####################################
\theoremstyle{definition}
\newtheorem{defn}{\indent 定义}[section]

\newtheorem{lemma}{\indent 引理}[section]    % 引理 定理 推论 准则 共用一个编号计数
\newtheorem{thm}[lemma]{\indent 定理}
\newtheorem{corollary}[lemma]{\indent 推论}
\newtheorem{criterion}[lemma]{\indent 准则}

\newtheorem{proposition}{\indent 命题}[section]
\newtheorem{example}{\indent \color{SeaGreen}{例}}[section] % 绿色文字的 例 ,不需要就去除\color{SeaGreen}{}
\newtheorem*{rmk}{\indent \color{red}{注}}

% 两种方式定义中文的 证明 和 解 的环境:
% 缺点:\qedhere 命令将会失效【技术有限,暂时无法解决】
\renewenvironment{proof}{\par\textbf{证明.}\;}{\qed\par}
\newenvironment{solution}{\par{\textbf{解.}}\;}{\qed\par}

% 缺点:\bf 是过时命令,可以用 textb f等替代,但编译会有关于字体的警告,不过不影响使用【技术有限,暂时无法解决】
%\renewcommand{\proofname}{\indent\bf 证明}
%\newenvironment{solution}{\begin{proof}[\indent\bf 解]}{\end{proof}}
% ######### 定理环境 end  #####################################

% ↓↓↓↓↓↓↓↓↓↓↓↓↓↓↓↓↓ 以下是自定义的命令  ↓↓↓↓↓↓↓↓↓↓↓↓↓↓↓↓

% 用于调整表格的高度  使用 \hline\xrowht{25pt}
\newcommand{\xrowht}[2][0]{\addstackgap[.5\dimexpr#2\relax]{\vphantom{#1}}}

% 表格环境内长内容换行
\newcommand{\tabincell}[2]{\begin{tabular}{@{}#1@{}}#2\end{tabular}}

% 使用\linespread{1.5} 之后 cases 环境的行高也会改变,重新定义一个 ca 环境可以自动控制 cases 环境行高
\newenvironment{ca}[1][1]{\linespread{#1} \selectfont \begin{cases}}{\end{cases}}
% 和上面一样
\newenvironment{vx}[1][1]{\linespread{#1} \selectfont \begin{vmatrix}}{\end{vmatrix}}

\def\d{\textup{d}} % 直立体 d 用于微分符号 dx
\def\R{\mathbb{R}} % 实数域
\def\N{\mathbb{N}} % 自然数域
\def\C{\mathbb{C}} % 复数域
\def\Z{\mathbb{Z}} % 整数环
\def\Q{\mathbb{Q}} % 有理数域
\newcommand{\bs}[1]{\boldsymbol{#1}}    % 加粗,常用于向量
\newcommand{\ora}[1]{\overrightarrow{#1}} % 向量

% 数学 平行 符号
\newcommand{\pll}{\kern 0.56em/\kern -0.8em /\kern 0.56em}

% 用于空行\myspace{1} 表示空一行 填 2 表示空两行  
\newcommand{\myspace}[1]{\par\vspace{#1\baselineskip}}

%s.t. 用\st就能打出s.t.
\DeclareMathOperator{\st}{s.t.}

%罗马数字 \rmnum{}是小写罗马数字, \Rmnum{}是大写罗马数字
\makeatletter
\newcommand{\rmnum}[1]{\romannumeral #1}
\newcommand{\Rmnum}[1]{\expandafter@slowromancap\romannumeral #1@}
\makeatother
\begin{document}
	% \title{{\Huge{\textbf{$Complex \,\, Analysis$\footnote{课堂教材:\textbf{《$Complex \,\, Analysis$》---  $Elias \,\, M. \,\, Stein$}}}}}}
\author{$-TW-$}
\date{\today}
\maketitle                   % 在单独的标题页上生成一个标题

\thispagestyle{empty}        % 前言页面不使用页码
\begin{center}
	\Huge\textbf{序}
\end{center}


\vspace*{3em}
\begin{center}
	\large{\textbf{天道几何,万品流形先自守;}}\\
	
	\large{\textbf{变分无限,孤心测度有同伦。}}
\end{center}

\vspace*{3em}
\begin{flushright}
	\begin{tabular}{c}
		\today \\ \small{\textbf{长夜伴浪破晓梦,梦晓破浪伴夜长}}
	\end{tabular}
\end{flushright}


\newpage                      % 新的一页
\pagestyle{plain}             % 设置页眉和页脚的排版方式(plain:页眉是空的,页脚只包含一个居中的页码)
\setcounter{page}{1}          % 重新定义页码从第一页开始
\pagenumbering{Roman}         % 使用大写的罗马数字作为页码
\tableofcontents              % 生成目录

\newpage                      % 以下是正文
\pagestyle{plain}
\setcounter{page}{1}          % 使用阿拉伯数字作为页码
\pagenumbering{arabic}
\setcounter{chapter}{-1}    % 设置 -1 可作为第零章绪论从第零章开始 
	\else
	\fi
	%  ############################ 正文部分

\chapter{$Week \,\, 4$}
\section{曲线积分}
\paragraph{积分}
下面先给出复数域上积分的定义.
\begin{defn}\label{def 4.1.1}
	Let $z(t) = x(t) + i y(t)$, $t \in [a , b] \subset \R$. If $x(t) , y(t)$ are differentiable, we define $z^{'}(t) = x^{'}(t) + i y^{'}(t)$.\\
	Similarly, if $x(t) , y(t)$ are continuous, we define
	\begin{align}
		\int_{a}^{b}{z(t) dt} = \int_{a}^{b}{x(t) dt} + i \int_{a}^{b}{y(t) dt}
	\end{align}
\end{defn}

\vspace{2em}
容易证明,复数域上的积分同样具有三角不等式.
\begin{proposition}\label{prop 4.1.1}
	Let $f : [a , b] \longrightarrow \C$ be continuous. Then 
	\begin{align}
		\left| \int_{a}^{b}{f(t) dt} \right| \leq \int_{a}^{b}{\left| f(t) \right| dt}
	\end{align}
	
	\vspace{2em}
	\begin{proof}
		Write $\int_{a}^{b}{f(t) dt} = r e^{i\vartheta}$, $r \geq 0$. Then
		\begin{align}
			r = e^{-i\vartheta} \int_{a}^{b}{f(t) dt} = \int_{a}^{b}{e^{-i\vartheta} f(t) dt} 
			&= \left| \int_{a}^{b}{Re \, e^{-i\vartheta} f(t) dt} \right| \\
			&\leq \int_{a}^{b}{\left| Re \, e^{-i\vartheta} f(t) \right| dt} \\
			&\leq \int_{a}^{b}{\left| e^{-i\vartheta} f(t) \right| dt} = \int_{a}^{b}{\left| f(t) \right| dt}
		\end{align}
	\end{proof}
\end{proposition}

\newpage
\paragraph{曲线积分}
下面给出复数域上\textbf{连续}道路的\textbf{曲线积分}的定义.
\begin{defn}\label{def 4.1.2}
	Let $\Omega \subset \C$ be open. Given a smooth path $\gamma$ in $\Omega$ parametrized by $z : [a , b] \longrightarrow \Omega$ and a continuous funciton $f : \Omega \longrightarrow \C$. We define the \underline{\textcolor{blue}{\textbf{integral of $f$ along $\gamma$}}} by
	\begin{align}
		\int_{\gamma}{f(z) dz} \coloneqq \int_{a}^{b}{f(z(t)) z^{'}(t) dt}
	\end{align}
	Let $\widetilde{z}(t) : [c , d] \longrightarrow \Omega$ be equivalent to $z(t)$. Then
	\begin{align}
		\int_{a}^{b}{f(z(t)) z^{'}(t) dt} = \int_{c}^{d}{f(\widetilde{z}(t)) \widetilde{z}^{'}(t) dt}
	\end{align}
\end{defn}

\vspace{2em}
下面给出\textbf{分段连续}道路的曲线积分及\textbf{曲线长度}的定义.
\begin{defn}\label{def 4.1.3}
	If $\gamma$ is piecewise smooth and $z(t)$ is a piecewise smooth parametrization as before, we define
	\begin{align}
		\int_{\gamma}{f(z) dz} = \sum_{k = 0}^{n - 1}{\int_{a_{k}}^{a_{k + 1}}{f(z(t)) z^{'}(t) dt}}
	\end{align}
	The \underline{\textcolor{blue}{\textbf{length}}} of the smooth curve $\gamma$ is 
	\begin{align}
		\textcolor{blue}{length(\gamma)} = \int_{a}^{b}{\left| z^{'}(t) \right| dt}
	\end{align}
\end{defn}

\vspace{2em}
If $f = u + i v$, $z(t) = x(t) + i y(t)$, then
\begin{align}
	\int_{\gamma}{f(z) dz} 
	= \int_{a}^{b}{f(z(t)) z^{'}(t) dt} 
	&= \int_{a}^{b}{(u + iv)(x^{'}(t) + i y^{'}(t)) dt} \\
	&= \int_{a}^{b}{(ux^{'}(t) - vy^{'}(t)) dt} + i \int_{a}^{b}{(vx^{'}(t) + uy^{'}(t)) dt} \\
	&= \int_{\gamma}{(udx - vdy)} + i \int_{\gamma}{(vdx + udy)}
\end{align}

\vspace{2em}
下面给出曲线积分的几条性质.
\begin{proposition}\label{prop 4.1.2}
	记$\gamma^{-}$ 为$\gamma$ 的反向.
	\begin{enumerate}
		\item[(1)]$\int_{\gamma}{f(z) dz} = -\int_{\gamma^{-}}{f(z) dz}$.
		
		\item[(2)]If $f(z) , g(z)$ are continuous, and $\gamma$ is a path, then $\forall \alpha , \beta \in \C$,
		\begin{align}
			\int_{\gamma}{(\alpha f + \beta g) dz} = \alpha \int_{\gamma}{f dz} + \beta \int_{\gamma}{g dz}
		\end{align}
		
		\item[(3)]
		\begin{align}
			\left| \int_{\gamma}{f(z) dz} \right| \leq \sup_{\gamma}{\left| f(z) \right|} \cdot length(\gamma)
		\end{align}
	\end{enumerate}
\end{proposition}

\newpage
\paragraph{原函数}
下面我们给出\textbf{原函数}的概念.
\begin{defn}\label{def 4.1.4}
	If $f : \Omega \longrightarrow \C$. Assume $\exists$ a complex differentiable $F : \Omega \longrightarrow \C$, $\st$
	\begin{align}
		F^{'}(z) = f(z) , \,\, for \,\, every \,\, z \in \Omega
	\end{align}
	Then we say $f$ admits a \underline{\textcolor{blue}{\textbf{primitival}}} (or an antiderivative) on $\Omega$.
\end{defn}

\vspace{2em}
下面的命题说明若函数有原函数,则其\uwave{曲线积分只与始末点有关,而与路径无关}.
\begin{proposition}\label{prop 4.1.3}
	If $f$ is a continuous function that admits a primitive $F$ on $\Omega$, and $\gamma$ is a path in $\Omega$ that begins at $w_1$ and ends at $w_2$, then
	\begin{align}
		\int_{\gamma}{f(z) dz}=  F(w_2) - F(w_1)
	\end{align}
	
	\vspace{2em}
	\begin{proof}
		Let $z(t) : [a , b] \longrightarrow \Omega$ be a parametrization for $\gamma$ with $z(a) = w_1$, $z(b) = w_2$.
		\begin{itemize}
			\item Assume $\gamma$ is smooth. Compute
			\begin{align}
				\int_{\gamma}{f(z) dz} = \int_{a}^{b}{f(z(t)) z^{'}(t) dt} = \int_{a}^{b}{F^{'}(z(t)) z^{'}(t) dt} = \int_{a}^{b}{\frac{dF(z(t))}{dt} dt}
			\end{align}
			According to \textbf{the fundamental theorem of calculus}, we get\\
			(分别对实部和虚部运用微积分基本定理)
			\begin{align}
				\int_{\gamma}{f(z) dz} 
				= \int_{a}^{b}{F^{'}(z(t)) z^{'}(t) dt} 
				&= \int_{a}^{b}{\frac{dF(z(t))}{dt} dt} \\
				&= F(z(b)) - F(z(a)) = F(w_2) - F(w_1)
			\end{align}
			
			\item $\gamma$ is piecewise smooth, we can proof similarly.
		\end{itemize}
	\end{proof}
\end{proposition}

\vspace{2em}
由命题 \ref{prop 4.1.3},可得到有原函数的函数$f$ 在闭曲线上积分为0.
\begin{corollary}\label{cor 4.1.1}
	If $\gamma$ is a closed path in $\Omega$, $f$ is continuous and admits a primitive on $\Omega$, then
	\begin{align}
		\int_{\gamma}{f(z) dz} = 0
	\end{align}
\end{corollary}

\newpage
同时,由命题 \ref{prop 4.1.3},还可得到区域$\Omega$ 上导数恒为0的全纯函数只能为常值函数.
\begin{corollary}\label{cor 4.1.2}
	If $f$ is holomorphic on a region $\Omega$ and $f^{'} \equiv 0$, then $f$ is constant.
\end{corollary}

\vspace{2em}
下面给出具有原函数的充要条件.
\begin{thm}\label{thm 4.1.3}
	Let $\Omega \subset \C$ be a region. $f : \Omega \longrightarrow \C$ be a continuous function. Then the following statements are equivalent:
	\begin{enumerate}
		\item[(1)]$f$ admits a primitive on $\Omega$.
		
		\item[(2)]$\forall \alpha , \beta \in \C$, $\int_{\gamma}{f(z) dz}$ is invariant for any path $\gamma$ in $\Omega$ that joins $\alpha$ to $\beta$.
		
		\item[(3)]$\forall \alpha , \beta \in \C$, $\int_{\gamma}{f(z) dz}$ is invariant for any zig-zag path $\gamma$ in $\Omega$ that joins $\alpha$ to $\beta$.
		
		\vspace{1em}
		\begin{rmk}
			我们将在\textbf{定理 \ref{thm 5.2.5}} 中补充一条具有原函数的充要条件. (详见\textbf{Thm \ref{thm 5.2.5} (4)} )
		\end{rmk}
		
		\vspace{2em}
		\begin{proof}
			(1) $\Rightarrow$ (2) and (2) $\Rightarrow$ (3) are clear.
			\begin{enumerate}
				\item[(3) $\Rightarrow$ (1):]Fix $\alpha \in \Omega$ and define $F : \Omega \longrightarrow \C$ by 
				\begin{align}
					F(z_0) = \int_{\gamma}{f(z) dz} , \,\, z_0 = x_0 + i y_0 \in \Omega
				\end{align}
				where $\gamma$ is any zig-zag path joining $\alpha$ to $z_0$.
				\begin{center}
					(\textbf{$F$ is Well-defined} : Condition (3) tells $F(z_0)$ is independent of the choice of $\gamma$.)
				\end{center}
				
				\vspace{1.5em}
				Let $F(z) = U + i V$, $f(z) = u + i v$. It suffices to show
				\begin{align}
					\begin{cases}
						U_{x}(x_0 , y_0) &= u(x_0 , y_0) \,\, , \,\, V_{x}(x_0 , y_0) = v(x_0 , y_0) \\
						U_{y}(x_0 , y_0) &= -v(x_0 , y_0) \,\, , \,\, V_{y}(x_0 , y_0) = u(x_0 , y_0)
					\end{cases}
				\end{align}
				
				\vspace{1.5em}
				\begin{itemize}
					\item \textcolor{red}{$U_{x}(x_0 , y_0) = u(x_0 , y_0) \,\, , \,\, V_{x}(x_0 , y_0) = v(x_0 , y_0)$}\\
					Let $h \in \R$. Let $\gamma$ be a zig-zag path joining $\alpha$ to $z_0$, \\
					$\gamma_{H} : z_{H}(t) = z_0 + th$, $0 \leq t \leq 1$. $\gamma_H \subset \Omega$. Then
					\begin{align}
						F(z_0 + h) = \int_{\gamma \circ \gamma_H}{f(z) dz} 
						&= \int_{\gamma}{f(z) dz} + \int_{\gamma_H}{f(z) dz} \\
						&= F(z_0) + \int_{\gamma_H}{f(z) dz}
					\end{align}
					Then we get
					\begin{align}
						\frac{F(z_0 + h) - F(z_0)}{h} = \int_{0}^{1}{f(z_0 + th) dt}
					\end{align}
					Since $f$ is continuous,
					\begin{align}
						\lim_{\substack{h \to 0 \\ h \, \in \, \R}}{\frac{F(z_0 + h) - F(z_0)}{h}} 
						&= \lim_{\substack{h \to 0 \\ h \, \in \, \R}}{\int_{0}^{1}{f(z_0 + th) dt}} \\
						&= f(z_0) \\
						&= u(x_0 , y_0) + i v(x_0 , y_0)
					\end{align}
					
					\item \textcolor{red}{$U_{y}(x_0 , y_0) = -v(x_0 , y_0) \,\, , \,\, V_{y}(x_0 , y_0) = u(x_0 , y_0)$}\\
					Similarly.
				\end{itemize}
			\end{enumerate}
		\end{proof}
	\end{enumerate}
\end{thm}

\newpage
\section{课堂例题$2024-03-18$}
\begin{enumerate}
	\item Let $f(x + iy) = x$. Consider two paths $\gamma_1 , \gamma_2$ joining 0 to 1.
	\begin{align}
		z_{1}(t) &= t , \,\, 0 \leq t \leq 1 \\
		z_{2}(t) &= 
		\begin{cases}
			t + 2t i , \,\, 0 \leq t \leq \frac{1}{2} \\
			t + 2(1 - t)i , \,\, \frac{1}{2} \leq t \leq 1
		\end{cases}
	\end{align}
	Evaluate
	\begin{align}
		\int_{\gamma_1}{f(z) dz} &, \,\, \int_{\gamma_2}{f(z) dz} \\
		( = \frac{1}{2} &, \,\, = \frac{1 - i}{2})
	\end{align}
	
	\vspace{2em}
	\item Let $f(z) = \frac{1}{z}$, $z \in \C \backslash \{ 0 \}$. Let $\gamma : z(t) = R e^{it}$, $R > 0$, $0 \leq t \leq 2\pi$.\\
	Evaluate
	\begin{align}
		\int_{\gamma}{f(z) dz} \,\, (= 2\pi i)
	\end{align}
	
	\vspace{2em}
	\item Let $f(z) = z^3$ and let $\sigma$ be any path joining 1 to $2 + i$. Evaluate
	\begin{align}
		\int_{\gamma}{f(z) dz}
	\end{align}
	
	\vspace{2em}
	\item 课本第一章练习$T26$.
\end{enumerate}

\newpage
\section{$Cauchy's \,\, Theorem$}
\begin{center}
	这节我们来介绍一个重要的定理——\textbf{Cauchy's Theorem}.
\end{center}

\paragraph{单连通}
下面先给出一个定理并借此给出曲线的\textbf{内部}和\textbf{外部}的定义.
\begin{thm}\label{thm 4.3.1}
	\underline{\textbf{Jordan Curve Therom}}.\\
	Let $\gamma$ be a simple closed curve on $\C$. Then $\C \backslash \gamma$ has two connected components. The bounded component is called the \underline{\textcolor{blue}{\textbf{interior of $\gamma$}}} and the unbounded component is called the \underline{\textcolor{blue}{\textbf{exterior of $\gamma$}}}.\\
	
	If the simple closed path $\gamma$ is positively oriented, then Interior($\gamma$) is to the \textbf{left} while traversing $\gamma$.
	
	\vspace{2em}
	\begin{proof}
		证明见书\footnote{课堂教材:\textbf{《$Complex \,\, Analysis$》---  $Elias \,\, M. \,\, Stein$}}$Page \,\, 351$
	\end{proof}
\end{thm}

\vspace{2em}
下面给出\textbf{单连通集}的定义.
\begin{defn}\label{def 4.3.1}
	A region $\Omega \subset \C$ is \underline{\textcolor{blue}{\textbf{simply connected}}} if for every closed path $\gamma \subset \Omega$, $Interior(\gamma) \subset \Omega$.
\end{defn}

\vspace{2em}
\begin{example}\label{ex 4.3.1}
	下面给出几个常见的单连通集 / 非单连通集的例子.
	\begin{itemize}
		\item $\C$, $D_{r}(z_0)$, $r > 0 , z_0 \in \C$ are simply connected.
		
		\item $\C \backslash \{ 0 \}$, $D_{r}^{*}(z_0)$ are not simply connected.
		
		\item $\C \backslash \R_{\leq 0}$ is simply connected.
	\end{itemize}
\end{example}

\newpage
\paragraph{Cauchy's Theorem}
下面介绍\textbf{Cauchy's Theorem}.
\begin{thm}\label{thm 4.3.2}
	Let $\Omega \subset \C$ be \textcolor{red}{\textbf{simply connected}}, $f : \Omega \longrightarrow \C$ be holomorphic. Then for any closed path $\gamma$,
	\begin{align}
		\int_{\gamma}{f(z) dz} = 0
	\end{align}
	
	\vspace{1em}
	\begin{rmk}
		条件中$\Omega$ 单连通\textcolor{red}{\textbf{不可省略}}. 下面给出反例.
		\begin{example}\label{ex 4.3.2}
			Let $\Omega = \C \backslash \{ 0 \}$, $f(z) = \frac{1}{z}$. Then $f$ is holomorphic on $\Omega$ and ($C_{1}(0)$ 表示单位圆周)
			\begin{align}
				\int_{C_{1}(0)}{f(z) dz} = 2\pi i \neq 0
			\end{align}
		\end{example}
	\end{rmk}
	
	\vspace{2em}
	\begin{proof}
		The result would follow if \uwave{$f$ has a primitive on $\Omega$}. \\
		By Thm \ref{thm 4.1.3}, thus it suffices to show that for any two zig-zag paths $\gamma_1 , \gamma_2$ having the same starting points and ending points,
		\begin{align}
			&\int_{\gamma_1}{f(z) dz} = \int_{\gamma_2}{f(z) dz} \\
			&i.e. \,\, \int_{\gamma_{1} \circ \gamma_{2}^{-1}}{f(z) dz} = 0
		\end{align}
		Equivalently, we have to show $\int_{\gamma}{f(z) dz} = 0$ for any closed zig-zag path $\gamma$.\\
		By concatecnating some horizontal or vertical paths, any closed zig-zag path is the union of rectangle paths.\\
		Thus we are done if we can show $\int_{R}{f(z) dz} = 0$, where $R$ is a rectangle path.\\
		Note that 
		\begin{align}
			\int_{R}{f(z) dz} = \int_{T_1}{f(z) dz} + \int_{T_2}{f(z) dz}
		\end{align}
		Then the theorem boils down to showing $\int_{T}{f(z) dz} = 0$ for any triangle path $T$ in $\Omega$.
		\begin{center}
			(This is Coursat Theorem on P34.)
		\end{center}
	\end{proof}
\end{thm}

\newpage
下面给出\textbf{Cauchy's Theorem} 的另一种叙述,\textcolor{red}{\textbf{这里并不对$f$ 的定义域$\Omega$ 做单连通要求}}.
\begin{thm}\label{thm 4.3.3}
	Let $\gamma$ be a simple closed path. If $f$ is holomorphic in Interior($\gamma$) and continuous on $\gamma$, then
	\begin{align}
		\int_{\gamma}{f(z) dz} = 0
	\end{align}
\end{thm}

\newpage
\section{课堂例题$2024-03-22$}
\begin{enumerate}
	\item 课本第一章练习$T25$.
	
	\vspace{2em}
	
	\item 课本第二章练习$T5 , T6$.
\end{enumerate}






	%  ############################
	\ifx\allfiles\undefined
\end{document}
\fi