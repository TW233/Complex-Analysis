\ifx\allfiles\undefined
\documentclass[12pt, a4paper,oneside, UTF8]{ctexbook}
\usepackage[dvipsnames]{xcolor}
\usepackage{amsmath}   % 数学公式
\usepackage{amsthm}    % 定理环境
\usepackage{amssymb}   % 更多公式符号
\usepackage{graphicx}  % 插图
%\usepackage{mathrsfs}  % 数学字体
%\usepackage{newtxtext,newtxmath}
%\usepackage{arev}
\usepackage{kmath,kerkis}
\usepackage{newtxtext}
\usepackage{bbm}
\usepackage{enumitem}  % 列表
\usepackage{geometry}  % 页面调整
%\usepackage{unicode-math}
\usepackage[colorlinks,linkcolor=black]{hyperref}


\usepackage{ulem}	   % 用于更多的下划线格式,
					   % \uline{}下划线,\uuline{}双下划线,\uwave{}下划波浪线,\sout{}中间删除线,\xout{}斜删除线
					   % \dashuline{}下划虚线,\dotuline{}文字底部加点


\graphicspath{ {flg/},{../flg/}, {config/}, {../config/} }  % 配置图形文件检索目录
\linespread{1.5} % 行高

% 页码设置
\geometry{top=25.4mm,bottom=25.4mm,left=20mm,right=20mm,headheight=2.17cm,headsep=4mm,footskip=12mm}

% 设置列表环境的上下间距
\setenumerate[1]{itemsep=5pt,partopsep=0pt,parsep=\parskip,topsep=5pt}
\setitemize[1]{itemsep=5pt,partopsep=0pt,parsep=\parskip,topsep=5pt}
\setdescription{itemsep=5pt,partopsep=0pt,parsep=\parskip,topsep=5pt}

% 定理环境
% ########## 定理环境 start ####################################
\theoremstyle{definition}
\newtheorem{defn}{\indent 定义}[section]

\newtheorem{lemma}{\indent 引理}[section]    % 引理 定理 推论 准则 共用一个编号计数
\newtheorem{thm}[lemma]{\indent 定理}
\newtheorem{corollary}[lemma]{\indent 推论}
\newtheorem{criterion}[lemma]{\indent 准则}

\newtheorem{proposition}{\indent 命题}[section]
\newtheorem{example}{\indent \color{SeaGreen}{例}}[section] % 绿色文字的 例 ,不需要就去除\color{SeaGreen}{}
\newtheorem*{rmk}{\indent \color{red}{注}}

% 两种方式定义中文的 证明 和 解 的环境:
% 缺点:\qedhere 命令将会失效【技术有限,暂时无法解决】
\renewenvironment{proof}{\par\textbf{证明.}\;}{\qed\par}
\newenvironment{solution}{\par{\textbf{解.}}\;}{\qed\par}

% 缺点:\bf 是过时命令,可以用 textb f等替代,但编译会有关于字体的警告,不过不影响使用【技术有限,暂时无法解决】
%\renewcommand{\proofname}{\indent\bf 证明}
%\newenvironment{solution}{\begin{proof}[\indent\bf 解]}{\end{proof}}
% ######### 定理环境 end  #####################################

% ↓↓↓↓↓↓↓↓↓↓↓↓↓↓↓↓↓ 以下是自定义的命令  ↓↓↓↓↓↓↓↓↓↓↓↓↓↓↓↓

% 用于调整表格的高度  使用 \hline\xrowht{25pt}
\newcommand{\xrowht}[2][0]{\addstackgap[.5\dimexpr#2\relax]{\vphantom{#1}}}

% 表格环境内长内容换行
\newcommand{\tabincell}[2]{\begin{tabular}{@{}#1@{}}#2\end{tabular}}

% 使用\linespread{1.5} 之后 cases 环境的行高也会改变,重新定义一个 ca 环境可以自动控制 cases 环境行高
\newenvironment{ca}[1][1]{\linespread{#1} \selectfont \begin{cases}}{\end{cases}}
% 和上面一样
\newenvironment{vx}[1][1]{\linespread{#1} \selectfont \begin{vmatrix}}{\end{vmatrix}}

\def\d{\textup{d}} % 直立体 d 用于微分符号 dx
\def\R{\mathbb{R}} % 实数域
\def\N{\mathbb{N}} % 自然数域
\def\C{\mathbb{C}} % 复数域
\def\Z{\mathbb{Z}} % 整数环
\def\Q{\mathbb{Q}} % 有理数域
\newcommand{\bs}[1]{\boldsymbol{#1}}    % 加粗,常用于向量
\newcommand{\ora}[1]{\overrightarrow{#1}} % 向量

% 数学 平行 符号
\newcommand{\pll}{\kern 0.56em/\kern -0.8em /\kern 0.56em}

% 用于空行\myspace{1} 表示空一行 填 2 表示空两行  
\newcommand{\myspace}[1]{\par\vspace{#1\baselineskip}}

%s.t. 用\st就能打出s.t.
\DeclareMathOperator{\st}{s.t.}

%罗马数字 \rmnum{}是小写罗马数字, \Rmnum{}是大写罗马数字
\makeatletter
\newcommand{\rmnum}[1]{\romannumeral #1}
\newcommand{\Rmnum}[1]{\expandafter@slowromancap\romannumeral #1@}
\makeatother
\begin{document}
	% \title{{\Huge{\textbf{$Complex \,\, Analysis$\footnote{课堂教材:\textbf{《$Complex \,\, Analysis$》---  $Elias \,\, M. \,\, Stein$}}}}}}
\author{$-TW-$}
\date{\today}
\maketitle                   % 在单独的标题页上生成一个标题

\thispagestyle{empty}        % 前言页面不使用页码
\begin{center}
	\Huge\textbf{序}
\end{center}


\vspace*{3em}
\begin{center}
	\large{\textbf{天道几何,万品流形先自守;}}\\
	
	\large{\textbf{变分无限,孤心测度有同伦。}}
\end{center}

\vspace*{3em}
\begin{flushright}
	\begin{tabular}{c}
		\today \\ \small{\textbf{长夜伴浪破晓梦,梦晓破浪伴夜长}}
	\end{tabular}
\end{flushright}


\newpage                      % 新的一页
\pagestyle{plain}             % 设置页眉和页脚的排版方式(plain:页眉是空的,页脚只包含一个居中的页码)
\setcounter{page}{1}          % 重新定义页码从第一页开始
\pagenumbering{Roman}         % 使用大写的罗马数字作为页码
\tableofcontents              % 生成目录

\newpage                      % 以下是正文
\pagestyle{plain}
\setcounter{page}{1}          % 使用阿拉伯数字作为页码
\pagenumbering{arabic}
\setcounter{chapter}{-1}    % 设置 -1 可作为第零章绪论从第零章开始 
	\else
	\fi
	%  ############################ 正文部分

\chapter{$Week \,\, 12$}
\section{复对数}
	此节我们来回顾一下$\S 3.1, 3.3$ 节中有关\textbf{复对数}的有关概念,并给出一些与之相关的结论.
	\begin{defn}\label{def 12.1.1}
		$\forall z \in \C$, the solution to $e^w = z$ is given by
		\begin{align}
			Log(z) = \log{\left| z \right|} + i (\arg{z} + 2k\pi), \,\, k \in \N
		\end{align}
		A holomorphic function $g : U \rightarrow \C$ is called \underline{\textcolor{blue}{\textbf{a branch of Logz on U}}} if
		\begin{center}
			$e^{g(z)} = z$ for all $z \in U$
		\end{center}
	\end{defn}

\vspace*{4em}

	下面给出一个常用结论及其推论.
	\begin{proposition}\label{prop 12.1.1}
		Let $\Omega \subset \C$ be simply connected, $f : \Omega \rightarrow \C$ be holomorphic. Assume $f(z) \neq 0$, $\forall z \in \Omega$(\textbf{nonvanishing}). Then $\exists$ holomorphic $g : \Omega \rightarrow \C$, $\st$
		\begin{align}
			e^{g(z)} = f(z)
		\end{align}
	\end{proposition}
	
	\vspace*{2em}
	
	\begin{corollary}\label{cor 12.1.1}
		Let $\Omega \subset \C$ be simply connected and let $f : \Omega \rightarrow \C$ be holomorphic and \textbf{nonvanishing} on $\Omega$. Then $\exists$ holomorphic $g : \Omega \rightarrow \C$, $\st$
		\begin{align}
			\left( g(z) \right)^2 = f(z)
		\end{align}
	\end{corollary}

\newpage
\section{$Conformal \,\, Mappings$}
	下面进入新的非常有意思的一章内容——\textbf{共形映射(Conformal Mappings)}. 这一章我们将从几何的角度,研究两个复平面的特定区域之间的\textbf{几何变换关系}.
	
	\vspace*{1em}
	先来给出\textbf{共形映射}的定义.
	\begin{defn}\label{def 12.2.1}
		A bijective holomorphic function $f : U \rightarrow V$ is called a \underline{\textcolor{blue}{\textbf{conformal map}}} or \underline{\textcolor{blue}{\textbf{biholomorphism}}}. Given such a mapping $f$, we say that $U \& V$ are \underline{\textcolor{blue}{\textbf{conformally equivalent}}}.
		
		\vspace*{2em}
		\begin{rmk}
			即我们将\textbf{全纯双射}叫做\textbf{共形映射}. 事实上,通过下面的\textbf{Prop \ref{prop 12.2.1}},可以容易得到共形映射$f : U \rightarrow V$ 事实上给出了复平面$\C$ 的所有子区域类上的一个\textbf{等价关系},即
			\begin{center}
				$U \sim V \,\, \Leftrightarrow \,\, \exists$ conformal map $f : U \rightarrow V$
			\end{center}
			而我们把这个\textbf{等价关系}称作是\textbf{共性等价 (conformally equivalent)}.
		\end{rmk}
	\end{defn}
	
	\vspace*{8em}
	下面给出一个强有力的命题,它给出了\textbf{全纯单射的逆映射的全纯性}.
	\begin{proposition}\label{prop 12.2.1}
		If $f : U \rightarrow V$ is holomorphic and injective, then
		\begin{center}
			$f^{'}(z) \neq 0$, $\forall z \in U$
		\end{center}
		Moreover, the inverse of $f$ defined on its range(i.e. $f(U)$) is holomorphic, and thus the inverse of a conformal mapping is holomocphic.
	\end{proposition}

\newpage
\section{上半平面$\mathbb{H}$ 到$\mathbb{D}$ 的共形映射}
	先给出一个我们后面常用的记号:
	\begin{align}
		\mathbb{H} = \{ z \in \C \mid Im(z) > 0 \}
	\end{align}
	
	\vspace*{4em}
	下面给出一个上半平面$\mathbb{H}$ 到$\mathbb{D}$ 的共形映射.
	\begin{thm}\label{thm 12.3.1}
		The map following is a conformal map.
		\begin{align}
			F : \mathbb{H} &\longrightarrow \mathbb{D} \\
			z &\longmapsto \frac{-z + i}{z + i}
		\end{align}
	\end{thm}

\newpage
\section{Exercise 2024-05-13}
	\begin{enumerate}
		\item P248 T1. 
		
		\vspace*{2em}
		
		\item P251 T13.
		
		\vspace*{2em}
		
		\item P209 $\S 1.2$ 例1 - 8.
	\end{enumerate}




	%  ############################
	\ifx\allfiles\undefined
\end{document}
\fi