\ifx\allfiles\undefined
\documentclass[12pt, a4paper,oneside, UTF8]{ctexbook}
\usepackage[dvipsnames]{xcolor}
\usepackage{amsmath}   % 数学公式
\usepackage{amsthm}    % 定理环境
\usepackage{amssymb}   % 更多公式符号
\usepackage{graphicx}  % 插图
%\usepackage{mathrsfs}  % 数学字体
%\usepackage{newtxtext,newtxmath}
%\usepackage{arev}
\usepackage{kmath,kerkis}
\usepackage{newtxtext}
\usepackage{bbm}
\usepackage{enumitem}  % 列表
\usepackage{geometry}  % 页面调整
%\usepackage{unicode-math}
\usepackage[colorlinks,linkcolor=black]{hyperref}


\usepackage{ulem}	   % 用于更多的下划线格式,
					   % \uline{}下划线,\uuline{}双下划线,\uwave{}下划波浪线,\sout{}中间删除线,\xout{}斜删除线
					   % \dashuline{}下划虚线,\dotuline{}文字底部加点


\graphicspath{ {flg/},{../flg/}, {config/}, {../config/} }  % 配置图形文件检索目录
\linespread{1.5} % 行高

% 页码设置
\geometry{top=25.4mm,bottom=25.4mm,left=20mm,right=20mm,headheight=2.17cm,headsep=4mm,footskip=12mm}

% 设置列表环境的上下间距
\setenumerate[1]{itemsep=5pt,partopsep=0pt,parsep=\parskip,topsep=5pt}
\setitemize[1]{itemsep=5pt,partopsep=0pt,parsep=\parskip,topsep=5pt}
\setdescription{itemsep=5pt,partopsep=0pt,parsep=\parskip,topsep=5pt}

% 定理环境
% ########## 定理环境 start ####################################
\theoremstyle{definition}
\newtheorem{defn}{\indent 定义}[section]

\newtheorem{lemma}{\indent 引理}[section]    % 引理 定理 推论 准则 共用一个编号计数
\newtheorem{thm}[lemma]{\indent 定理}
\newtheorem{corollary}[lemma]{\indent 推论}
\newtheorem{criterion}[lemma]{\indent 准则}

\newtheorem{proposition}{\indent 命题}[section]
\newtheorem{example}{\indent \color{SeaGreen}{例}}[section] % 绿色文字的 例 ,不需要就去除\color{SeaGreen}{}
\newtheorem*{rmk}{\indent \color{red}{注}}

% 两种方式定义中文的 证明 和 解 的环境:
% 缺点:\qedhere 命令将会失效【技术有限,暂时无法解决】
\renewenvironment{proof}{\par\textbf{证明.}\;}{\qed\par}
\newenvironment{solution}{\par{\textbf{解.}}\;}{\qed\par}

% 缺点:\bf 是过时命令,可以用 textb f等替代,但编译会有关于字体的警告,不过不影响使用【技术有限,暂时无法解决】
%\renewcommand{\proofname}{\indent\bf 证明}
%\newenvironment{solution}{\begin{proof}[\indent\bf 解]}{\end{proof}}
% ######### 定理环境 end  #####################################

% ↓↓↓↓↓↓↓↓↓↓↓↓↓↓↓↓↓ 以下是自定义的命令  ↓↓↓↓↓↓↓↓↓↓↓↓↓↓↓↓

% 用于调整表格的高度  使用 \hline\xrowht{25pt}
\newcommand{\xrowht}[2][0]{\addstackgap[.5\dimexpr#2\relax]{\vphantom{#1}}}

% 表格环境内长内容换行
\newcommand{\tabincell}[2]{\begin{tabular}{@{}#1@{}}#2\end{tabular}}

% 使用\linespread{1.5} 之后 cases 环境的行高也会改变,重新定义一个 ca 环境可以自动控制 cases 环境行高
\newenvironment{ca}[1][1]{\linespread{#1} \selectfont \begin{cases}}{\end{cases}}
% 和上面一样
\newenvironment{vx}[1][1]{\linespread{#1} \selectfont \begin{vmatrix}}{\end{vmatrix}}

\def\d{\textup{d}} % 直立体 d 用于微分符号 dx
\def\R{\mathbb{R}} % 实数域
\def\N{\mathbb{N}} % 自然数域
\def\C{\mathbb{C}} % 复数域
\def\Z{\mathbb{Z}} % 整数环
\def\Q{\mathbb{Q}} % 有理数域
\newcommand{\bs}[1]{\boldsymbol{#1}}    % 加粗,常用于向量
\newcommand{\ora}[1]{\overrightarrow{#1}} % 向量

% 数学 平行 符号
\newcommand{\pll}{\kern 0.56em/\kern -0.8em /\kern 0.56em}

% 用于空行\myspace{1} 表示空一行 填 2 表示空两行  
\newcommand{\myspace}[1]{\par\vspace{#1\baselineskip}}

%s.t. 用\st就能打出s.t.
\DeclareMathOperator{\st}{s.t.}

%罗马数字 \rmnum{}是小写罗马数字, \Rmnum{}是大写罗马数字
\makeatletter
\newcommand{\rmnum}[1]{\romannumeral #1}
\newcommand{\Rmnum}[1]{\expandafter@slowromancap\romannumeral #1@}
\makeatother
\begin{document}
	% \title{{\Huge{\textbf{$Complex \,\, Analysis$\footnote{课堂教材:\textbf{《$Complex \,\, Analysis$》---  $Elias \,\, M. \,\, Stein$}}}}}}
\author{$-TW-$}
\date{\today}
\maketitle                   % 在单独的标题页上生成一个标题

\thispagestyle{empty}        % 前言页面不使用页码
\begin{center}
	\Huge\textbf{序}
\end{center}


\vspace*{3em}
\begin{center}
	\large{\textbf{天道几何,万品流形先自守;}}\\
	
	\large{\textbf{变分无限,孤心测度有同伦。}}
\end{center}

\vspace*{3em}
\begin{flushright}
	\begin{tabular}{c}
		\today \\ \small{\textbf{长夜伴浪破晓梦,梦晓破浪伴夜长}}
	\end{tabular}
\end{flushright}


\newpage                      % 新的一页
\pagestyle{plain}             % 设置页眉和页脚的排版方式(plain:页眉是空的,页脚只包含一个居中的页码)
\setcounter{page}{1}          % 重新定义页码从第一页开始
\pagenumbering{Roman}         % 使用大写的罗马数字作为页码
\tableofcontents              % 生成目录

\newpage                      % 以下是正文
\pagestyle{plain}
\setcounter{page}{1}          % 使用阿拉伯数字作为页码
\pagenumbering{arabic}
\setcounter{chapter}{-1}    % 设置 -1 可作为第零章绪论从第零章开始 
	\else
	\fi
	%  ############################ 正文部分

\chapter{$Week \,\, 8$}
\section{均值定理}
	下面补充一个\textbf{CIF (Thm \ref{thm 5.2.1}, 柯西积分公式)}的推论,即\textbf{均值定理}.
	\begin{proposition}\label{prop 8.1.1}
		\textbf{(Mean Value Property.)} \\
		If $f$ is holomorphic on $D_{R}(z_0)$, where $z_0 \in \C$, $R > 0$, then
		\begin{align}
			f(z_0) = \frac{1}{2 \pi}\int_{0}^{2\pi}{f(z_0 + re^{i\vartheta}) d\vartheta} , \,\, 0 < r < R
		\end{align}
	
		\vspace{2em}
		\begin{proof}
			By \textbf{CIF (Thm \ref{thm 5.2.1})},
			\begin{align}
				f(z_0) = \frac{1}{2 \pi i} \int_{C_{r}(z_0)}{\frac{f(\zeta)}{\zeta - z_0} d\zeta}
				&= \frac{1}{2 \pi i} \int_{0}^{2\pi}{\frac{f(z_0 + re^{i\vartheta})}{re^{i\vartheta}} r \cdot i \cdot e^{i\vartheta} d\vartheta} \\
				&= \frac{1}{2 \pi}\int_{0}^{2\pi}{f(z_0 + re^{i\vartheta}) d\vartheta}
			\end{align}
		\end{proof}
	\end{proposition}

\newpage
\section{奇点}
	下面给出\textbf{奇点}的定义.
	\begin{defn}\label{def 8.2.1}
		A complex number $z_0$ is a \underline{\textcolor{blue}{\textbf{singular point}}} (or a \underline{\textcolor{blue}{\textbf{singularity}}}) of $f$ if $f$ is not analytic at $z_0$. We say $z_0$ is an \underline{\textcolor{blue}{\textbf{isolated singularity}}} if $f$ is analytic in a deleted neighbourhood of $z_0$.
		
		\vspace{1em}
		\begin{rmk}
			大多数情况下我们研究的都是\textbf{孤立奇点},但也存在着\textbf{非孤立奇点}.
			\begin{example}\label{ex 8.2.1}
				\begin{itemize}
					\item $0$ is an isolated singularity of $\frac{1}{z} , \frac{1}{\sin{z}} , \frac{1}{e^z - 1}$.
					
					\item \textbf{Poles are isolated singularities. (极点均为孤立奇点)}
					
					\item For 
					\begin{align}
						\frac{1}{\sin{\frac{\pi}{z}}}
					\end{align}
					$0$ is not an isolated singularity.
				\end{itemize}
			\end{example}
		\end{rmk}
	\end{defn}

\vspace{2em}
\subsection{$Classification \,\, of \,\, isolated \,\, singularities$}
	下面我们对\textbf{奇点}进行分类.
	\begin{defn}\label{def 8.2.2}
		Let $f : D_{r}^{*}(z_0) \rightarrow \C$ where $r > 0 , z_0 \in \C$ be holomorphic with the Laurent expansion
		\begin{align}
			f(z) = \sum_{-\infty}^{\infty}{a_n (z - z_0)^n}
		\end{align}
		\begin{enumerate}
			\item[(1)]$z_0$ is called a \underline{\textcolor{blue}{\textbf{removable singularity (可去奇点)}}} if $a_{-n} = 0$, $n = 1 , 2 , \cdots$.
			
			\item[(2)]$z_0$ is called a \underline{\textcolor{blue}{\textbf{pole (极点)}}} if $a_{-n} \neq 0 , a_{-(n + k)} = 0$, $k = 1 , 2 , \cdots$.
			
			\item[(3)]$z_0$ is called a \underline{\textcolor{blue}{\textbf{essential singularity (本性奇点)}}} if $a_{-n} \neq 0$ for infinitely many $n \geq 1$.
		\end{enumerate}
	
		\vspace{2em}
		\begin{example}\label{ex 8.2.2}
			\begin{itemize}
				\item $f(z) = z^{-n} , n \geq 1$ has a \textbf{pole} of order $n$ at $z = 0$.
				
				\item $f(z) = e^{\frac{1}{z}}$ has an \textbf{essential singularity} at $z = 0$.
				
				\item $f(z) = \frac{e^z - 1}{z}$ has a \textbf{removable singularity} at $z = 0$.
				\begin{proof}
					\begin{align}
						f(z) = \sum_{k = 1}^{\infty}{\frac{z^{k - 1}}{k!}}
					\end{align}
				\end{proof}
			\end{itemize}
		\end{example}
	\end{defn}

\newpage
\section{孤立奇点的等价刻画}
	\begin{center}
		下面主要研究\textbf{孤立奇点}附近的性态,并给出各类孤立奇点的等价刻画.
	\end{center}

\subsection{$Removable \,\, Singularity$}
	首先,介于\textbf{可去奇点}的良好性质,我们可在简单的操作后令函数\textbf{全纯},即:
	\begin{rmk}
		If $z_0$ is a removable singularity of $f : D_{r}^{*}(z_0) \rightarrow \C$, then we define $f(z_0) = a_0$ so that $f$ is holomorphic in $D_{r}(z_0)$, where
		\begin{align}
			f(z) = \sum_{n = 0}^{\infty}{a_n (z - z_0)^n} , \,\, z \in D_{r}^{*}(z_0)
		\end{align}
	\end{rmk}
	
	\vspace{2em}

	下面给出\textbf{可去奇点}的\textbf{等价刻画}.
	\begin{thm}\label{thm 8.3.1}
		If $z_0$ is a singularity of $f$, then 
		\begin{center}
			$z_0 \in \C$ is a removable singularity \hspace*{1em} iff \hspace*{1em} $f$ is bounded near $z_0$. \\
			(i.e. in a deleted neighbourhood of $z_0$)
		\end{center}
	
		\vspace{2em}
		\begin{proof}
			\begin{enumerate}
				\item[$``\Rightarrow"$]: We can define the value of $f$ at $z_0$ $\st$
				\begin{center}
					$f$ is holomorphic in $D_{r}(z_0)$ for some $r > 0$.
				\end{center}
				In particular, $f$ is continuous in $\overline{D_{\frac{r}{2}}(z_0)}$ and so $f$ is bounded near $z_0$.
				
				\vspace{2em}
				\item[$`` \Leftarrow "$]: Define
				\begin{align}
					g : D_{r}(z_0) &\longrightarrow \C \\
					z &\longmapsto g(z) = 
					\begin{cases}
						(z - z_0)^2 f(z) , z \neq z_0 \\
						0 , z = z_0
					\end{cases}
				\end{align}
				Then
				\begin{align}
					g^{'}(z) = 2(z - z_0)f(z) + (z - z_0)^2 f^{'}(z) , \,\, \forall z \in D_{r}^{*}(z_0)
				\end{align}
				Since $f$ is bounded near $z_0$,
				\begin{align}
					\lim_{z \to z_0}{\frac{g(z) - g(z_0)}{z - z_0}}
					= \lim_{z \to z_0}{(z - z_0)f(z)}
					= 0
				\end{align}
				Thus $g$ is holomorphic in $D_{r}(z_0)$ and so $g$ is analytic in $D_{r}(z_0)$,
				\begin{align}
					g(z) = \sum_{n = 0}^{\infty}{c_n (z - z_0)^n},  \,\, \forall z \in D_{r}(z_0)
				\end{align}
				Note that $c_0 = g(z_0) = 0$, $c_1 = g^{'}(z_0) = 0$. Thus
				\begin{align}
					g(z) = \sum_{n = 0}^{\infty}{a_n (z - z_0)^{n + 2}}, \,\, where \,\, a_n = c_{n + 2}, n = 0 , 1 , 2 , \cdots
				\end{align}
				Therefore
				\begin{align}
					f(z) = (z - z_0)^{-2} g(z) = \sum_{n = 0}^{\infty}{a_n (z - z_0)^n} , \,\, \forall z \in D_{r}^{*}(z_0)
				\end{align}
				and so $z_0$ is a removable singularity of $f$.
			\end{enumerate}
		\end{proof}
	\end{thm}

	\vspace{3em}
	从上述定理的证明过程中,可以直接得到下面的推论,也是对\textbf{可去奇点}的\textbf{等价刻画}.
	\begin{corollary}\label{cor 8.3.2}
		If $z_0$ is a singularity of $f$, then
		\begin{center}
			$z_0$ is a removable singularity \hspace*{1em} iff \hspace*{1em} $\underset{z \to z_0}{\lim}{(z - z_0)f(z)} = 0$.
		\end{center}
	\end{corollary}

\newpage
\subsection{$Pole$}
	作为\textbf{定理 \ref{thm 8.3.1}}的推论,下面我们给出\textbf{极点}的\textbf{等价刻画}.
	\begin{corollary}\label{cor 8.3.3}\footnote{对应课本P85 Cor 3.2}
		If $z_0$ is a singularity of $f$, then
		\begin{center}
			$z_0$ is a pole \hspace*{1em} iff \hspace*{1em} $\left| f(z) \right| \to \infty$ as $z \to z_0$.
		\end{center}
	
		\vspace{2em}
		\begin{proof}
			$`` \Rightarrow "$: $\frac{1}{f(z_0)} = 0 \,\, \Rightarrow \,\, \left| f(z) \right| \to \infty$ as $z \to z_0$. 
			
			\vspace{1em}
			$`` \Leftarrow "$: Suppose $\left| f(z) \right| \to \infty$ as $z \to z_0$, then $\exists r > 0$, $\st$
			\begin{center}
				$f(z) \neq 0 , \,\, \forall z \in D_{r}^{*}(z)$
			\end{center}
			and so $\frac{1}{f}$ is holomorphic on $D_{r}^{*}(z_0)$. Moreover, $\left| \frac{1}{f(z)} \right| \to 0$ as $z \to z_0$. \\
			By \textbf{Thm \ref{thm 8.3.1}}, $\frac{1}{f}$ is bounded near $z_0$,
			\begin{center}
				$\Rightarrow \,\, z_0$ is a removable singularity of $\frac{1}{f}$.
			\end{center}
			Since $\left| \frac{1}{f(z)} \right| \to 0$ as $z \to z_0$, we have
			\begin{align}
				\frac{1}{f(z)} = \sum_{n = 0}^{\infty}{a_n (z - z_0)^n} , \,\, \forall z \in D_{r}^{*}(z_0) , \,\, where \,\, a_0 = 0
			\end{align}
			Therefore, if we define $\frac{1}{f(z_0)} = 0$, then $\frac{1}{f}$ is holomorphic on $D_{r}(z_0)$. By \textbf{Def \ref{def 7.1.3}}, $z_0$ is a pole of $f$. 
		\end{proof}
	\end{corollary}

\vspace{2em}
\subsection{$Essential \,\, Singularity$}
	在排除了\textbf{定理 \ref{thm 8.3.1}}和\textbf{Cor \textbf{cor 8.3.3}}的情况后,下面我们给出\textbf{本性奇点}的\textbf{等价刻画}.
	\begin{corollary}\label{cor 8.3.4}
		If $z_0$ is a singularity of $f$, then
		\begin{center}
			$z_0$ is an essential singularity \hspace*{1em} iff \hspace*{1em} $\underset{z \to z_0}{\lim}{\left| f(z) \right|}$ does not exists. \\
			(Here we allow the limit to be $\infty$)
		\end{center}
	\end{corollary}
	
	\begin{example}\label{ex 8.3.1}
		Consider $f(z) = e^{\frac{1}{z}}$. Since 
		\begin{align}
			\lim_{\substack{z \in \R \\ z \to 0^{+}}}{e^{\frac{1}{z}}} = \infty , \,\, \lim_{\substack{z \in \R \\ z \to 0^{-}}}{e^{\frac{1}{z}}} = 0
		\end{align}
		Therefore $0$ is an essential singularity of $f$.
	\end{example}

\newpage
\section{课堂例题$2024-04-15$}
\begin{enumerate}
	\item \textbf{(课前 Question 1.)} \\
	Let $z_1 , z_2 \in \C$ with $Rez_1 \leq 0 , Rez_2 \leq 0$. Show
	\begin{center}
		$\left| e^{z_1} - e^{z_2} \right| \leq \left| z_1 - z_2 \right|$
	\end{center}

	\vspace{2em}
	
	\item \textbf{(课前 Question 2.)} \\
	If $f , g$ are entire functions that agree in infinite number of points, then $f = g$?
	
	\vspace{2em}
	\begin{solution}
		Eg: $f(z) = \sin{z} , g(z) = e^1 \sin{z}$.
	\end{solution}
	
	\vspace{2em}
	
	\item \textbf{(课前 Question 3.)} \\
	Is there a holomorphic function $f : \C \backslash \{ 0 \} \rightarrow \C$ with a simple pole at $z = 0$, $\st$
	\begin{align}
		\int_{C_{1}(0)}{f(z) dz} = 0 \,\, ?
	\end{align}

	\vspace{2em}
	\begin{solution}
		simple pole $\,\, \Rightarrow \,\,$ $Res_{0}f \neq 0$ $\,\, \Rightarrow \,\,$ $\int_{C_{1}(0)}{f(z) dz} = 2 \pi i \cdot Res_{0}f \neq 0$.
	\end{solution}

	\vspace{2em}
	
	\item 课本第三章练习$T13$.
\end{enumerate}

\newpage
\section{本性奇点附近的不稳定性}
	由\textbf{定理 \ref{thm 8.3.1}}和\textbf{推论 \ref{cor 8.3.3}}可知,函数在\textbf{可去奇点}和\textbf{极点}附近的性态是\textbf{``稳定的"},是\textbf{``有迹可循的"}. 但接下来我们将说明,全纯函数在\textbf{本性奇点}附近的性态是\textbf{``不稳定的" (erratically)},是\textbf{``无迹可寻的"}.

\vspace{2em}
\subsection{$Casorati-Weierstrass$ 定理}
	下面的定理说明了,全纯函数在\textbf{本性奇点}附近的值域是整个复数域$\C$ 的\textbf{稠密子集}.
	\begin{thm}\label{thm 8.5.1}\footnote{\textbf{《$Complex \,\, Analysis$》---  $Elias \,\, M. \,\, Stein$} P86 Thm 3.3}
		\textbf{Casorati-Weierstrass}. \\
		Let $f : D_{r}^{*}(z_0) \rightarrow \C$ be holomorphic and $z_0$ be an essential singularity of $f$. Then the image of $f$ is dense in $\C$, that is,
		\begin{center}
			$\forall w \in \C$, $\forall \epsilon > 0$, $\exists z \in D_{r}^{*}(z_0)$, $\st \left| f(z) - w \right| < \epsilon$.
		\end{center}
	
		\vspace{4em}
		\begin{proof}
			反证法. Assume $\exists w \in \C$, $\delta > 0$, $\st \forall z \in D_{r}^{*}(z_0)$, $\left| f(z) - w \right| \geq \delta$. \\
			Define $g(z) = \frac{1}{f(z) - w}$ on $D_{r}^{*}(z_0)$. Then $g(z)$ is holomorphic and bounded on $D_{r}^{*}(z_0)$. \\
			By \textbf{Thm \ref{thm 8.3.1}}, $z_0$ is a removable singularity of $g$. \\
			Define $g(z_0)$ $\st$ $g(z)$ is holomorphic on $D_{r}(z_0)$.
			
			\vspace{1em}
			
			\begin{itemize}
				\item If $g(z_0) \neq 0$, since
				\begin{align}
					f(z) = \frac{1}{g(z)} + w
				\end{align}
				Then $z_0$ is a removable singularity of $f$, which is a contradiction.
				
				\vspace{2em}
				
				\item If $g(z_0) = 0$, then $z_0$ is a pole of $f(z) - w$. \\
				By \textbf{Cor \ref{cor 8.3.3}}, $z_0$ is also a pole of $f(z)$, which is also a contradiction.
			\end{itemize}
		
			\vspace{1em}
			
			Therefore, the proof is complete.
		\end{proof}
	\end{thm}

\newpage
\subsection{$Picard$ 定理}
	下面给出一个有关\textbf{本性奇点附近性态}更强的结论,即\textbf{Picard定理}. 它说明了在至多除去一个点后,全纯函数在\textbf{本性奇点}附近能将复数域上每个点取无穷多次.
	
	\begin{thm}\label{thm 8.5.2}
		\textbf{Picard}. \\
		If $z_0$ is an essential singularity of $f$, then $f(z)$ can take any complex number infinitely many times with at most one exception.
	\end{thm}

\newpage
\section{亚纯函数,扩充复平面,无穷远处孤立奇点}
\subsection{亚纯函数}
	在给出了\textbf{孤立奇点}的定义和刻画后,下面来讨论一类比\textbf{全纯}条件稍弱的\textbf{亚纯函数}.
	\begin{defn}\label{def 8.6.1}
		A function $f$ on an open set $\Omega$ is \underline{\textcolor{blue}{\textbf{meromorphic}}} if $f$ is holomorphic on $\Omega$ except for a sequence of poles in $\Omega$ and the sequence has no limit in $\Omega$.
		
		\vspace{1em}
		\begin{rmk}
			\begin{itemize}
				\item 条件\textbf{``a sequence of poles"}既可为\textbf{有限序列}也可为\textbf{无穷序列}.
				
				\item 条件\textbf{``the sequence has no limit in $\Omega$"}说明极点序列在$\Omega$ 中\textbf{无聚点}.
			\end{itemize}
		\end{rmk}
	\end{defn}

	\vspace{2em}
	
	\begin{example}\label{ex 8.6.1}
		下面给出几个\textbf{亚纯函数}的例子.
		\begin{itemize}
			\item $f(z) = \frac{1}{\sin{z}}$ is a meromorphic function on $\C$.
			
			\vspace{2em}
			\begin{proof}
				$\sin{z} = 0 \,\, \Leftrightarrow \,\, z = k \pi , k \in \Z$
			\end{proof}
			
			\vspace{2em}
			
			\item \begin{align}
				g(z) = \frac{1}{\sin{\frac{1}{1 - z}}}
			\end{align}
			is a meromorphic function on $\mathbb{D}$. However, $\forall \epsilon > 0$, $g(z)$ is not meromorphic on $\mathbb{D}_{1 + \epsilon}(0)$.
			
			\vspace{2em}
			\begin{proof}
				Poles of $g(z)$ are $1 - \frac{1}{k\pi}$, $k = 1 , 2 , \cdots$.
			\end{proof}
		\end{itemize}
	\end{example}

\newpage
\subsection{扩充复平面}
	与扩充实数系概念类似,下面给出\textbf{扩充复平面}的定义.
	\begin{defn}\label{def 8.6.2}
		The \underline{\textcolor{blue}{\textbf{extended complex plane}}}
		\begin{center}
			$\hat{\C} \coloneqq \C \cup \{ \infty \}$
		\end{center}
	
		\vspace{1em}
		
		\begin{rmk}
			由拓扑知识可知,对$\C$ 添加元素$\infty$ 后进行\textbf{单点紧致化}\footnote{可参考书籍\textbf{《基础拓扑学讲义》--- 尤承业} P60 习题18、19}后,可得到新的拓扑空间与球面$S^2$ 同胚,即
			\begin{center}
				$\hat{\C} \cong S^2$
			\end{center}
			此时可将\textbf{扩充复平面$\hat{\C}$} 视作一个\textbf{球面},其中北极点为无穷远点$\infty$.
		\end{rmk}
	\end{defn}
	
\vspace{4em}
\subsection{无穷远处孤立奇点}
\paragraph{定义}
	首先给出\textbf{无穷远处孤立奇点}的定义.
	\begin{defn}\label{def 8.6.3}
		Suppose $f : \hat{\C} \rightarrow \hat{\C}$ be a function on $\hat{\C}$. We say $\infty$ is an \underline{\textcolor{blue}{\textbf{isolated singularity}}} of $f$ if $\exists R > 0$, $\st$ 
		\begin{center}
			f is holomorphic on $\{ w \in \C \mid \left| w \right| > R \}$.
		\end{center}
		The set $\{ w \in \C \mid \left| w \right| > R \}$ for some $R > 0$ is called a \underline{\textcolor{blue}{\textbf{deleted neighbourhood}}} of $\infty$.
		
		\vspace{2em}
		
		\begin{rmk}
			\begin{itemize}
				\item 此处若将\textbf{扩充复平面$\hat{\C}$} 视作球面,则有关\textbf{无穷远处去心邻域}的定义是自然的.
				
				\vspace{1em}
				
				\item 与数学分析处理方法类似,$f$ 在$\infty$ 附近的性质与函数$F(z) = f(\frac{1}{z})$ 在$z = 0$ 处性质等价.
			\end{itemize}
		\end{rmk}
	\end{defn}

	\newpage
	下面利用$F(z) = f(\frac{1}{z})$ 的转化,我们给出\textbf{无穷远处可去奇点、极点、本性奇点}的定义.
	\begin{defn}\label{def 8.6.4}
		Suppose $f : \C \rightarrow \hat{\C}$ be a function on $\hat{\C}$. We say that 
		\begin{itemize}
			\item $f$ has a \underline{\textcolor{blue}{\textbf{removable singularity}}} at $\infty$ if $F$ has a removable singularity at $0$.
			
			\item $f$ has a \underline{\textcolor{blue}{\textbf{pole}}} at $\infty$ if $F$ has a pole at $0$.
			
			\item $f$ has an \underline{\textcolor{blue}{\textbf{essential singularity}}} at $\infty$ if $F$ has an essential singularity at $0$.
		\end{itemize}
	\end{defn}

	\vspace{2em}
	\begin{example}\label{ex 8.6.2}
		下面给出几个\textbf{无穷远处孤立奇点}的例子.
		\begin{itemize}
			\item $p(z) = z^n + a_1 z^{n - 1} + \cdots + a_{n - 1}z + a_n$, where $a_k \in \C$, has a pole of order $n$ at $\infty$.
			
			\vspace{1em}
			\begin{proof}
				Consider $F(z) = p(\frac{1}{z}) = \frac{1}{z^n} + \cdots + \frac{a_{n - 1}}{z} + a_n$. Then $F(z)$ has a pole of order $n$ at $0$.
			\end{proof}
		
			\vspace{2em}
			
			\item $e^z , \sin{z} , \cos{z}$ all have essential singularities at $\infty$.
		\end{itemize}
	\end{example}

\vspace{2em}
\paragraph{无穷远处孤立奇点的等价刻画}
	下面给出\textbf{无穷远处孤立奇点的等价刻画}.
	\begin{proposition}\label{prop 8.6.1}
		Suppose $f : \C \rightarrow \C$ is an entire function. Then
		\begin{enumerate}
			\item[(1)]$f$ has a pole at $\infty$ iff $f$ is a nonconstant polynomial.
			
			\item[(2)]$f$ has a removable singularity at $\infty$ iff $f$ is a constant.
		\end{enumerate}
	
		\vspace{2em}
		\begin{proof}
			Since $f$ is entire, its Taylor Expansion $f(z) = \overset{\infty}{\underset{n = 0}{\sum}}{a_n z^n}$ converges for all $z \in \C$. \\
			By definition,
			\begin{enumerate}
				\item[(1)]$f$ has a pole at $\infty$ iff $F(z) = f(\frac{1}{z}) = \overset{\infty}{\underset{n = 0}{\sum}}{a_n z^{-n}}$ has a pole at $0$,  \\
				which is equivalent to the existence of a positive integer $N$ $\st$ $a_{N} \neq 0 , a_{N + k} = 0 , k = 1 , 2 , \cdots$. \\
				Then $f(z) = F(\frac{1}{z}) = \overset{N}{\underset{n = 0}{\sum}}{a_n z^n}$ is a nonconstant polynomial.
				
				\vspace{1em}
				
				\item[(2)]$f$ has a removable singularity at $\infty$ iff $F(z) = f(\frac{1}{z}) = \overset{\infty}{\underset{n = 0}{\sum}}{a_n z^{-n}}$ has a removable singularity at $0$, \\
				which is equivalent to $a_k = 0$, $k = 1 , 2 , \cdots$. \\
				Then $f(z) = F(\frac{1}{z}) = a_0$ is constant.
			\end{enumerate}
		\end{proof}
	\end{proposition}

\newpage
\section{课堂例题$2024-04-19$}
	\begin{enumerate}
		\item 课本第三章练习$T14$、$T15$.
	\end{enumerate}
	






	%  ############################
	\ifx\allfiles\undefined
\end{document}
\fi