\ifx\allfiles\undefined
\input{../config/config}
\begin{document}
	% \title{{\Huge{\textbf{$Complex \,\, Analysis$\footnote{课堂教材:\textbf{《$Complex \,\, Analysis$》---  $Elias \,\, M. \,\, Stein$}}}}}}
\author{$-TW-$}
\date{\today}
\maketitle                   % 在单独的标题页上生成一个标题

\thispagestyle{empty}        % 前言页面不使用页码
\begin{center}
	\Huge\textbf{序}
\end{center}


\vspace*{3em}
\begin{center}
	\large{\textbf{天道几何,万品流形先自守;}}\\
	
	\large{\textbf{变分无限,孤心测度有同伦。}}
\end{center}

\vspace*{3em}
\begin{flushright}
	\begin{tabular}{c}
		\today \\ \small{\textbf{长夜伴浪破晓梦,梦晓破浪伴夜长}}
	\end{tabular}
\end{flushright}


\newpage                      % 新的一页
\pagestyle{plain}             % 设置页眉和页脚的排版方式(plain:页眉是空的,页脚只包含一个居中的页码)
\setcounter{page}{1}          % 重新定义页码从第一页开始
\pagenumbering{Roman}         % 使用大写的罗马数字作为页码
\tableofcontents              % 生成目录

\newpage                      % 以下是正文
\pagestyle{plain}
\setcounter{page}{1}          % 使用阿拉伯数字作为页码
\pagenumbering{arabic}
\setcounter{chapter}{-1}    % 设置 -1 可作为第零章绪论从第零章开始 
	\else
	\fi
	%  ############################ 正文部分

\chapter{$Week \,\, 8$}
\section{均值定理}
	下面补充一个\textbf{CIF (Thm \ref{thm 5.2.1}, 柯西积分公式)}的推论,即\textbf{均值定理}.
	\begin{proposition}\label{prop 8.1.1}
		\textbf{(Mean Value Property.)} \\
		If $f$ is holomorphic on $D_{R}(z_0)$, where $z_0 \in \C$, $R > 0$, then
		\begin{align}
			f(z_0) = \frac{1}{2 \pi}\int_{0}^{2\pi}{f(z_0 + re^{i\vartheta}) d\vartheta} , \,\, 0 < r < R
		\end{align}
	
		\vspace{2em}
		\begin{proof}
			By \textbf{CIF (Thm \ref{thm 5.2.1})},
			\begin{align}
				f(z_0) = \frac{1}{2 \pi i} \int_{C_{r}(z_0)}{\frac{f(\zeta)}{\zeta - z_0} d\zeta}
				&= \frac{1}{2 \pi i} \int_{0}^{2\pi}{\frac{f(z_0 + re^{i\vartheta})}{re^{i\vartheta}} r \cdot i \cdot e^{i\vartheta} d\vartheta} \\
				&= \frac{1}{2 \pi}\int_{0}^{2\pi}{f(z_0 + re^{i\vartheta}) d\vartheta}
			\end{align}
		\end{proof}
	\end{proposition}

\newpage
\section{奇点}
	下面给出\textbf{奇点}的定义.
	\begin{defn}\label{def 8.2.1}
		A complex number $z_0$ is a \underline{\textcolor{blue}{\textbf{singular point}}} (or a \underline{\textcolor{blue}{\textbf{singularity}}}) of $f$ if $f$ is not analytic at $z_0$. We say $z_0$ is an \underline{\textcolor{blue}{\textbf{isolated singularity}}} if $f$ is analytic in a deleted neighbourhood of $z_0$.
		
		\vspace{1em}
		\begin{rmk}
			大多数情况下我们研究的都是\textbf{孤立奇点},但也存在着\textbf{非孤立奇点}.
			\begin{example}\label{ex 8.2.1}
				\begin{itemize}
					\item $0$ is an isolated singularity of $\frac{1}{z} , \frac{1}{\sin{z}} , \frac{1}{e^z - 1}$.
					
					\item \textbf{Poles are isolated singularities. (极点均为孤立奇点)}
					
					\item For 
					\begin{align}
						\frac{1}{\sin{\frac{\pi}{z}}}
					\end{align}
					$0$ is not an isolated singularity.
				\end{itemize}
			\end{example}
		\end{rmk}
	\end{defn}

\vspace{2em}
\subsection{\textbf{Classification of isolated singularities}}
	下面我们对\textbf{奇点}进行分类.
	\begin{defn}\label{def 8.2.2}
		Let $f : D_{r}^{*}(z_0) \rightarrow \C$ where $r > 0 , z_0 \in \C$ be holomorphic with the Laurent expansion
		\begin{align}
			f(z) = \sum_{-\infty}^{\infty}{a_n (z - z_0)^n}
		\end{align}
		\begin{enumerate}
			\item[(1)]$z_0$ is called a \underline{\textcolor{blue}{\textbf{removable singularity (可去奇点)}}} if $a_{-n} = 0$, $n = 1 , 2 , \cdots$.
			
			\item[(2)]$z_0$ is called a \underline{\textcolor{blue}{\textbf{pole (极点)}}} if $a_{-n} \neq 0 , a_{-(n + k)} = 0$, $k = 1 , 2 , \cdots$.
			
			\item[(3)]$z_0$ is called a \underline{\textcolor{blue}{\textbf{essential singularity (本性奇点)}}} if $a_{-n} \neq 0$ for infinitely many $n \geq 1$.
		\end{enumerate}
	
		\vspace{2em}
		\begin{example}\label{ex 8.2.2}
			\begin{itemize}
				\item $f(z) = z^{-n} , n \geq 1$ has a \textbf{pole} of order $n$ at $z = 0$.
				
				\item $f(z) = e^{\frac{1}{z}}$ has an \textbf{essential singularity} at $z = 0$.
				
				\item $f(z) = \frac{e^z - 1}{z}$ has a \textbf{removable singularity} at $z = 0$.
				\begin{proof}
					\begin{align}
						f(z) = \sum_{k = 1}^{\infty}{\frac{z^{k - 1}}{k!}}
					\end{align}
				\end{proof}
			\end{itemize}
		\end{example}
	\end{defn}

\newpage
\section{孤立奇点的等价刻画}
	\begin{center}
		下面主要研究\textbf{孤立奇点}附近的性态,并给出各类孤立奇点的等价刻画.
	\end{center}

\subsection{\textbf{Removable Singularity}}
	首先,介于\textbf{可去奇点}的良好性质,我们可在简单的操作后令函数\textbf{全纯},即:
	\begin{rmk}
		If $z_0$ is a removable singularity of $f : D_{r}^{*}(z_0) \rightarrow \C$, then we define $f(z_0) = a_0$ so that $f$ is holomorphic in $D_{r}(z_0)$, where
		\begin{align}
			f(z) = \sum_{n = 0}^{\infty}{a_n (z - z_0)^n} , \,\, z \in D_{r}^{*}(z_0)
		\end{align}
	\end{rmk}
	
	\vspace{2em}

	下面给出\textbf{可去奇点}的\textbf{等价刻画}.
	\begin{thm}\label{thm 8.3.1}
		If $z_0$ is a singularity of $f$, then 
		\begin{center}
			$z_0 \in \C$ is a removable singularity \hspace*{1em} iff \hspace*{1em} $f$ is bounded near $z_0$. \\
			(i.e. in a deleted neighbourhood of $z_0$)
		\end{center}
	
		\vspace{2em}
		\begin{proof}
			\begin{enumerate}
				\item[$``\Rightarrow"$]: We can define the value of $f$ at $z_0$ $\st$
				\begin{center}
					$f$ is holomorphic in $D_{r}(z_0)$ for some $r > 0$.
				\end{center}
				In particular, $f$ is continuous in $\overline{D_{\frac{r}{2}}(z_0)}$ and so $f$ is bounded near $z_0$.
				
				\vspace{2em}
				\item[$`` \Leftarrow "$]: Define
				\begin{align}
					g : D_{r}(z_0) &\longrightarrow \C \\
					z &\longmapsto g(z) = 
					\begin{cases}
						(z - z_0)^2 f(z) , z \neq z_0 \\
						0 , z = z_0
					\end{cases}
				\end{align}
				Then
				\begin{align}
					g^{'}(z) = 2(z - z_0)f(z) + (z - z_0)^2 f^{'}(z) , \,\, \forall z \in D_{r}^{*}(z_0)
				\end{align}
				Since $f$ is bounded near $z_0$,
				\begin{align}
					\lim_{z \to z_0}{\frac{g(z) - g(z_0)}{z - z_0}}
					= \lim_{z \to z_0}{(z - z_0)f(z)}
					= 0
				\end{align}
				Thus $g$ is holomorphic in $D_{r}(z_0)$ and so $g$ is analytic in $D_{r}(z_0)$,
				\begin{align}
					g(z) = \sum_{n = 0}^{\infty}{c_n (z - z_0)^n},  \,\, \forall z \in D_{r}(z_0)
				\end{align}
				Note that $c_0 = g(z_0) = 0$, $c_1 = g^{'}(z_0) = 0$. Thus
				\begin{align}
					g(z) = \sum_{n = 0}^{\infty}{a_n (z - z_0)^{n + 2}}, \,\, where \,\, a_n = c_{n + 2}, n = 0 , 1 , 2 , \cdots
				\end{align}
				Therefore
				\begin{align}
					f(z) = (z - z_0)^{-2} g(z) = \sum_{n = 0}^{\infty}{a_n (z - z_0)^n} , \,\, \forall z \in D_{r}^{*}(z_0)
				\end{align}
				and so $z_0$ is a removable singularity of $f$.
			\end{enumerate}
		\end{proof}
	\end{thm}

	\vspace{3em}
	从上述定理的证明过程中,可以直接得到下面的推论,也是对\textbf{可去奇点}的\textbf{等价刻画}.
	\begin{corollary}\label{cor 8.3.2}
		If $z_0$ is a singularity of $f$, then
		\begin{center}
			$z_0$ is a removable singularity \hspace*{1em} iff \hspace*{1em} $\underset{z \to z_0}{\lim}{(z - z_0)f(z)} = 0$.
		\end{center}
	\end{corollary}

\newpage
\subsection{\textbf{Pole}}
	作为\textbf{定理 \ref{thm 8.3.1}}的推论,下面我们给出\textbf{极点}的\textbf{等价刻画}.
	\begin{corollary}\label{cor 8.3.3}\footnote{对应课本P85 Cor 3.2}
		If $z_0$ is a singularity of $f$, then
		\begin{center}
			$z_0$ is a pole \hspace*{1em} iff \hspace*{1em} $\left| f(z) \right| \to \infty$ as $z \to z_0$.
		\end{center}
	
		\vspace{2em}
		\begin{proof}
			$`` \Rightarrow "$: $\frac{1}{f(z_0)} = 0 \,\, \Rightarrow \,\, \left| f(z) \right| \to \infty$ as $z \to z_0$. 
			
			\vspace{1em}
			$`` \Leftarrow "$: Suppose $\left| f(z) \right| \to \infty$ as $z \to z_0$, then $\exists r > 0$, $\st$
			\begin{center}
				$f(z) \neq 0 , \,\, \forall z \in D_{r}^{*}(z)$
			\end{center}
			and so $\frac{1}{f}$ is holomorphic on $D_{r}^{*}(z_0)$. Moreover, $\left| \frac{1}{f(z)} \right| \to 0$ as $z \to z_0$. \\
			By \textbf{Thm \ref{thm 8.3.1}}, $\frac{1}{f}$ is bounded near $z_0$,
			\begin{center}
				$\Rightarrow \,\, z_0$ is a removable singularity of $\frac{1}{f}$.
			\end{center}
			Since $\left| \frac{1}{f(z)} \right| \to 0$ as $z \to z_0$, we have
			\begin{align}
				\frac{1}{f(z)} = \sum_{n = 0}^{\infty}{a_n (z - z_0)^n} , \,\, \forall z \in D_{r}^{*}(z_0) , \,\, where \,\, a_0 = 0
			\end{align}
			Therefore, if we define $\frac{1}{f(z_0)} = 0$, then $\frac{1}{f}$ is holomorphic on $D_{r}(z_0)$. By \textbf{Def \ref{def 7.1.3}}, $z_0$ is a pole of $f$. 
		\end{proof}
	\end{corollary}

\vspace{2em}
\subsection{\textbf{Essential Singularity}}
	在排除了\textbf{定理 \ref{thm 8.3.1}}和\textbf{Cor \textbf{cor 8.3.3}}的情况后,下面我们给出\textbf{本性奇点}的\textbf{等价刻画}.
	\begin{corollary}\label{cor 8.3.4}
		If $z_0$ is a singularity of $f$, then
		\begin{center}
			$z_0$ is an essential singularity \hspace*{1em} iff \hspace*{1em} $\underset{z \to z_0}{\lim}{\left| f(z) \right|}$ does not exists. \\
			(Here we allow the limit to be $\infty$)
		\end{center}
	\end{corollary}
	
	\begin{example}\label{ex 8.3.1}
		Consider $f(z) = e^{\frac{1}{z}}$. Since 
		\begin{align}
			\lim_{\substack{z \in \R \\ z \to 0^{+}}}{e^{\frac{1}{z}}} = \infty , \,\, \lim_{\substack{z \in \R \\ z \to 0^{-}}}{e^{\frac{1}{z}}} = 0
		\end{align}
		Therefore $0$ is an essential singularity of $f$.
	\end{example}









\newpage
\section{课堂例题$2024-04-15$}
\begin{enumerate}
	\item \textbf{(课前 Question 1.)} \\
	Let $z_1 , z_2 \in \C$ with $Rez_1 \leq 0 , Rez_2 \leq 0$. Show
	\begin{center}
		$\left| e^{z_1} - e^{z_2} \right| \leq \left| z_1 - z_2 \right|$
	\end{center}

	\vspace{2em}
	
	\item \textbf{(课前 Question 2.)} \\
	If $f , g$ are entire functions that agree in infinite number of points, then $f = g$?
	
	\vspace{2em}
	\begin{solution}
		Eg: $f(z) = \sin{z} , g(z) = e^1 \sin{z}$.
	\end{solution}
	
	\vspace{2em}
	
	\item \textbf{(课前 Question 3.)} \\
	Is there a holomorphic function $f : \C \backslash \{ 0 \} \rightarrow \C$ with a simple pole at $z = 0$, $\st$
	\begin{align}
		\int_{C_{1}(0)}{f(z) dz} = 0 \,\, ?
	\end{align}

	\vspace{2em}
	\begin{solution}
		simple pole $\,\, \Rightarrow \,\,$ $Res_{0}f \neq 0$ $\,\, \Rightarrow \,\,$ $\int_{C_{1}(0)}{f(z) dz} = 2 \pi i \cdot Res_{0}f \neq 0$.
	\end{solution}

	\vspace{2em}
	
	\item 课本第三章练习$T13$.
\end{enumerate}







	%  ############################
	\ifx\allfiles\undefined
\end{document}
\fi