\ifx\allfiles\undefined
\input{../config/config}
\begin{document}
	% \title{{\Huge{\textbf{$Complex \,\, Analysis$\footnote{课堂教材:\textbf{《$Complex \,\, Analysis$》---  $Elias \,\, M. \,\, Stein$}}}}}}
\author{$-TW-$}
\date{\today}
\maketitle                   % 在单独的标题页上生成一个标题

\thispagestyle{empty}        % 前言页面不使用页码
\begin{center}
	\Huge\textbf{序}
\end{center}


\vspace*{3em}
\begin{center}
	\large{\textbf{天道几何,万品流形先自守;}}\\
	
	\large{\textbf{变分无限,孤心测度有同伦。}}
\end{center}

\vspace*{3em}
\begin{flushright}
	\begin{tabular}{c}
		\today \\ \small{\textbf{长夜伴浪破晓梦,梦晓破浪伴夜长}}
	\end{tabular}
\end{flushright}


\newpage                      % 新的一页
\pagestyle{plain}             % 设置页眉和页脚的排版方式(plain:页眉是空的,页脚只包含一个居中的页码)
\setcounter{page}{1}          % 重新定义页码从第一页开始
\pagenumbering{Roman}         % 使用大写的罗马数字作为页码
\tableofcontents              % 生成目录

\newpage                      % 以下是正文
\pagestyle{plain}
\setcounter{page}{1}          % 使用阿拉伯数字作为页码
\pagenumbering{arabic}
\setcounter{chapter}{-1}    % 设置 -1 可作为第零章绪论从第零章开始 
	\else
	\fi
	%  ############################ 正文部分

\chapter{$Week \,\, 13$}
\section{Automorphisms of $\mathbb{D} , \,\, \C , \,\, \mathbb{H}$}
	下面我们来研究复平面上三种特殊区域的\textbf{自同构群 (Automorphism)}. 
	
	\vspace*{1em}
	首先给出\textbf{自同构 (Automorphism)}的概念.
	\begin{defn}\label{def 13.1.1}
		A conformal map from an open set $\Omega$ to itself is called an \underline{\textcolor{blue}{\textbf{automorphism of $\Omega$}}}. The set of all automorphisms of $\Omega$ is denoted by \underline{\textcolor{blue}{\textbf{$Aut(\Omega)$}}}.
	\end{defn}
	
	\vspace*{6em}
	下面我们分别对开圆盘$\mathbb{D}$, 复平面$\C$, 及上半平面$\mathbb{H}$(下一周) 上的\textbf{自同构群 (Automorphisms)}的结构进行研究和讨论.
	
\newpage
\section{Automorphisms of $\mathbb{D}$}
\subsection{$The \,\, Blaschke \,\, Factors$}
	首先先来介绍一类特殊的,同时也是$\mathbb{D}$ 上最重要的一类\textbf{自同构}——\textbf{Blaschke Factors}:
	\begin{align}
		\varphi_{\alpha}(z) = \frac{\alpha - z}{1 - \overline{\alpha} z} \in Aut(\mathbb{D}), \,\, where \,\, \alpha \in \C \,\, with \,\, \left| \alpha \right| < 1
	\end{align}
	\begin{center}
		(可用\textbf{Maximum Modulus Principle (Thm \ref{thm 11.2.2})} 证明$\varphi_\alpha (\mathbb{D}) \subset \mathbb{D}$)
	\end{center}
	同时容易得到
	\begin{align}
		\varphi_\alpha \circ \varphi_\alpha (z) = z , \,\, \forall z \in \mathbb{D}
	\end{align}

\vspace*{6em}
\subsection{Automorphisms of $\mathbb{D}$}
	在介绍完\textbf{Blaschke Factors}后,下面我们给出$\mathbb{D}$ 上自同构的一般的刻画.
	\begin{thm}\label{thm 13.2.1}
		\textbf{Automorphisms of $\mathbb{D}$}. \\
		If $f \in Aut(\mathbb{D})$, then $\exists \vartheta \in \R , \,\, \alpha \in \mathbb{D}$, $\st$
		\begin{align}
			f(z) = e^{i\vartheta} \cdot \varphi_\alpha(z), \,\, where \,\, \varphi_\alpha(z) = \frac{\alpha - z}{1 - \overline{\alpha} z}
		\end{align}
	\end{thm}
	
	\vspace*{2em}
	\begin{corollary}\label{cor 13.2.2}
		The only automorphism of $\mathbb{D}$ that fixes the origin are the rotations.
	\end{corollary}

\newpage
\section{Automorphisms of $\C$}
	复平面$\C$ 上的自同构非常的简单,这里要用到书中\textbf{第三章练习15}的结论(习题证明可见我另一篇知乎文章).
	\begin{thm}\label{thm 13.3.1}
		\textbf{Automorphisms of $\C$}. \\
		If $f \in Aut(\Omega)$, then
		\begin{align}
			f(z) = az + b \,\, for \,\, some \,\, a , b \in \C , \,\, a \neq 0
		\end{align}
	\end{thm}

\newpage
\section{Exercise 2024-05-20}
	\begin{enumerate}
		\item P209-212 Examples. 
		
		\vspace*{2em}
		
		\item P248 T1-7
	\end{enumerate}





	%  ############################
	\ifx\allfiles\undefined
\end{document}
\fi