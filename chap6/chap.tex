\ifx\allfiles\undefined
\input{../config/config}
\begin{document}
	% \title{{\Huge{\textbf{$Complex \,\, Analysis$\footnote{课堂教材:\textbf{《$Complex \,\, Analysis$》---  $Elias \,\, M. \,\, Stein$}}}}}}
\author{$-TW-$}
\date{\today}
\maketitle                   % 在单独的标题页上生成一个标题

\thispagestyle{empty}        % 前言页面不使用页码
\begin{center}
	\Huge\textbf{序}
\end{center}


\vspace*{3em}
\begin{center}
	\large{\textbf{天道几何,万品流形先自守;}}\\
	
	\large{\textbf{变分无限,孤心测度有同伦。}}
\end{center}

\vspace*{3em}
\begin{flushright}
	\begin{tabular}{c}
		\today \\ \small{\textbf{长夜伴浪破晓梦,梦晓破浪伴夜长}}
	\end{tabular}
\end{flushright}


\newpage                      % 新的一页
\pagestyle{plain}             % 设置页眉和页脚的排版方式(plain:页眉是空的,页脚只包含一个居中的页码)
\setcounter{page}{1}          % 重新定义页码从第一页开始
\pagenumbering{Roman}         % 使用大写的罗马数字作为页码
\tableofcontents              % 生成目录

\newpage                      % 以下是正文
\pagestyle{plain}
\setcounter{page}{1}          % 使用阿拉伯数字作为页码
\pagenumbering{arabic}
\setcounter{chapter}{-1}    % 设置 -1 可作为第零章绪论从第零章开始 
	\else
	\fi
	%  ############################ 正文部分

\chapter{$Week \,\, 6$}
\section{课堂例题$2024-04-01$}
\begin{center}
	本节为习题课. (博士研究生助教代课)
\end{center}
\begin{enumerate}
	\item 课本第二章练习$T1 , T2 , T3 , T4$.
\end{enumerate}

\newpage

\section{函数项级数,全纯函数解析}
\paragraph{回顾}
在介绍复数域上函数项级数的性质之前,先回顾一下函数列的性质 (Thm \ref{thm 5.4.1}).
\begin{thm}\label{thm 6.2.1}
	\textbf{一致收敛 $\Rightarrow$ 积分与极限可交换次序}.\\
	Let $\Omega \subset \C$ be open, $\{ f_n : \Omega \longrightarrow \C \}_{n = 1}^{\infty}$ be a sequence of continuous functions that converges uniformly to $f$ \textbf{on every compact subset of $\Omega$}. Then
	\begin{enumerate}
		\item[(1)]$f$ is continuous.
		
		\vspace{1em}
		
		\item[(2)]If $\gamma \subset \Omega$ is a path with finite length, then
		\begin{align}
			\lim_{n \to \infty}{\int_{\gamma}{f_{n}(z) dz}} = \int_{\gamma}{f(z) dz}
		\end{align}
		
		\vspace{1em}
		
		\item[(3)]If $f_n$ is holomorphic for all $n$, then so is $f$. Moreover, \\
		$\{ f_{n}^{'} \}_{n = 1}^{\infty}$ converges uniformly to $f^{'}$ \textbf{on every compact subset of $\Omega$}.
	\end{enumerate}
	
	\vspace{1em}
	\begin{rmk}
		\begin{itemize}
			\item 事实上,结论 (3)可做推广,即当函数列$\{ f_n \}_{n = 1}^{\infty}$ 满足上述条件时,有:
			\begin{center}
				$\{ f_{n}^{(k)} \}_{n = 1}^{\infty}$ converges uniformly to $f^{(k)}$ \textbf{on every compact subset of $\Omega$}.
			\end{center}
			
			\vspace{2em}
			
			\item 注意实变函数列与复变函数列的\textbf{可微性}的区别,即实变函数列不满足定理中的 (3). 下面给出结论 (3)在实变函数列下的反例.
			
			\vspace{1em}
			
			\begin{example}\label{ex 6.2.1}
				Let $f_{n}(x) = \sqrt{x^2 + \frac{1}{n}}$, $x \in [-1 , 1]$. Then $\{ f_n \}_{n = 1}^{\infty}$ converges uniformly to $f(x) = \left| x \right|$. \\
				Though $f_n$, $n = 1 , 2 , \cdots$ are differentiable over $[-1 , 1]$, the limit function $f(x) = \left| x \right|$ is \textbf{not differentiable} at $x = 0$.
			\end{example}
		\end{itemize}
	\end{rmk}
	
	\vspace{2em}
	\begin{proof}
		详见定理 \ref{thm 5.4.1}证明.
	\end{proof}
\end{thm}

\newpage

\paragraph{函数项级数}
下面给出函数项级数收敛的定义.
\begin{defn}\label{def 6.2.1}
	Let $\Omega \subset \C$ be open, $\{ f_n : \Omega \longrightarrow \C \}_{n = 1}^{\infty}$ be a sequence of functions. \\
	We say $\overset{\infty}{\underset{n = 1}{\sum}}{f_n}$ \underline{\textcolor{blue}{\textbf{converges}}} if $\{ S_{N} = \overset{N}{\underset{n = 1}{\sum}}{f_n} \}_{N = 1}^{\infty}$ converges.  \\
	We say $\overset{\infty}{\underset{n = 1}{\sum}}{f_n}$ \underline{\textcolor{blue}{\textbf{converges uniformly}}} if $\{ S_{N} = \overset{N}{\underset{n = 1}{\sum}}{f_n} \}_{N = 1}^{\infty}$ converges uniformly).
\end{defn}

\vspace{2em}
下面给出判断函数项级数一致收敛性的经典方法 (\textbf{Weierstrass M-test}).

\begin{proposition}\label{prop 6.2.1}
	\textbf{Weierstrass M-test}. \\
	If $\left| f_{n}(z) \right| \leq M_n$, $n = 1 , 2 , \cdots , \forall z \in \Omega$, and $\overset{\infty}{\underset{n = 1}{\sum}}{M_n} < \infty$, then $\overset{\infty}{\underset{n = 1}{\sum}}{f_n}$ converges uniformly.
	
	\vspace{2em}
	\begin{proof}
		Let $S_N = \overset{N}{\underset{n = 1}{\sum}}{f_n}$, $\forall N \in \N$. Since $\overset{\infty}{\underset{n = 1}{\sum}}{M_n} < \infty$, then\\
		$\forall \epsilon > 0 , \exists N \in \N , \st n > m > N$
		\begin{align}
			\left| S_n - S_m \right| = \left| f_{n}(z) + \cdots + f_{m + 1}(z) \right| \leq \sum_{j = m + 1}^{\infty}{M_j} \leq \epsilon , \,\, \forall z \in \Omega
		\end{align}
		Therefore, $\{ S_n \}_{n = 1}^{\infty}$ converges uniformly. $\,\, \Rightarrow \,\,$ $\{ f_n \}_{n = 1}^{\infty}$ converges uniformly.
	\end{proof}
\end{proposition}

\vspace{2em}
\paragraph{解析与全纯等价}
下面证明\textbf{全纯}函数均\textbf{解析 (可展成幂级数)}. 
\begin{center}
	\textbf{(该定理与定理 \ref{thm 3.1.2}共同说明了,解析 $\Leftrightarrow$ 全纯)}
\end{center}
\begin{thm}\label{thm 6.2.2}
	Suppose $f$ is holomorphic on an open set $\Omega \subset \C$. Then $\forall z_0 \in \Omega$ with $D_{r}(z_0) \subset \Omega$ for some $r > 0$, $f$ has a power series expansion at $z_0$.
	\begin{align}
		f(z) = \sum_{n = 0}^{\infty}{a_n (z - z_0)^n} , \,\, \forall z \in D_{r}(z_0)
	\end{align}
	$where \,\, a_n = \frac{f^{(n)}(z_0)}{n!} , \,\, \forall n \geq 0$.\\
	
	\begin{proof}
		Fix $z \in D_{r}(z_0)$. By \textbf{CIF (Thm \ref{thm 5.2.1})},
		\begin{align}
			f(z) = \frac{1}{2\pi i}\int_{C_{r}(z_0)}{\frac{f(\zeta)}{\zeta - z} d\zeta}
		\end{align}
		Then
		\begin{align}
			\frac{1}{\zeta - z} 
			= \frac{1}{\zeta - z_0 - (z - z_0)} 
			= \frac{1}{\zeta - z_0} \cdot \frac{1}{1 - \frac{z - z_0}{\zeta - z_0}} 
		\end{align}
		Since $z \in D_{r}(z_0)$ is fixed, and $\forall \zeta \in C_{r}(z_0)$, there exists $0 < r < 1$, $\st$
		\begin{align}
			\left| \frac{z - z_0}{\zeta - z_0} \right| < r
		\end{align}
		Therefore,
		\begin{align}
			\sum_{n = 0}^{N}{\left( \frac{z - z_0}{\zeta - z_0} \right)^n} \,\, \Rightarrow \,\, \frac{1}{1 - \frac{z - z_0}{\zeta - z_0}} , \,\, N \to \infty , \,\, \forall \zeta \in C_{r}(z_0)
		\end{align}
		converges uniformly \textbf{w.r.t. (with respect to , 关于)} $\zeta \in C_{r}(z_0)$. i.e.
		\begin{align}
			\sum_{n = 0}^{\infty}{\left( \frac{z - z_0}{\zeta - z_0} \right)^n} = \frac{1}{1 - \frac{z - z_0}{\zeta - z_0}} , \,\, \forall \zeta \in C_{r}(z_0)
		\end{align}
		Let
		\begin{align}
			g_{N}(\zeta)
			&= \sum_{n = 0}^{N}{\left( f(\zeta) \cdot \frac{1}{\zeta - z_0} \cdot \left( \frac{z - z_0}{\zeta - z_0} \right)^n \right)} \\
			&= f(\zeta) \cdot \frac{1}{\zeta - z_0} \sum_{n = 0}^{N}{\left( \frac{z - z_0}{\zeta - z_0} \right)^n} , \,\, \zeta \in C_{r}(z_0)
		\end{align}
		Then we have
		\begin{align}
			g_{N} \,\, &\Rightarrow \,\, f(\zeta) \cdot \frac{1}{\zeta - z_0} \cdot \frac{1}{1 - \frac{z - z_0}{\zeta - z_0}} \\
			&= \frac{f(\zeta)}{\zeta - z} , \,\, N \to \infty , \,\, \zeta \in C_{r}(z_0)
		\end{align}
		converges uniformly w.r.t. $\zeta \in C_{r}(z_0)$. Therefore, by Thm \ref{thm 6.2.1},
		\begin{align}
			f(z) 
			= \frac{1}{2\pi i}\int_{C_{r}(z_0)}{\frac{f(\zeta)}{\zeta - z} d\zeta}
			&= \frac{1}{2\pi i} \int_{C_{r}(z_0)}{\lim_{N \to \infty}{g_{N}(\zeta) d\zeta}} \\
			&= \lim_{N \to \infty}{\frac{1}{2 \pi i} \int_{C_{r}(z_0)}{g_{N}(\zeta) d\zeta}} \\
			&= \lim_{N \to \infty}{\frac{1}{2 \pi i} \int_{C_{r}(z_0)}{ \left( f(\zeta) \cdot \frac{1}{\zeta - z_0} \sum_{n = 0}^{N}{\left( \frac{z - z_0}{\zeta - z_0} \right)^n } \right) d\zeta } } \\
			&= \lim_{N \to \infty}{\sum_{n = 0}^{N} \frac{1}{2 \pi i} \int_{C_{r}(z_0)}{ \left( f(\zeta) \cdot \frac{1}{\zeta - z_0} \left( \frac{z - z_0}{\zeta - z_0} \right)^n \right) d\zeta } } \\
			&= \sum_{n = 0}^{\infty} \left( \frac{1}{2 \pi i} \int_{C_{r}(z_0)}{ \frac{f(\zeta)}{(\zeta - z_0)^{n + 1}} d\zeta } \right) (z - z_0)^n \\
			&= \sum_{n = 0}^{\infty}{a_n (z - z_0)^n}
		\end{align}
		By \textbf{CIF (Thm \ref{thm 5.2.2})}, we have
		\begin{align}
			a_n = \frac{1}{2 \pi i} \int_{C_{r}(z_0)}{ \frac{f(\zeta)}{(\zeta - z_0)^{n + 1}} d\zeta } = \frac{f^{(n)}(z_0)}{n!} , \,\, \forall n \geq 0
		\end{align}
	\end{proof}
	
	\begin{rmk}
		事实上,该定理也提供了\textbf{CIF 高阶形式 (Thm \ref{thm 5.2.2})} 的另一种证明 (比较系数可得).
	\end{rmk}
\end{thm}

\newpage

\section{解析延拓}
\paragraph{定义}
下面先给出\textbf{解析延拓 (Analytic continuation)}的定义.
\begin{defn}\label{def 6.3.1}
	Suppose $f$ and $F$ are holomorphic in nonempty regions $\Omega$ and $\hat{\Omega}$ respectively with $\Omega \subset \hat{\Omega}$. If $f(z) = F(z)$ in $\Omega$, then we say $F$ is an \underline{\textcolor{blue}{\textbf{analytic continuation}}} of $f$ in $\hat{\Omega}$.
\end{defn}

\vspace{2em}
\paragraph{解析延拓的唯一性}
在说明这之前,先给出一个有关全纯函数的非常重要的结论.

\begin{thm}\label{thm 6.3.1}
	Suppose $f$ is holomorphic in a region $\Omega$ that vanishes on a sequence of distinct points with a limit in $\Omega$. Then $f(z) \equiv 0$ for all $z \in \Omega$.
	
	\vspace{1em}
	\begin{rmk}
		该定理说明了,不恒为零的全纯函数的\textbf{零点均为孤立点} (不为聚点).
	\end{rmk}
	
	\vspace{1em}
	\begin{proof}
		反证法. Suppose $\{ w_k \}_{k = 1}^{\infty} \subset \Omega$ with $\underset{k \to \infty}{\lim}{w_k} = z_0 \in \Omega$ and $f(w_k) = 0$, $k = 1 , 2 , \cdots$. \\
		Since $f$ is holomorphic in $\Omega$, $\forall z \in D_{r}(z_0)$
		\begin{align}
			f(z) = \sum_{n = 0}^{\infty}{a_n (z - z_0)^n} , \,\, for \,\, some \,\, r > 0
		\end{align}
		下面分为两步证明.
		\begin{itemize}
			\item We first show $f(z) \equiv 0$ on $D_{r}(z_0)$. \\
			Assume $f(z) \neq 0$ for $z \in D_{r}(z_0)$, then $\exists$ smalledt integer $m$ $\st$ $a_m \neq 0$. Now
			\begin{align}
				f(z) = a_m (z - z_0)^m (1 + g(z)) , \,\, where \,\, g(z) = \sum_{n = m + 1}^{\infty}{a_n (z - z_0)^{n - m}}
			\end{align}
			Since $g(z) \to 0$ as $z \to z_0$, $\forall \epsilon < 1$, there exists $\delta > 0$, $\st$
			\begin{align}
				\left| g(z) \right| \leq \epsilon < 1 , \,\, \forall z \in D_{\delta}^{*}(z_0)
			\end{align}
			Since $w_k \to z_0 \in \Omega$, $\exists k_0 \in \N$, $\st$ $w_{k_0} \in D_{\delta}^{*}(z_0)$. Then
			\begin{align}
				f(w_{k_0}) = a_m (z - z_0)^m (1 + g(w_{k_0})) \neq 0
			\end{align}
			which is a contradiction with that $f(w_{k_0}) = 0$.
			
			\item Then we shall show $f \equiv 0$ on $\Omega$. \\
			Let $U$ be the interior of $\{ z \in \Omega \mid f(z) = 0 \}$. Since $f(z) \equiv 0 , \forall z \in D_{r}(z_0)$, $U \neq \varnothing$ and $U$ is open. \\
			Moreover, 下面我们证明 $U$ is closed.
			\begin{itemize}
				\item For all $\{ z_k \}_{k = 1}^{\infty} \subset U$ with $z_k \to p \in \Omega$. 与第一步证明相同,可以得到\\
				$\exists r_p > 0$, $\st$ $f(z) \equiv 0 , \forall z \in D_{r_p}(p)$. 于是$p \in U$. 即$U$ 包含了自身序列的所有极限点. \\
				which means that $U$ is closed.
			\end{itemize}
			Therefore, $U \subset \Omega$ is both open and closed. Since $U \neq \varnothing$ and $\Omega$ is connected, then $U = \Omega$, \\
			which means $f \equiv 0$ on $\Omega$.
		\end{itemize}
	\end{proof}
\end{thm}

\vspace{2em}
通过上述定理可得到,全纯函数的取值只由区域上\textbf{可数个点}决定.
\begin{corollary}\label{cor 6.3.2}
	Suppose $f , g$ are holomorphic in a region $\hat{\Omega}$ and $f(z) = g(z)$ for all $z \in \Omega$, where $\Omega$ is an open subset of $\hat{\Omega}$. Then $f(z) = g(z)$ for $z \in \hat{\Omega}$.
\end{corollary}

\vspace{2em}
By Cor \ref{cor 6.3.2},我们得到解析延拓若存在,则\textbf{必唯一}.
\begin{corollary}\label{cor 3.3.3}
	Suppose $F$ and $G$ are both analytic continuation of $f$ into $\hat{\Omega}$. Then
	\begin{center}
		$F = G$ in $\hat{\Omega}$
	\end{center}
\end{corollary}

\newpage

\section{对称原理}
\paragraph{引入}
在实分析中,我们曾探讨过有关\textbf{连续函数的延拓 (Tietze 延拓定理}). 

但在复分析中,对于\textbf{全纯函数的延拓}似乎不再那么容易与显然,因为全纯函数不仅要求在复平面上光滑,而且还具有一些 additional characteristically rigid properties. 
\begin{center}
	(书P67 Problem 1. 给出了无法 (解析)延拓至$\overline{\mathbb{D}}$ 的定义在$\mathbb{D}$ 上的全纯函数.)
\end{center}
而本节将给出一种十分有用的情况下全纯函数的延拓,即在\textbf{关于实轴对称}的区域上的延拓.

\vspace{2em}
\paragraph{对称原理}
为了讨论的方便,接下来的命题将默认以下几个记号:
\begin{itemize}
	\item Let $\Omega$ be an open subset of $\C$ that is \textbf{symmetric} w.r.t. the real axis.
	
	\item Let $\Omega^{+}$ denote the part of $\Omega$ that lies in the upper half-plane and $\Omega^{-}$ the part that lies in the lower half-plane.
\end{itemize}

\vspace{2em}
下面给出对称原理.
\begin{thm}\label{thm 6.4.1}
	\textbf{Symmetric principle}. \\
	If $f^{+}$ and $f^{-}$ are holomorphic in $\Omega^{+}$ and $\Omega^{-}$ respectively that extends continuously to $I$ ($I = \Omega \cap \R$) and $f^{+}(x) = f^{-}(x)$ for all $x \in I$. Then
	\begin{align}
		f(z) = 
		\begin{cases}
			f^{+}(z) , z \in \Omega^{+} \\
			f^{+}(z) = f^{-}(z) , z \in I \\
			f^{-}(z) , z \in \Omega^{-}
		\end{cases}
	\end{align}
	is holomorphic in $\Omega$.
	
	\vspace{2em}
	\begin{proof}
		详见书P58 Thm 5.5 证明.
	\end{proof}
\end{thm}

\newpage
\paragraph{\textbf{Schwartz} 反射原理}
有了上述对称原理的铺垫后,下面给出全纯函数在\textbf{关于实轴对称}区域上的延拓定理. (\textbf{Schwartz 反射原理})
\begin{thm}\label{thm 6.4.2}
	\textbf{Schwartz reflection principle}. \\
	Suppose $f$ is holomorphic in $\Omega^{+}$ that extends continuously to $I$ and $f$ is real-valued on $I$. Then $\exists F$ holomorphic in all $\Omega$, $\st$ $F = f$ in $\Omega^{+}$.
	
	\vspace{2em}
	\begin{proof}
		Define $f^{-}(z) = \overline{f(\overline{z})}$ for $z \in \Omega^{-}$. Fix $z_0 \in \Omega^{-} , \exists r > 0 , \st D_{r}(z_0) \subset \Omega^{-}$.\\
		Then $\overline{z_0} \in \Omega^{+}$ and $D_{r}(\overline{z_0}) \subset \Omega^{+}$. $\forall x \in D_{r}(z_0)$, $\overline{x} \in D_{r}(\overline{z_0})$\\
		Since $f$ is holomorphic in $\Omega^{+}$, by Thm \ref{thm 6.2.2} (全纯函数均解析)
		\begin{align}
			f(\overline{z}) = \sum_{n = 0}^{\infty}{a_n (\overline{z} - \overline{z_0})^n} , \,\, \forall \overline{z} \in D_{r}(\overline{z_0})
		\end{align}
		Then we have
		\begin{align}
			f^{-}(z) = \sum_{n = 0}^{\infty}{\overline{a_n} (z - z_0)^n} = \sum_{n = 0}^{\infty}{a_n (z - z_0)^n} , \,\, \forall z \in D_{r}(z_0)
		\end{align}
		Therefore, $f^{-}$ is analytic in $\omega^{-}$. By Thm \ref{thm 3.1.2}, $f^{-}$ is holomorphic in $\Omega^{-}$. (解析函数均全纯)\\
		Therefore, by \textbf{Symmetric principle (对称原理)}, 
		\begin{align}
			F(z) = 
			\begin{cases}
				f^{+}(z) , z \in \Omega^{+} \\
				f^{+}(z) = f^{-}(z) , z \in I \\
				f^{-}(z) , z \in \Omega^{-}
			\end{cases}
		\end{align}
		is holomorphic in $\Omega$.
	\end{proof}
\end{thm}

\newpage

\section{课堂例题$2024-04-07$}
\begin{enumerate}
	\item \textbf{(课本P66 Ex10.)}\\
	Weierstrass's theorem states that a continuous function on $[0 , 1]$ can be uniformly approxiamted by polynomials. \textbf{Can every continuous function on the closed unit disc be approxiated uniformly by polynomials in the variable $z$?}
	
	\vspace{2em}
	
	\begin{solution}
		Absolutely No. Take $f(z) = \left| z \right|$ continuous on $\mathbb{D}$ into consideration. \\
		If there exists polynomials $\{ f_n \}_{n = 1}^{\infty}$, $\st$ $f_n \Rightarrow f$, then by Thm \ref{thm 6.2.1}, \\
		$f$ is holomorphic on $\mathbb{D}$, which is a contradiction with that $f$ is not differentiable at $x = 0$.
	\end{solution}
	
	\vspace{2em}
	
	\item Suppose $f$ is entire and real-valued on the real-axis. If $f(1 + i) = 2 + i$, then what is $f(1 - i)$?
	
	\vspace{2em}
	
	\begin{solution}
		Ans : $2 - i$. (by \textbf{Schwartz reflection principle})
	\end{solution}
	
	\vspace{2em}
	
	\item 课本第二章练习$T7-10 , T15$.
\end{enumerate}






	%  ############################
	\ifx\allfiles\undefined
\end{document}
\fi