\ifx\allfiles\undefined
\documentclass[12pt, a4paper,oneside, UTF8]{ctexbook}
\usepackage[dvipsnames]{xcolor}
\usepackage{amsmath}   % 数学公式
\usepackage{amsthm}    % 定理环境
\usepackage{amssymb}   % 更多公式符号
\usepackage{graphicx}  % 插图
%\usepackage{mathrsfs}  % 数学字体
%\usepackage{newtxtext,newtxmath}
%\usepackage{arev}
\usepackage{kmath,kerkis}
\usepackage{newtxtext}
\usepackage{bbm}
\usepackage{enumitem}  % 列表
\usepackage{geometry}  % 页面调整
%\usepackage{unicode-math}
\usepackage[colorlinks,linkcolor=black]{hyperref}


\usepackage{ulem}	   % 用于更多的下划线格式,
					   % \uline{}下划线,\uuline{}双下划线,\uwave{}下划波浪线,\sout{}中间删除线,\xout{}斜删除线
					   % \dashuline{}下划虚线,\dotuline{}文字底部加点


\graphicspath{ {flg/},{../flg/}, {config/}, {../config/} }  % 配置图形文件检索目录
\linespread{1.5} % 行高

% 页码设置
\geometry{top=25.4mm,bottom=25.4mm,left=20mm,right=20mm,headheight=2.17cm,headsep=4mm,footskip=12mm}

% 设置列表环境的上下间距
\setenumerate[1]{itemsep=5pt,partopsep=0pt,parsep=\parskip,topsep=5pt}
\setitemize[1]{itemsep=5pt,partopsep=0pt,parsep=\parskip,topsep=5pt}
\setdescription{itemsep=5pt,partopsep=0pt,parsep=\parskip,topsep=5pt}

% 定理环境
% ########## 定理环境 start ####################################
\theoremstyle{definition}
\newtheorem{defn}{\indent 定义}[section]

\newtheorem{lemma}{\indent 引理}[section]    % 引理 定理 推论 准则 共用一个编号计数
\newtheorem{thm}[lemma]{\indent 定理}
\newtheorem{corollary}[lemma]{\indent 推论}
\newtheorem{criterion}[lemma]{\indent 准则}

\newtheorem{proposition}{\indent 命题}[section]
\newtheorem{example}{\indent \color{SeaGreen}{例}}[section] % 绿色文字的 例 ,不需要就去除\color{SeaGreen}{}
\newtheorem*{rmk}{\indent \color{red}{注}}

% 两种方式定义中文的 证明 和 解 的环境:
% 缺点:\qedhere 命令将会失效【技术有限,暂时无法解决】
\renewenvironment{proof}{\par\textbf{证明.}\;}{\qed\par}
\newenvironment{solution}{\par{\textbf{解.}}\;}{\qed\par}

% 缺点:\bf 是过时命令,可以用 textb f等替代,但编译会有关于字体的警告,不过不影响使用【技术有限,暂时无法解决】
%\renewcommand{\proofname}{\indent\bf 证明}
%\newenvironment{solution}{\begin{proof}[\indent\bf 解]}{\end{proof}}
% ######### 定理环境 end  #####################################

% ↓↓↓↓↓↓↓↓↓↓↓↓↓↓↓↓↓ 以下是自定义的命令  ↓↓↓↓↓↓↓↓↓↓↓↓↓↓↓↓

% 用于调整表格的高度  使用 \hline\xrowht{25pt}
\newcommand{\xrowht}[2][0]{\addstackgap[.5\dimexpr#2\relax]{\vphantom{#1}}}

% 表格环境内长内容换行
\newcommand{\tabincell}[2]{\begin{tabular}{@{}#1@{}}#2\end{tabular}}

% 使用\linespread{1.5} 之后 cases 环境的行高也会改变,重新定义一个 ca 环境可以自动控制 cases 环境行高
\newenvironment{ca}[1][1]{\linespread{#1} \selectfont \begin{cases}}{\end{cases}}
% 和上面一样
\newenvironment{vx}[1][1]{\linespread{#1} \selectfont \begin{vmatrix}}{\end{vmatrix}}

\def\d{\textup{d}} % 直立体 d 用于微分符号 dx
\def\R{\mathbb{R}} % 实数域
\def\N{\mathbb{N}} % 自然数域
\def\C{\mathbb{C}} % 复数域
\def\Z{\mathbb{Z}} % 整数环
\def\Q{\mathbb{Q}} % 有理数域
\newcommand{\bs}[1]{\boldsymbol{#1}}    % 加粗,常用于向量
\newcommand{\ora}[1]{\overrightarrow{#1}} % 向量

% 数学 平行 符号
\newcommand{\pll}{\kern 0.56em/\kern -0.8em /\kern 0.56em}

% 用于空行\myspace{1} 表示空一行 填 2 表示空两行  
\newcommand{\myspace}[1]{\par\vspace{#1\baselineskip}}

%s.t. 用\st就能打出s.t.
\DeclareMathOperator{\st}{s.t.}

%罗马数字 \rmnum{}是小写罗马数字, \Rmnum{}是大写罗马数字
\makeatletter
\newcommand{\rmnum}[1]{\romannumeral #1}
\newcommand{\Rmnum}[1]{\expandafter@slowromancap\romannumeral #1@}
\makeatother
\begin{document}
	% \title{{\Huge{\textbf{$Complex \,\, Analysis$\footnote{课堂教材:\textbf{《$Complex \,\, Analysis$》---  $Elias \,\, M. \,\, Stein$}}}}}}
\author{$-TW-$}
\date{\today}
\maketitle                   % 在单独的标题页上生成一个标题

\thispagestyle{empty}        % 前言页面不使用页码
\begin{center}
	\Huge\textbf{序}
\end{center}


\vspace*{3em}
\begin{center}
	\large{\textbf{天道几何,万品流形先自守;}}\\
	
	\large{\textbf{变分无限,孤心测度有同伦。}}
\end{center}

\vspace*{3em}
\begin{flushright}
	\begin{tabular}{c}
		\today \\ \small{\textbf{长夜伴浪破晓梦,梦晓破浪伴夜长}}
	\end{tabular}
\end{flushright}


\newpage                      % 新的一页
\pagestyle{plain}             % 设置页眉和页脚的排版方式(plain:页眉是空的,页脚只包含一个居中的页码)
\setcounter{page}{1}          % 重新定义页码从第一页开始
\pagenumbering{Roman}         % 使用大写的罗马数字作为页码
\tableofcontents              % 生成目录

\newpage                      % 以下是正文
\pagestyle{plain}
\setcounter{page}{1}          % 使用阿拉伯数字作为页码
\pagenumbering{arabic}
\setcounter{chapter}{-1}    % 设置 -1 可作为第零章绪论从第零章开始 
	\else
	\fi
	%  ############################ 正文部分

\chapter{$Week \,\, 14$}
\section{Automorphisms of $\mathbb{H}$}
\begin{center}
	最后一周的内容让我们来研究上半平面$\mathbb{H}$ 上自同构的结构.
\end{center}

\subsection{M\"{o}bius Transforms / Fractional Linear Transforms (分式线性变换)}
	先来给出\textbf{M\"{o}bius Transform} 的概念.
	\begin{defn}\label{def 14.1.1}
		Given $a , b , c , d \in \C$, $\st ad - bc \neq 0$, then
		\begin{align}
			z \mapsto \frac{az + b}{cz + d} \,\, \text{is called a \underline{\textcolor{blue}{\textbf{M\"{o}bius Transform}}}.}
		\end{align}
		
		\vspace*{2em}
		
		\begin{rmk}
			M\"{o}bius Transform is a composition of 3 elementary ones:
			\begin{align}
				\begin{cases}
					1.\text{\textcolor{red}{scaling}:} \sigma_{p}(z) = pz , \,\, p \in \C , p \neq 0 \\
					2.\text{\textcolor{red}{translation}:} \tau_{q}(z) = z + q , \,\, q \in \C \\
					3.\text{\textcolor{red}{inversion}:} I(z) = \frac{1}{z} (\text{反演})
				\end{cases}
			\end{align}
		\end{rmk}
	\end{defn}
	
	\vspace*{4em}
	下面给出一些记号:
	\begin{itemize}
		\item Let $A = \begin{pmatrix}
			a \,\, &b \\
			c \,\, &d
		\end{pmatrix}$ with $det(A) \neq 0$.
		\begin{align}
			M_{A}(z) = \frac{az + b}{cz + d}
		\end{align}
		
		\vspace*{2em}
		
		\item (\textbf{实2阶特殊线性群})
		\begin{align}
			SL_2(\R) = \left\{ A = \begin{pmatrix}
				a \,\, &b \\
				c \,\, &d
			\end{pmatrix} \,\, \Big| \,\, a , b , c , d \in \R , \,\, det(A) = 1 \right\}
		\end{align}
	\end{itemize}
	
	\vspace*{6em}
	关于\textbf{M\"{o}bius Transform},不难证明有以下事实:
	\begin{enumerate}
		\item \textbf{Fact 1}: Let $A_1 , A_2$ be 2 invertible metrics, then
		\begin{align}
			M_{A_1 A_2}(z) = M_{A_1} \circ M_{A_2}(z)
		\end{align}
		
		\vspace*{2em}
		
		\item \textbf{Fact 2}: M\"{o}bius Transform preserves lines $\&$ circles.
		\begin{rmk}
			此处指的是
			\begin{center}
				直线$\longrightarrow$ 直线 / 圆 \\
				$\&$ \\
				圆$\longrightarrow$ 直线 / 圆
			\end{center}
		\end{rmk}
		
		\vspace*{2em}
		
		\item \textbf{Fact 3}: Any circle can be preserved by the equation
		\begin{align}
			b \left| z \right|^2 - 2 Re(\overline{a}z) + c = 0 , \,\, where \,\, a \in \C , b , c \in \R , bc < \left| a \right|^2
		\end{align}
	\end{enumerate}

\newpage
\subsection{Automorphisms of $\mathbb{H}$}
	有了以上概念的铺垫,上半平面$\mathbb{H}$ 上的自同构的描述就显得十分轻松.
	\begin{thm}\label{thm 14.1.1}
		\textbf{Automorphisms of $\mathbb{H}$}. \\
		If $f \in Aut(\mathbb{H})$, then $\exists A \in SL_2(\R)$, $\st$
		\begin{align}
			f(z) = M_{A}(z)
		\end{align}
	\end{thm}




	%  ############################
	\ifx\allfiles\undefined
\end{document}
\fi