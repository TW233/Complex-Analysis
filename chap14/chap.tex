\ifx\allfiles\undefined
\input{../config/config}
\begin{document}
	% \title{{\Huge{\textbf{$Complex \,\, Analysis$\footnote{课堂教材:\textbf{《$Complex \,\, Analysis$》---  $Elias \,\, M. \,\, Stein$}}}}}}
\author{$-TW-$}
\date{\today}
\maketitle                   % 在单独的标题页上生成一个标题

\thispagestyle{empty}        % 前言页面不使用页码
\begin{center}
	\Huge\textbf{序}
\end{center}


\vspace*{3em}
\begin{center}
	\large{\textbf{天道几何,万品流形先自守;}}\\
	
	\large{\textbf{变分无限,孤心测度有同伦。}}
\end{center}

\vspace*{3em}
\begin{flushright}
	\begin{tabular}{c}
		\today \\ \small{\textbf{长夜伴浪破晓梦,梦晓破浪伴夜长}}
	\end{tabular}
\end{flushright}


\newpage                      % 新的一页
\pagestyle{plain}             % 设置页眉和页脚的排版方式(plain:页眉是空的,页脚只包含一个居中的页码)
\setcounter{page}{1}          % 重新定义页码从第一页开始
\pagenumbering{Roman}         % 使用大写的罗马数字作为页码
\tableofcontents              % 生成目录

\newpage                      % 以下是正文
\pagestyle{plain}
\setcounter{page}{1}          % 使用阿拉伯数字作为页码
\pagenumbering{arabic}
\setcounter{chapter}{-1}    % 设置 -1 可作为第零章绪论从第零章开始 
	\else
	\fi
	%  ############################ 正文部分

\chapter{$Week \,\, 14$}
\section{Automorphisms of $\mathbb{H}$}
\begin{center}
	最后一周的内容让我们来研究上半平面$\mathbb{H}$ 上自同构的结构.
\end{center}

\subsection{M\"{o}bius Transforms / Fractional Linear Transforms (分式线性变换)}
	先来给出\textbf{M\"{o}bius Transform} 的概念.
	\begin{defn}\label{def 14.1.1}
		Given $a , b , c , d \in \C$, $\st ad - bc \neq 0$, then
		\begin{align}
			z \mapsto \frac{az + b}{cz + d} \,\, \text{is called a \underline{\textcolor{blue}{\textbf{M\"{o}bius Transform}}}.}
		\end{align}
		
		\vspace*{2em}
		
		\begin{rmk}
			M\"{o}bius Transform is a composition of 3 elementary ones:
			\begin{align}
				\begin{cases}
					1.\text{\textcolor{red}{scaling}:} \sigma_{p}(z) = pz , \,\, p \in \C , p \neq 0 \\
					2.\text{\textcolor{red}{translation}:} \tau_{q}(z) = z + q , \,\, q \in \C \\
					3.\text{\textcolor{red}{inversion}:} I(z) = \frac{1}{z} (\text{反演})
				\end{cases}
			\end{align}
		\end{rmk}
	\end{defn}
	
	\vspace*{4em}
	下面给出一些记号:
	\begin{itemize}
		\item Let $A = \begin{pmatrix}
			a \,\, &b \\
			c \,\, &d
		\end{pmatrix}$ with $det(A) \neq 0$.
		\begin{align}
			M_{A}(z) = \frac{az + b}{cz + d}
		\end{align}
		
		\vspace*{2em}
		
		\item (\textbf{实2阶特殊线性群})
		\begin{align}
			SL_2(\R) = \left\{ A = \begin{pmatrix}
				a \,\, &b \\
				c \,\, &d
			\end{pmatrix} \,\, \Big| \,\, a , b , c , d \in \R , \,\, det(A) = 1 \right\}
		\end{align}
	\end{itemize}
	
	\vspace*{6em}
	关于\textbf{M\"{o}bius Transform},不难证明有以下事实:
	\begin{enumerate}
		\item \textbf{Fact 1}: Let $A_1 , A_2$ be 2 invertible metrics, then
		\begin{align}
			M_{A_1 A_2}(z) = M_{A_1} \circ M_{A_2}(z)
		\end{align}
		
		\vspace*{2em}
		
		\item \textbf{Fact 2}: M\"{o}bius Transform preserves lines $\&$ circles.
		\begin{rmk}
			此处指的是
			\begin{center}
				直线$\longrightarrow$ 直线 / 圆 \\
				$\&$ \\
				圆$\longrightarrow$ 直线 / 圆
			\end{center}
		\end{rmk}
		
		\vspace*{2em}
		
		\item \textbf{Fact 3}: Any circle can be preserved by the equation
		\begin{align}
			b \left| z \right|^2 - 2 Re(\overline{a}z) + c = 0 , \,\, where \,\, a \in \C , b , c \in \R , bc < \left| a \right|^2
		\end{align}
	\end{enumerate}

\newpage
\subsection{Automorphisms of $\mathbb{H}$}
	有了以上概念的铺垫,上半平面$\mathbb{H}$ 上的自同构的描述就显得十分轻松.
	\begin{thm}\label{thm 14.1.1}
		\textbf{Automorphisms of $\mathbb{H}$}. \\
		If $f \in Aut(\mathbb{H})$, then $\exists A \in SL_2(\R)$, $\st$
		\begin{align}
			f(z) = M_{A}(z)
		\end{align}
	\end{thm}




	%  ############################
	\ifx\allfiles\undefined
\end{document}
\fi